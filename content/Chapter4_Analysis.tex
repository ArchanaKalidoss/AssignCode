\chapter{Analysis of Resource-centric Organizational Modeling}
\label{chap:analysis}

This chapter positions the thesis work with respect to the other existing approach. The first section provides an overview about the properties of organizations as it helps to understand the concepts in the context of organizations. The second section provides detailed requirement analysis about the research objectives described in Chapter \ref{chap:introduction}. The third section provides a detailed literature review about the existing approaches. This review is used to evaluate the proposed approach in the following Chapter \ref{chap:approach}.



%%%%%%%%%%%%%%%%%%%%%%%%%%%%%%%%%%%%%%%%%%%%%%%%%%%%%%%%%%%%%%%%%%%%%%%%%
\section{Properties of Organizations}
\label{sec:propertiesorganization}
%%%%%%%%%%%%%%%%%%%%%%%%%%%%%%%%%%%%%%%%%%%%%%%%%%%%%%%%%%%%%%%%%%%%%%%%%
---A short description about prooperties of organizations---


%%%%%%%%%%%%%%%%%%%%%%%%%%%%%%%%%%%%%%%%%%%%%%%%%%%%%%%%%%%%%%%%%%%%%%%%%
\section{Requirement Analysis}
\label{sec:requirementssupoorting}
%%%%%%%%%%%%%%%%%%%%%%%%%%%%%%%%%%%%%%%%%%%%%%%%%%%%%%%%%%%%%%%%%%%%%%%%%
This section provides a detailed requirement analysis of research objectives mentioned in Chapter \ref{chap:introduction}. The below mentioned requirements are part of the functioning system proposed in the following Chapter \ref{chap:approach}. Also the exact phase when each of the requirements get satisfied are provided in Table \ref{tab:subrequirements}.


\textit{Organizational intention transparency (R1)}:  An intention can be broken down into definitive actionable components, or sub-intentions, upon which individual resources can act. When these lower level sub-intentions are made  achievable for individual resources, they can be combined to provide successful execution of higher level intention. Different organizational members can observe lower level and higher level intentions in their organizations. Intentions are traceable in the different levels of the organizational hierarchy. This means that the status of each intention can be accessed by members in different levels of the organizations. This kind of transparency within an organization reduces inefficiencies in intention execution, and is a key factor in attracting and retaining high  performers in the labor market \cite{McManus2007}. Requirement R1 has to be satisfied in the modeling phase itself as the designing of intentions, sub-intentions, strategies are done during the modeling time. 

\textit{Organizational intention resource-based cost estimation (R2)}:Linking intentions with strategies enable us a cost estimation for each intention. This is because intentions are realized through some strategies, strategies are associated with organizational capabilities which in turn has been associated with organizational resources. Cost is estimated in a recursive manner which has been explained in detail with an example in the following Chapter \ref{chap:approach}. To incorporate the cost estimation of intentions, we have to understand the recursive structure of the intentions associated with strategies. Since intentions are defined hierarchically, they can contain and extend intentions. Here strategy represents a means for achieving the intention. Further on, the cost of a strategy can be analyzed using the costs of derived sub-intentions, process definitions and so on. Including resources cost in intention cost calculation is important. This is achieved by associating resource models' cost with process models' cost. The recursion is stopped when the intention derivation process reaches the operational level. At the moment a  intention is achieved, some resources should be allocated to maintain the desired state (intention maintenance costs)\cite{Mandic2010}. Allocation of resources is mainly done at the operational level,hence requirement R2 has to be satisfied during the deployment phase.

\textit{Organizational intention achievability estimation (R3)}: The sub-intentions are projections of their super intentions, and satisfaction of the sub-intentions ensures satisfaction of the super intentions. Hence validity of an organizational intention is achievable when the intentions can be refined by defining sub-intentions, which can then be defined recursively as strategy and then to independent informal process models. Lower-level requirements can be validated against higher-level intentions, thus enabling validation of strategic alignment of  higher level intentions. The objectives of business strategy are found in the highest levels of the intention model.\cite{Bleistein2006}.Requirement R3  can be found during the modeling time of the process models as intention achieveability estimations are done before starting the execution of the intention based on the related intentions.

\textit{Intention oriented working style (R4)}: As each member of the organization is aware of the higher level and lower level intentions and he can engage for these explicit intentions. Intention orientation is the degree to which a person or organization focuses on tasks and the end results of those tasks. Strong intention orientation advocates a focus on the ends that the tasks are made for instead of the tasks themselves and how those ends will affect either the person or the entire company. Those with strong intention orientation will be able to accurately judge the effects of reaching the intention as well as the ability to fulfill that particular intention with current resources and skills \cite{Lacom}. The distinction between explicit knowledge of each sub intentions should not be seen as a division but rather as a continuum which aligns towards achieving the higher level intention . Though Requirement R4 itself has sub-requirement of R1, R4 has to be done at the run time which makes it distinct from the Requirement 1.

 \textit{Social organizational modeling (R5)}: Different members of an organization participate to create organizational intentions, as a result intentions are shaped based on all members but directed by the executives. The  social  extension  of  a  business  process  can  be  regarded  as  a  process optimization phase, where the organization seeks efficiency  by  extending  the  reach  of  a  business  process  to  a  broader  class  of  stakeholders\cite{Brambilla2012}.Requirement R5 would be done at the run time as the input from different members of the organization provided during the process execution. But the list of participants who can have the priviliges such as own/edit/follow/view access to the models can be determined beforehand.
 
 \textit{Re-use of organizational knowledge (R6)}: Intentions specific solutions can be extracted as abstract re-usable entities, organizational strategy patterns and can be re-used in multiple context definitions. These field tested solutions are made as descriptive model informations which can be re-used.  Re-using the informations as models from the previous executions trims out the model designing time \cite{Yu2000}.


\begin{table} [htbp]
	\centering
	\begin{tabular} {p{4cm}p{3cm}p{9cm}}
		\toprule
		\textbf{Requirements}                                                      & \textbf{Requirement Satisfaction Phase} & \textbf{Pre-requisites}    \\
		\midrule                                                                                                               
		Requirement 1 (R1)                    & Modeling phase                 &\begin{tabular}[c]{@{}l@{}}1. Main intention can be refinable into sub-intentions.\\ 2. Organizational members can view the intentions \\ at different levels.\end{tabular}                \\ 
		
		Requirement 2 (R2)                      & Deployment phase               &\begin{tabular}[c]{@{}l@{}}1. intention cost estimation that includes all recursive \\ sub-intentions and resources.\\ 2. Cost estimation including the strategy. \end{tabular}                \\         
		
		
		Requirement 3 (R3)                     & Modeling phase            &\begin{tabular}[c]{@{}l@{}}1. Each sub-intention should be achievable and valid.     \end{tabular}                \\      
		
		Requirement 4 (R4)                     & Deployment phase               &\begin{tabular}[c]{@{}l@{}}1. Satisfaction of R1.\\  2.  Understanding of the intentions and \\ how they can be reached.   \end{tabular}                \\                         
		
		
		
		Requirement 5 (R5)                      &Both Modeling phase and Deployment phase              &\begin{tabular}[c]{@{}l@{}}1. Satisfaction of R1.\\  2. The output of intention is based on the inputs \\ provided by different members of the organization.   \end{tabular}                \\ 
		
		Requirement 6 (R6)                      & Modeling phase              &\begin{tabular}[c]{@{}l@{}}1. Satisfaction of R1,R2,R3,R4 and R5\\     \end{tabular}                \\     
		
		\bottomrule
	\end{tabular}
	\caption{Requirements Satisfaction Phase}
	\label{tab:subrequirements}
\end{table}

%%%%%%%%%%%%%%%%%%%%%%%%%%%%%%%%%%%%%%%%%%%%%%%%%%%%%%%%%%%%%%%%%%%%%%%%%
\section{Literature Review}
\label{sec:literaturereview}
%%%%%%%%%%%%%%%%%%%%%%%%%%%%%%%%%%%%%%%%%%%%%%%%%%%%%%%%%%%%%%%%%%%%%%%%%




In the literature, several work has been done in order to support and automate the processes of dynamic in nature whose execution steps cannot be determined beforehand \cite{Dustdar2004,Herrmann2011}. The approach \textit{Adaptive Case Management}, proposed by Hermann et. al \cite{Herrmann2011} bridges the gap between business processes management and flexibility in adapting knowledge intensive processes by defining activities and re-using created activity structure. When the required activities changes dynamically, capturing them for re-use are not helpful \cite{Sungur2015}. Though the approach \textit{Ad-hoc and Collaborative Processes} proposed by Dustdar et. al. overcomes the challenges in process aware collaborations, defining activities in a ad-hoc fashion does not support human actor in various cases \cite{Sungur2015}. Also the approach proposed in Chapter \ref{chap:approach} serves as a complementary to the above discussed two approaches.  This is accomplished by enabling required resources in different activities.   

