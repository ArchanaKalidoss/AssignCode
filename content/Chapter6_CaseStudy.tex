\chapter{Case Study on Resource-centric Organizational Modeling}
\label{chap:casestudy}
In this chapter, we provide architecture of the functioning system as a first section. This section provides, implementation details along with the reason for making certain decisions regarding the implementation. The final section validates the system by validating it with the proposed approach. This section also has some requirement evaluation with the state of the art approaches.

%%%%%%%%%%%%%%%%%%%%%%%%%%%%%%%%%%%%%%%%%%%%%%%%%%%%%%%%%%%%%%%%%%%%%%%%%
\section{Architecture of the Functioning System}
\label{sec:architectureofthefunctioningsystem}
%%%%%%%%%%%%%%%%%%%%%%%%%%%%%%%%%%%%%%%%%%%%%%%%%%%%%%%%%%%%%%%%%%%%%%%%%
As discussed in the Chapter \ref{chap:approach}, informal process targets for accomplishment of an intention. Thus in the Figure \ref{fig:architectureofthecasestudy}, we associate intentions with both process definitions and strategy definitions as intention definitions are used by both process definitions and strategy definitions. Intentions are associated through resources either through strategies or through informal processes. Intentions can be refined by defining sub-intentions, which can be defined recursively as independent informal process. For example, in our motivating scenario the main intention increase revenue and number of unit sales can be refined into sub-intention of improve customer help desk portal. This sub-intention can be associated with process models. This \textit{intention-based} approach enables describing process declaratively,  i.e., without describing \textit{how} the intention is achieved, and providing information about \textit{what} has to be achieved. This avoids the need for predefined business logic in the representations of informal process \cite{Sungur2014a}. 



\begin{figure}
	\centering
	\includegraphics [width= \textwidth]{architectureofthecasestudy.pdf}
	\caption{Architecture of the Functioning System}
		\label{fig:architectureofthecasestudy}
	\end{figure}

%%%%%%%%%%%%%%%%%%%%%%%%%%%%%%%%%%%%%%%%%%%%%%%%%%%%%%%%%%%%%%%%%%%%%%%%%
\subsection{Application Flow}
\label{subsec:applicationflow}
%%%%%%%%%%%%%%%%%%%%%%%%%%%%%%%%%%%%%%%%%%%%%%%%%%%%%%%%%%%%%%%%%%%%%%%%%
Figure \ref{fig:UIArchitecture}

Each entity item has basic properties such as \textit{name} and \textit{target namespace}. The entities are identified using their unique id which is generated using the combination of name and target namespace. Other entities that are associated with a particular entity are resolved through this unique identifier. For example, in our motivating scenario consider the intention \textit{improve the customer help desk portal} when creating model for this intention, business expert provide name and namespace for this intention and add it to the database. A unique identifier is generated for the intention model using the combination of name and namespace by the system. The strategy (in our scenario \textit{through application  development}) that is associated with this intention, just contains only this unique identifer for the reference. 

\begin{figure}
	\centering
	\includegraphics [width= \textwidth]{UIArchitecture.pdf}
	\caption{User interface URL navigation of the functioning system}
	\label{fig:UIArchitecture}
\end{figure}

%%%%%%%%%%%%%%%%%%%%%%%%%%%%%%%%%%%%%%%%%%%%%%%%%%%%%%%%%%%%%%%%%%%%%%%%%
\section{A Concrete View of Entity Types}
\label{sec:concreteviewofentitytypes}
%%%%%%%%%%%%%%%%%%%%%%%%%%%%%%%%%%%%%%%%%%%%%%%%%%%%%%%%%%%%%%%%%%%%%%%%%
It is important to discuss the concrete concepts of informal process from an organizational aspect, because they have a direct effect on the outcome of the informal process \cite{Sungur2014}. This section discusses about how Intention-centric organizational modeling realized the user interface editor from an organizational aspect by taking the motivating scenario discussed in Chapter \ref {chap:motivatingScenario}. Though developing schema definitions are not part of the thesis implementation, it has been provided because the editor has a view that is capable of adding, viewing, deleting and updating model data aligned with the schema definition. A typical XML Schema Definition of entity type has been provided in the listing \ref{lst:xsdlist}.

\begin{Listing}
	\begin{lstlisting}
	<xs:complexType name="tEntityType" abstract="true">
	<xs:complexContent>
	<xs:extension base="tExtensibleElements">
	<xs:sequence>
	<xs:element name="Tags" type="tTags" minOccurs="0"/>
	<xs:element name="DerivedFrom" minOccurs="0">
	<xs:complexType>
	<xs:attribute name="typeRef" type="xs:QName" use="required"/>
	</xs:complexType>
	</xs:element>
	<xs:element name="PropertiesDefinition" minOccurs="0">
	<xs:complexType>
	<xs:attribute name="element" type="xs:QName"/>
	<xs:attribute name="type" type="xs:QName"/>
	</xs:complexType>
	</xs:element>
	</xs:sequence>
	<xs:attribute name="name" type="xs:NCName" use="required"/>
	<xs:attribute name="abstract" type="tBoolean" default="no"/>
	<xs:attribute name="final" type="tBoolean" default="no"/>
	<xs:attribute name="targetNamespace" type="xs:anyURI" use="optional"/>
	</xs:extension>
	</xs:complexContent>
	</xs:complexType>>
	\end{lstlisting}
	\caption{XML Schema Definition of Entity Type}
	\label{lst:xsdlist}
\end{Listing}

%%%%%%%%%%%%%%%%%%%%%%%%%%%%%%%%%%%%%%%%%%%%%%%%%%%%%%%%%%%%%%%%%%%%%%%%%
\section{Realization of Motivating Scenario}
\label{sec:realization}
%%%%%%%%%%%%%%%%%%%%%%%%%%%%%%%%%%%%%%%%%%%%%%%%%%%%%%%%%%%%%%%%%%%%%%%%%
 Figure \ref{fig:realizationofmotivatingscenario}

\begin{figure}
	\centering
	\includegraphics[width=\textwidth,angle=0]{PhasesofMotivatingScenario.pdf}
	\caption{Realization of Motivating Scenario}
	\label{fig:realizationofmotivatingscenario}
\end{figure}

%%%%%%%%%%%%%%%%%%%%%%%%%%%%%%%%%%%%%%%%%%%%%%%%%%%%%%%%%%%%%%%%%%%%%%%%%
\subsection{Realization of Context Definitions}
%%%%%%%%%%%%%%%%%%%%%%%%%%%%%%%%%%%%%%%%%%%%%%%%%%%%%%%%%%%%%%%%%%%%%%%%%
Each informal process start from an initial context, i.e., IPE Context and aims to achieve an intention, i.e., an IPE Intention \cite{Sungur2014a}. After reaching an intention, there is resulting IPE Context. In the motivating scenario  

%%%%%%%%%%%%%%%%%%%%%%%%%%%%%%%%%%%%%%%%%%%%%%%%%%%%%%%%%%%%%%%%%%%%%%%%%
\subsection{Realization of Intention Definitions}
%%%%%%%%%%%%%%%%%%%%%%%%%%%%%%%%%%%%%%%%%%%%%%%%%%%%%%%%%%%%%%%%%%%%%%%%%


%%%%%%%%%%%%%%%%%%%%%%%%%%%%%%%%%%%%%%%%%%%%%%%%%%%%%%%%%%%%%%%%%%%%%%%%%
\subsection{Realization of Strategy Definitions}
%%%%%%%%%%%%%%%%%%%%%%%%%%%%%%%%%%%%%%%%%%%%%%%%%%%%%%%%%%%%%%%%%%%%%%%%%

%%%%%%%%%%%%%%%%%%%%%%%%%%%%%%%%%%%%%%%%%%%%%%%%%%%%%%%%%%%%%%%%%%%%%%%%%
\subsection{Realization of Capabilitiy Definitions}
%%%%%%%%%%%%%%%%%%%%%%%%%%%%%%%%%%%%%%%%%%%%%%%%%%%%%%%%%%%%%%%%%%%%%%%%%
There are two types of capabilities. Functional capabilities and cross-functional capabilities. Functional capabilities must be associated with instance descriptors. Cross-functional capabilities are capabilities containing multiple functional capabilities. We need to have the ability to add and remove instance descriptors for an entity type, e.g, resource definitions,  informal process definitions, etc. An instance descriptor of a functional capability should refer to a resource definition meaning that a capability is provided by a resource definition. So an instance descriptor of a capability refers to a resource definition and we can manually add and remove resource definitions in general.




%%%%%%%%%%%%%%%%%%%%%%%%%%%%%%%%%%%%%%%%%%%%%%%%%%%%%%%%%%%%%%%%%%%%%%%%%
\subsection{Realization of Resource Definitions}
%%%%%%%%%%%%%%%%%%%%%%%%%%%%%%%%%%%%%%%%%%%%%%%%%%%%%%%%%%%%%%%%%%%%%%%%%
As discussed earlier each resource can be related to another resource which are defined using predefined or custom \textit{relationships} \cite{Sungur2014a}. These resources are managed through \textit{Resource Organizers}, this is because resource organizers are used to bring together the relevant interrelated resources that work towards to achieve the corresponding intentions. TOSCA \cite{Binz2014} can be used to model all nodes and relationship among them. In our context, we can consider resources as nodes to make use of the TOSCA's service. The schema definition of considering each resource as node is provided in the listing \ref{lst:xsdnodetype}. In the developed editor, the resource models are managed by embedding  the open source modeling tool Winery web page \cite{Kopp2013} in our editor's web page. This is because, it creates a new service template that contains an application topology by using the topology modeler. Winery also offers all available node types in a palette. From there, the user drags the desired node type and drops it into the editing area. There, the node type
becomes a node template: a node in the topology graph. Node templates can be annotated with requirements and capabilities, property values, and policies. 

In order to achieve this we use tosca repository url referring to winery and the other one referring to topology modeler of the winery. Using these values we create corresponding url required for our modeling based on the name and namespace properties of an entity. The functionality to generate resource model page, using tosca repository url and topology modeler url is provided below.

\fbox{
	\begin{minipage}{\textwidth}
		\{topology-modeler-url\}?repositoryURL=\{encoded-tosca-repository-url\}\&ns=\{encoded-target-namepsace\}\&id=\{encoded-id\}\#
	\end{minipage}
	}
			
\begin{Listing}
	\begin{lstlisting}
	<xs:complexType name="tNodeTemplate">
	<xs:complexContent>
	<xs:extension base="tEntityTemplate">
	<xs:sequence>
	<xs:element name="Requirements" minOccurs="0">
	<xs:complexType>
	<xs:sequence>
	<xs:element name="Requirement" type="tRequirement" maxOccurs="unbounded"/>
	</xs:sequence>
	</xs:complexType>
	</xs:element>
	<xs:element name="Capabilities" minOccurs="0">
	<xs:complexType>
	<xs:sequence>
	<xs:element name="Capability" type="tCapability" maxOccurs="unbounded"/>
	</xs:sequence>
	</xs:complexType>
	</xs:element>
	<xs:element name="Policies" minOccurs="0">
	<xs:complexType>
	<xs:sequence>
	<xs:element name="Policy" type="tPolicy" maxOccurs="unbounded"/>
	</xs:sequence>
	</xs:complexType>
	</xs:element>
	<xs:element name="DeploymentArtifacts" type="tDeploymentArtifacts" minOccurs="0"/>
	</xs:sequence>
	<xs:attribute name="name" type="xs:string" use="optional"/>
	<xs:attribute name="minInstances" type="xs:int" use="optional" default="1"/>
	<xs:attribute name="maxInstances" use="optional" default="1">
	<xs:simpleType>
	<xs:union>
	<xs:simpleType>
	<xs:restriction base="xs:nonNegativeInteger">
	<xs:pattern value="([1-9]+[0-9]*)"/>
	</xs:restriction>
	</xs:simpleType>
	<xs:simpleType>
	<xs:restriction base="xs:string">
	<xs:enumeration value="unbounded"/>
	</xs:restriction>
	</xs:simpleType>
	</xs:union>
	</xs:simpleType>
	</xs:attribute>
	</xs:extension>
	</xs:complexContent>
	</xs:complexType>
	\end{lstlisting}
	\caption{XML Schema Definition of Node Type}
	\label{lst:xsdnodetype}
\end{Listing}
		
%%%%%%%%%%%%%%%%%%%%%%%%%%%%%%%%%%%%%%%%%%%%%%%%%%%%%%%%%%%%%%%%%%%%%%%%%
\section{Validation}
\label{sec:validation}
%%%%%%%%%%%%%%%%%%%%%%%%%%%%%%%%%%%%%%%%%%%%%%%%%%%%%%%%%%%%%%%%%%%%%%%%%	
This section validates the degree of satisfaction of the research objectives discussed in Chapter \ref{chap:introduction} against the developed prototype through. Also, we claim that this master thesis is a part of creating models that are required for supporting and automating informal processes, it is important to evaluate the developed prototype along with the requirements that are discussed in the approach \textit{Informal Process Essentials} \cite{Sungur2014a}. In this section, examples are provided from motivating scenario which is discussed in the Chapter \ref{chap:motivatingScenario}. The concept of \textit{resource-centric} modeling approach has also been validated in the approach \textit{Informal Process Essentials} \cite{Sungur2014a}, where the author describes that the approach is right one since the focus is not on business logic rather on other dimensions like resources. The author also states that non-existence of business logic facilitates more autonomy for human performers and enables establishment of best practices. Since the above arguments justifies to the fact of providing more autonomous informal process modeling, one can claim that the approach of \textit{resource-centric modeling} is a valid one. Not stopping with these arguments, we also provide a detailed validation of research objectives discussed in Chapter \ref{chap:introduction} and validation of developed prototype with suitable examples.
		
%%%%%%%%%%%%%%%%%%%%%%%%%%%%%%%%%%%%%%%%%%%%%%%%%%%%%%%%%%%%%%%%%%%%%%%%%
\subsection{Validation of Research Objectives}
\label{subsec:validationofrequirements}
%%%%%%%%%%%%%%%%%%%%%%%%%%%%%%%%%%%%%%%%%%%%%%%%%%%%%%%%%%%%%%%%%%%%%%%%%
As discussed in Chapter \ref{chap:approach}, the research objectives are satisfied at the design level but their validity can be confirmed only by evaluating the research objectives with some sample scenarios provided in Chapter \ref{chap:motivatingScenario}.   

\textit{Organizational intentions transparency} (R1): A valid user whose credentials are stored in database is able to login successfully and view the intentions and its associated entities. Hence the research objective R1 is met.

\textit{Organizational intention resource-based cost estimation} (R2): An intention whose cost is unspecified for a sample intention, is calculated by the developed system recursively as mentioned in the Chapter \ref{chap:approach}. Thus the research objective R2 is also met.

\textit{Organizational intention achievability estimation} (R3): Similar to cost calculation, an intention instance whose achievability is not in prior is also estimated by the current functioning system. Hence research objective R3 is satisifed.

\textit{Intention oriented working style} (R4): The users can login and create intention models, strategy models, informal process models etc., through the developed editor. Hence research objective R4 is also met.

\textit{Participative organizational modeling} (R5): Each entity type that can be interactively acquirable has list of participants with their corresponding privileges. Thus this satisfies the requirements of research objective R5.

\textit{Re-use of organizational knowledge} (R6): The descriptive information about each models can be stored and re-used for next enactments. Hence research objective R6 is also met.
	
%%%%%%%%%%%%%%%%%%%%%%%%%%%%%%%%%%%%%%%%%%%%%%%%%%%%%%%%%%%%%%%%%%%%%%%%%
\subsection{Validation of Prototype}
\label{subsec:validationofprototype}
%%%%%%%%%%%%%%%%%%%%%%%%%%%%%%%%%%%%%%%%%%%%%%%%%%%%%%%%%%%%%%%%%%%%%%%%%
In the approach of \textit{Supporting Informal Processes} \cite{Sungur2014}, the author has categorized generic requirements that supports enactment of informal processes under three dimensions such as business logic, IT infrastructure and organization. In order to make the validation procedure simple, we have taken the concrete requirements discussed in the approach of \textit{Informal Process Essentials} \cite{Sungur2014a}. This is because the latter approach itself is an extended work of former approach.  

\textit{Enactable Informal Process Representation}: In this requirement, the core elements are performers, IT tools, data etc., and the requirement gets satisfied only when we are able to provide textual descriptions of how to make these ready. In the functioning web editor, the user can create textual information i.e models required for resources, contexts, strategies etc. Hence the developed prototype satisfies the requirement of providing enactable informal process representation. For example, using our motivating scenario we only provide definitions of intentions, contexts, strategies, capabilities and resources inside the editor but there is no functionality to  predefine business logic of these informal processes.

\textit{Resource Relationships Definition}: In an informal process, each resource can have a relation with other resource. For example in our motivating scenario, a resource with front-end developer capability has a "requires" relationship with front-end developer tools. In the functioning system, Winery modeling tool's repository web page has been included to edit the resource models. Thus the functioning system also satisfies the requirement of defining relationship between the resources. 

\textit{Resource Visibility Definition}: Informal processes contains resources that work towards the process' specific intention. These resources can participate in more than one informal process. For example, in the same organization as in our motivating scenario there can be another process working towards achieving an intention say \textit{improving skills of all the employees}. This can make an employee with a developer capability to participate in both the processes. Thus all the resources of an informal process has to be visible. In or functioning system we have septate user interface tab that details associated resources of an informal process. This satisfies the requirement of making the resource definitions visible.    

\textit{Support for Dynamically Changing Resources}: Due to dynamic nature of informal processes, the developed editor provides facility to add or remove resources. For example, in our motivating scenario consider the sub-intention of \textit{improving the help desk portal} where a new requirement of \textit{extending the help desk portal support in mobiles} may arise dynamically. In this situation, the editor must provide means to add new resource with new capability of \textit{mobile application developer capability}. The functioning system provides facility to add or remove resources associated with capabilities. Thus the requirement of providing support for dynamically changing resources is also satisfied by the editor. 
	


