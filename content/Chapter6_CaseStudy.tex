\chapter{Case Study}
\label{chap:casestudy}

%of the system: Class diagrams (components), ER, algorithms if there are any
%%Presentation of the functioning system and culmination of the work.
%%Analysis of the work as well as a comparison with the state of the art



%%%%%%%%%%%%%%%%%%%%%%%%%%%%%%%%%%%%%%%%%%%%%%%%%%%%%%%%%%%%%%%%%%%%%%%%%
\section{Realization}
\label{sec:realization}
%%%%%%%%%%%%%%%%%%%%%%%%%%%%%%%%%%%%%%%%%%%%%%%%%%%%%%%%%%%%%%%%%%%%%%%%%
%how did I implement, technology and frameworks used and why.

 This chapter discusses about how Intention-centric organizational modeling realized as an user interface editor. This also explains the realization of requirements described in Chapter 4. In the first section, we will first discuss the overview of basic concepts and then specify the design components required for the realization of this editor. The next section discusses about how the entity views in particular are realized. The last section covers in detail how individual requirement has been realized through the web editor.




%%%%%%%%%%%%%%%%%%%%%%%%%%%%%%%%%%%%%%%%%%%%%%%%%%%%%%%%%%%%%%%%%%%%%%%%%
\subsection{Realization of Entity Views}
\label{subsec:realofentityviews}
%%%%%%%%%%%%%%%%%%%%%%%%%%%%%%%%%%%%%%%%%%%%%%%%%%%%%%%%%%%%%%%%%%%%%%%%%
The XML Schema Definition of entity type has been provided in \ref{lst:xsdlist}



\begin{Listing}
	\begin{lstlisting}
	<xs:complexType name="tEntityType" abstract="true">
	<xs:complexContent>
	<xs:extension base="tExtensibleElements">
	<xs:sequence>
	<xs:element name="Tags" type="tTags" minOccurs="0"/>
	<xs:element name="DerivedFrom" minOccurs="0">
	<xs:complexType>
	<xs:attribute name="typeRef" type="xs:QName" use="required"/>
	</xs:complexType>
	</xs:element>
	<xs:element name="PropertiesDefinition" minOccurs="0">
	<xs:complexType>
	<xs:attribute name="element" type="xs:QName"/>
	<xs:attribute name="type" type="xs:QName"/>
	</xs:complexType>
	</xs:element>
	</xs:sequence>
	<xs:attribute name="name" type="xs:NCName" use="required"/>
	<xs:attribute name="abstract" type="tBoolean" default="no"/>
	<xs:attribute name="final" type="tBoolean" default="no"/>
	<xs:attribute name="targetNamespace" type="xs:anyURI" use="optional"/>
	</xs:extension>
	</xs:complexContent>
	</xs:complexType>>
	\end{lstlisting}
	\caption{XML Schema Definition of Entity Type}
	\label{lst:xsdlist}
\end{Listing}

%%%%%%%%%%%%%%%%%%%%%%%%%%%%%%%%%%%%%%%%%%%%%%%%%%%%%%%%%%%%%%%%%%%%%%%%%
\subsection{Realization of Organizational Intentions}
%%%%%%%%%%%%%%%%%%%%%%%%%%%%%%%%%%%%%%%%%%%%%%%%%%%%%%%%%%%%%%%%%%%%%%%%%


%%%%%%%%%%%%%%%%%%%%%%%%%%%%%%%%%%%%%%%%%%%%%%%%%%%%%%%%%%%%%%%%%%%%%%%%%
\subsection{Realization of Organizational  Strategies}
%%%%%%%%%%%%%%%%%%%%%%%%%%%%%%%%%%%%%%%%%%%%%%%%%%%%%%%%%%%%%%%%%%%%%%%%%

%%%%%%%%%%%%%%%%%%%%%%%%%%%%%%%%%%%%%%%%%%%%%%%%%%%%%%%%%%%%%%%%%%%%%%%%%
\subsection{Realization of Organizational Capabilities}
%%%%%%%%%%%%%%%%%%%%%%%%%%%%%%%%%%%%%%%%%%%%%%%%%%%%%%%%%%%%%%%%%%%%%%%%%
There are two types of capabilities. Functional capabilities and cross-functional capabilities. Functional capabilities must be associated with instance descriptors. Cross-functional capabilities are capabilities containing multiple functional capabilities. We need to have the ability to add and remove instance descriptors for an entity type, e.g, resource definitions,  informal process definitions, etc. An instance descriptor of a functional capability should refer to a resource definition meaning that a capability is provided by a resource definition. So an instance descriptor of a capability refers to a resource definition and we can manually add and remove resource definitions in general.




%%%%%%%%%%%%%%%%%%%%%%%%%%%%%%%%%%%%%%%%%%%%%%%%%%%%%%%%%%%%%%%%%%%%%%%%%
\subsection{Realization of Organizational Resources}
%%%%%%%%%%%%%%%%%%%%%%%%%%%%%%%%%%%%%%%%%%%%%%%%%%%%%%%%%%%%%%%%%%%%%%%%%
There are two keys, one for tosca repository and one for tosca modeling tool. Tosca repository url refers to winery and the other one refers to topology modeler. This should only contain root contexts and additional suffixes such as service templates in a separate key in default db.Out of these a function should create corresponding url for topology modeling. The topology modeler page of a specific service temple without knowing its id and namespace. This can be composed from different variables: :topology-modeler-url, :tosco-repository-url,:target-namespace, and :id.


\fbox{\begin{minipage}{\textwidth}
		\{topology-modeler-url\}?repositoryURL=\{encoded-tosca-repository-url\}\&ns=\{encoded-target-namepsace\}\&id=\{encoded-id\}\#
	\end{minipage}}
	
		
What important is, at the end editor should have a view that is capable of adding, viewing, deleting and updating models aligned with the XSD schema. In general, separate entities should reference each other but not contain each other. For instance, a strategy containing a goal should use goals id to resolve it but not the actual goal.
	
%%%%%%%%%%%%%%%%%%%%%%%%%%%%%%%%%%%%%%%%%%%%%%%%%%%%%%%%%%%%%%%%%%%%%%%%%
	\subsection{Realization of Organizational Processes}
%%%%%%%%%%%%%%%%%%%%%%%%%%%%%%%%%%%%%%%%%%%%%%%%%%%%%%%%%%%%%%%%%%%%%%%%%
	
	
%%%%%%%%%%%%%%%%%%%%%%%%%%%%%%%%%%%%%%%%%%%%%%%%%%%%%%%%%%%%%%%%%%%%%%%%%
	\subsection{Realization of Informal Process Instances}
%%%%%%%%%%%%%%%%%%%%%%%%%%%%%%%%%%%%%%%%%%%%%%%%%%%%%%%%%%%%%%%%%%%%%%%%%
	Informal process instances are separate entities similar to informal process models. They should be listed on the left side similar to informal process models. When one of the instance is selected by user, its properties should be presented on the right.
	
	
	
	
	
%%%%%%%%%%%%%%%%%%%%%%%%%%%%%%%%%%%%%%%%%%%%%%%%%%%%%%%%%%%%%%%%%%%%%%%%%
	\subsection{Realization of Requirements}
	\label{subsec:realizationofrequirements}
%%%%%%%%%%%%%%%%%%%%%%%%%%%%%%%%%%%%%%%%%%%%%%%%%%%%%%%%%%%%%%%%%%%%%%%%%
	
%%%%%%%%%%%%%%%%%%%%%%%%%%%%%%%%%%%%%%%%%%%%%%%%%%%%%%%%%%%%%%%%%%%%%%%%%
	\section{Validation}
	\label{sec:evaluation}
%%%%%%%%%%%%%%%%%%%%%%%%%%%%%%%%%%%%%%%%%%%%%%%%%%%%%%%%%%%%%%%%%%%%%%%%%
	%queries and print screens

	
%%%%%%%%%%%%%%%%%%%%%%%%%%%%%%%%%%%%%%%%%%%%%%%%%%%%%%%%%%%%%%%%%%%%%%%%%
	\subsection{Validation of Components}
	\label{subsec:validationofcomponents}
%%%%%%%%%%%%%%%%%%%%%%%%%%%%%%%%%%%%%%%%%%%%%%%%%%%%%%%%%%%%%%%%%%%%%%%%%
	
	
	
	
%%%%%%%%%%%%%%%%%%%%%%%%%%%%%%%%%%%%%%%%%%%%%%%%%%%%%%%%%%%%%%%%%%%%%%%%%
	\subsection{Validation of Prototype}
	\label{subsec:validationofprototype}
%%%%%%%%%%%%%%%%%%%%%%%%%%%%%%%%%%%%%%%%%%%%%%%%%%%%%%%%%%%%%%%%%%%%%%%%%
	
	


