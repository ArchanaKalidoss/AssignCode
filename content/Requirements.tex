\chapter{Requirements}
\label{chap:requirements}
%how do you position your work w.r.t. existing, APPROX. 3 sections

%%%%%%%%%%%%%%%%%%%%%%%%%%%%%%%%%%%%%%%%%%%%%%%%%%%%%%%%%%%%%%%%%%%%%%%%%
\section{Properties of Organizations}
\label{sec:propertiesorganization}
%%%%%%%%%%%%%%%%%%%%%%%%%%%%%%%%%%%%%%%%%%%%%%%%%%%%%%%%%%%%%%%%%%%%%%%%%


%%%%%%%%%%%%%%%%%%%%%%%%%%%%%%%%%%%%%%%%%%%%%%%%%%%%%%%%%%%%%%%%%%%%%%%%%
\section{Requirements Supporting Organizational Modeling}
\label{sec:requirementssupoorting}
%%%%%%%%%%%%%%%%%%%%%%%%%%%%%%%%%%%%%%%%%%%%%%%%%%%%%%%%%%%%%%%%%%%%%%%%%

\hspace{4ex} \textbf{Organizational intention transparency (R1)}:  A intention can be broken down into definitive actionable components, or sub-intentions, upon which individual resources can act. When these lower level sub-intentions are made  achievable for individual resources, they can be combined to provide successful execution of higher level intention. Different organizational members can observe lower level and higher level intentions in their organizations. intentions are traceable in the different levels of the organizational hierarchy. This kind of transparency within an organization reduces inefficiencies in intention execution, and is a key factor in attracting and retaining high  performers in the labor market \cite{McManus2007}.Requirement R1 has to be satisfied in the modeling time of the process itself as the framing of intentions, sub-intentions, strategies are done during the modeling time .

\hspace{4ex} \textbf{Organizational intention resource-based cost estimation. (R2)}:Linking intentions with capabilities and with resources enable us a cost estimation for each intention. Cost is estimated in a recursive manner. To incorporate the cost estimation of intentions, we have to understand the recursive structure of the intentions associated with capabilities.Since intentions are defined hierarchically, they can contain and extend intentions. Here strategy represents a means for achieving the intention. Further on, the cost of a strategy can be analyzed using the costs of derived  intentions, and so on. Including resources cost in intention cost calculation is important. The recursion is stopped when the intention derivation process reaches the operational
level. At the moment a  intention is achieved, some resources should be allocated to maintain the desired state (intention maintenance costs)\cite{Mandic2010}. Allocation of resources is mainly done at the operational level, Requirement R2 has to be satisfied in the run time of the process.

\hspace{4ex} \textbf{Organizational intention achievability estimation. (R3)}: The sub-intentions are projections of their super intentions, and satisfaction of the sub-intentions ensures satisfaction of the super intentions. Hence validity of an organizational intention is achievable when the intentions can be refined by defining sub-intentions, which can then be defined recursively as independent informal processes. Lower-level requirements can be validated against higher-level intentions, thus enabling validation of strategic alignment of  higher level intentions. The objectives of business strategy are found in the highest levels of the intention model.\cite{Bleistein2006}.Requirement R3 has to be satisfied during the modeling time of the process as intention achieveability estimations are done before starting the execution of the intention.

\hspace{4ex} \textbf{Intention oriented working style (R4)}: As each member of the organization is aware of the higher level and lower level intentions and he can engage for these explicit intentions. intention orientation is the degree to which a person or organization focuses on tasks and the end results of those tasks. Strong intention orientation advocates a focus on the ends that the tasks are made for instead of the tasks themselves and how those ends will affect either the person or the entire company. Those with strong intention orientation will be able to accurately judge the effects of reaching the intention as well as the ability to fulfill that particular intention with current resources and skills \cite{Lacom}. The distinction between explicit knowledge of each sub intentions should not be seen as a division but rather as a continuum which aligns towards achieving the higher level intention . Thought Requirement R4 itself has sub-requirement of R1, R4 has to be done at the run time which makes it distinct from the Requirement 1.

\hspace{4ex} \textbf{Social organizational modeling. (R5)}: Different members of an organization participate to create organizational intentions, as a result intentions are shaped based on all members but directed by the executives. The  social  extension  of  a  business  process  can  be  regarded  as  a  process optimization phase, where the organization seeks efficiency  by  extending  the  reach  of  a  business  process  to  a  broader  class  of  stakeholders\cite{Brambilla2012}.Requirement R5 would be done at the run time as the input from different members of the organization provided during the process execution.


\begin{table} [htbp]
	\centering
	\begin{tabular} {p{4cm}p{3cm}p{9cm}}
		\toprule
		\textbf{Requirements}                                                      & \textbf{Requirement Satisfaction Phase} & \textbf{Pre-requisites}    \\
		\midrule                                                                                                               
		Requirement 1 (R1)                    & Modeling phase                 &\begin{tabular}[c]{@{}l@{}}1. Main intention can be refinable into sub-intentions.\\ 2. Organizational members can view the intentions \\ at different levels.\end{tabular}                \\ 
		
		Requirement 2 (R2)                      & Deployment phase               &\begin{tabular}[c]{@{}l@{}}1. intention cost estimation that includes all recursive \\ sub-intentions and resources.\\ 2. Cost estimation including the strategy. \end{tabular}                \\         
		
		
		Requirement 3 (R3)                     & Modeling phase            &\begin{tabular}[c]{@{}l@{}}1. Each sub-intention should be achievable and valid.     \end{tabular}                \\      
		
		Requirement 4 (R4)                     & Deployment phase               &\begin{tabular}[c]{@{}l@{}}1. Satisfaction of R1.\\  2.  Understanding of the intentions and \\ how they can be reached.   \end{tabular}                \\                         
		
		
		
		Requirement 5 (R5)                      & Modeling phase                 &\begin{tabular}[c]{@{}l@{}}1. Satisfaction of R1.\\  2. The output of intention is based on the inputs \\ provided by different members of the organization.   \end{tabular}                \\     
		
		\bottomrule
	\end{tabular}
	\caption{Sub Requirements}
	\label{tab:subrequirements}
\end{table}