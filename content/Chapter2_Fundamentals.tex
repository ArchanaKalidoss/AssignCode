\chapter{Fundamentals and Related Work}
\label{chap:fundamentals}
%background - What do I need to know to understand the thesis?

%%%%%%%%%%%%%%%%%%%%%%%%%%%%%%%%%%%%%%%%%%%%%%%%%%%%%%%%%%%%%%%%%%%%%%%%%
\section{Basic Concepts}
\label{sec:basicconcepts}
%%%%%%%%%%%%%%%%%%%%%%%%%%%%%%%%%%%%%%%%%%%%%%%%%%%%%%%%%%%%%%%%%%%%%%%%%

%%%%%%%%%%%%%%%%%%%%%%%%%%%%%%%%%%%%%%%%%%%%%%%%%%%%%%%%%%%%%%%%%%%%%%%%%
\section{Second Phase of InProcXec}
\label{sec:inproxec}
%%%%%%%%%%%%%%%%%%%%%%%%%%%%%%%%%%%%%%%%%%%%%%%%%%%%%%%%%%%%%%%%%%%%%%%%%

%%%%%%%%%%%%%%%%%%%%%%%%%%%%%%%%%%%%%%%%%%%%%%%%%%%%%%%%%%%%%%%%%%%%%%%%%
\section{Human Centric Process}
\label{sec:humancentric}
%%%%%%%%%%%%%%%%%%%%%%%%%%%%%%%%%%%%%%%%%%%%%%%%%%%%%%%%%%%%%%%%%%%%%%%%%


%%%%%%%%%%%%%%%%%%%%%%%%%%%%%%%%%%%%%%%%%%%%%%%%%%%%%%%%%%%%%%%%%%%%%%%%%
\section{Definitions of Terms}
\label{sec:termdefinitions}
%%%%%%%%%%%%%%%%%%%%%%%%%%%%%%%%%%%%%%%%%%%%%%%%%%%%%%%%%%%%%%%%%%%%%%%%%
\textit{Informal Processes}  The processes that human participate and create knowledge are called unstructured/informal/human-centric processes.        \\

\textit{Organizational Intentions} Intentions are defined hierarchically, which can contain and extend sub-intentions.It is depicted by a double circle. The sub-intentions are refined starting from main intentions. Intentions are associated with capabilities or resources. An accomplishment of an intention changes state. An intention can extend another intention.        \\


\textit{Organizational Capabilities} Organizational capability is the ability to provide business values like software applications, resources, and potential of the actor to make decisions even in changing situations \cite{Stirna2012}.Describes a capability provided by a resource or required by an intention.    \\



\textit{Organizational Resources}         \\


\textit{Strategies}         \\




%%%%%%%%%%%%%%%%%%%%%%%%%%%%%%%%%%%%%%%%%%%%%%%%%%%%%%%%%%%%%%%%%%%%%%%%%
\section{Resource-centric Organizational Modeling}
%%%%%%%%%%%%%%%%%%%%%%%%%%%%%%%%%%%%%%%%%%%%%%%%%%%%%%%%%%%%%%%%%%%%%%%%%
 The Organizational Modeling element notation has been selected as per the guidelines mentioned in the paper by Daniel L.Moody \cite{Moody2009}. Also by observing  the fact that business process modelers are already well-known with the present process modeling notations such as Business Process Modeling Notation 2.0 (BPMN) \cite{bpm2011} and ArchiMate notation\cite{arc2013}, the shape depiction of organizational model elements are designed similar to those existing process notations. 

 Due to the importance of shapes in expressing the information visually , the notations are chosen in such a way that each element of Organizational Modeling  differ by shape. Also a legend will be always shown in the modeling notation to denote the meaning of each shape \cite{Moody2009}. As shape plays a primary role in discriminating between different element, organizational model notations are represented through individual shapes like rectangle, double circle, elliptic etc.,. The description of each element in the Organizational Model Notation is shown in the Table \ref{tab:notations}. 

\begin{center}
	\begin{longtable}{p{3cm}p{10cm}p{3cm}}
		\toprule 
		\textbf{Element} & \textbf{Definition} & \textbf{Notation} \\
		\midrule
		\endfirsthead
		Intentions 			& Intentions are purposeful concrete steps taken to achieve expected outcomes . They reflect the actual intention of an organization. & \begin{center} \includegraphics[width= 0.07\textwidth]{intentions.png}  \end{center}  \\  
		
		Capabilities	&  Capabilites are represented by a elliptical circle. Capability is an ability that should be possessed by an actor or a resource that work towards achievement of intention.   & \begin{center} \includegraphics[width= 0.1\textwidth]{capabilities.png} \end{center}   \\
		
		Context				& The environment that forms the setting for an event, statement, or idea, and in terms of which it can be fully understood. There are two Contexts: Initial and Final. The Initial Context is the situation which describes the driving forces that trigger the process to start. The Final Context is the expected situation once the process has finished.Both initial and final context are represented by an hexagonal shape except the final context has thick edges than initial context.  & \begin{center} \includegraphics[width= 0.1\textwidth]{context.png} \end{center}  \\
		
		
		Strategy		&  A method or plan chosen to bring about a desired future, such as accomplishment of a intention. Strategies are expressed by rectangles with sharp edges. In the conceptual Organizational Modeling, strategies are self-contained and loosely coupled elements.   & \begin{center} \includegraphics[width= 0.1\textwidth]{strategy.png} \end{center}   \\
		
		Resources					& The people and tools needed to fulfill the middle objectives or those/that work towards the achievement of goal . Resources are represented by a rounded rectangle. Resources are linked to capabilities and actors. & \begin{center} \includegraphics[width= 0.1\textwidth]{resources.png} \end{center}   \\
		
		Actors					& People who participate in the process. Actors are represented by a stick-man and they are linked to resource as actors can be resources. Actors define the strategy and goals.  & \begin{center} \includegraphics[width= 0.07\textwidth]{actor.png} \end{center}   \\
		
		Relationship				& A relationship is used specify the fixed links between the elements of the model. Relationship between two elements is represented by a single direction line which represents a sequence.  & \begin{center} \includegraphics[width= 0.1\textwidth]{relationship.png} \end{center}   \\
		
		
		\bottomrule
		\caption{Informal Process Modeling Notation}
		\label{tab:notations}		
	\end{longtable}	
\end{center}



%%%%%%%%%%%%%%%%%%%%%%%%%%%%%%%%%%%%%%%%%%%%%%%%%%%%%%%%%%%%%%%%%%%%%%%%%
\subsection{Process Representation}
%%%%%%%%%%%%%%%%%%%%%%%%%%%%%%%%%%%%%%%%%%%%%%%%%%%%%%%%%%%%%%%%%%%%%%%%%
Organizational Process Modeling depicted inFigure \ref{fig:processdiagram} captures required organizational capabilities that are satisfied by resource models  to enable the achievement of organizational goals in certain context definitions through a strategy. It is a top-down approach, i.e., first goals are defined and then sub-goals  are defined by refining main goal. Goals connect initial context definitions with final context definitions through a strategy.  To understand the definition of Organizational Process Modeling we need to interpret the Organizational Process Modeling Representation shown in Figure\ref{fig:processdiagram}. 

 The Organizational Process Modeling start with modeling of organizational goal (M1). Once the goal has been modeled, the second step is to model the strategies which can be a multi-instance strategy model(M2). The next step is to model the context definitions (M3.1), required organizational capabilities (M3.2) and refining the sub-goals from main goal in parallel. Once the required capabilities(M4) are matched by required resources(P1), modeling of resources(M5) can be done.  Based on created resource models (M5) and modeled context definitions(M3.1), strategies can be executed. The organizational goals would be iteratively updated supported to strategy execution.  


\begin{figure}
	\centering
	\includegraphics[width=\textwidth]{processmodeling.pdf}
	\caption{Process Modeling Diagram}
	\label{fig:processdiagram}
\end{figure}

%%%%%%%%%%%%%%%%%%%%%%%%%%%%%%%%%%%%%%%%%%%%%%%%%%%%%%%%%%%%%%%%%%%%%%%%%
\subsection{Entity Representation}
%%%%%%%%%%%%%%%%%%%%%%%%%%%%%%%%%%%%%%%%%%%%%%%%%%%%%%%%%%%%%%%%%%%%%%%%%

 The conceptual entity model of goals is shown in the \ref{fig:metamodel}. This model shows that top level goal is refined into sub-goals. A goal can be achieved through a strategy which is a plan of action designed to meet a goal. It also describes a set of interrelated resources which work together to achieve a collective goal. As reported by Sungur et al. \cite{Sungur2014a}, the concept of IPE provides an agent-based approach i.e., human performers are considered as agents who execute the processes autonomously. Based on the approach \cite{Sungur2014a} we provide a goal-oriented approach based on goals.

 Organizational Process Modeling  has \textit{Resources} which are used to achieve the goals. Organizational Process Modeling is Resource-centric approach as they support processes by providing required resources and thrives to successfully execute the processes by using qualified autonomous agents, i.e., actors under certain \textit{context definitions}.  Resources can be anything like people, IT tools, data that are used to accomplish the objectives.Emerging goals can result in the requirement of new capabilities, i.e., resources. A more specific type of resource is the type \textit{Actor}, which typically refers to human performers who autonomously and collaboratively conclude an organizational process using other available Organizational Process Modeling Resources.Actors work towards the goals defined in the process. Resource models are optional to make precise definitions of resources needed.

 In Sungur et al \cite{Sungur2014a} work, the concept of \textit{Informal Process Support Model} IPSM has been introduced which is to make use of existing knowledge of human performers. Here the initial creator of the model is experienced human performers. Based on their experience, they add relevant  resources of an informal process. Each of the resources has inter relationships among the resources themselves. The models are generated at runtime based on the interactions and activities of corresponding human performers. 

 An informal process targets for accomplishment of a goal. The goals can be refined by defining sub-goals, which can be defined recursively as independent informal processes. The goal-based approach enables describing processes declaratively, i.e., without describing \textit{how} the intention is achieved, and providing only information about \textit{what} is achieved. Thus, to avoid predefined business logic in the representations of informal processes. 

 Each informal process starts from an initial context, i.e., \textit{IPE Context} and aims to achieve a goal. After accomplishing the goal, there is a resulting context called as final context. Each Resource can be related to another Resource in the context of an informal process using predefined or custom \textit{Relationships}.
 
 \begin{figure}
 	\centering
 	\includegraphics[width=\textwidth]{entity.pdf}
 	\caption{Organizational Modelling Meta-Model}
 	\label{fig:metamodel}
 \end{figure}
 

