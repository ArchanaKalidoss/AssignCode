\chapter{Introduction}
\label{chap:introduction}

Resources such as human actors, development environments, materials, etc., of an organization play an important role to accomplish organizational intentions. Though organizations can reuse available information of a process for the execution of another business process, certain process that involve human knowledge cannot be reused. These type of process is not structured like traditional process, e.g., creating a new customer savings account. The reason for irregular structure of the process is, because the sequence of activities to be carried out in order to execute a process cannot be predefined due to its dynamic changing nature, e.g., research and development processes.

The process whose required activities and order of their execution cannot be determined beforehand are called \textit{informal process} \cite{Sungur2014}. These type of processes are human-centric as their dynamic nature is due to the involvement of human knowledge. These processes are vitally important for organizations and they need to be supported and automated \cite{Sungur2014a}. Though activities of processes that involve human knowledge are unpredictable, intentions, i.e., goals of informal processes are known before their enactment \cite{DiCiccio2015}. Thus, this thesis work focuses on realizing modeling of organizational processes oriented to intention. 

The next section of this chapter, provides a detailed motivational statement of this master thesis work, followed by a problem statement section which is then followed by contributions of this work. The final section provides an outline about the following chapters of the document. 

%%%%%%%%%%%%%%%%%%%%%%%%%%%%%%%%%%%%%%%%%%%%%%%%%%%%%%%%%%%%%%%%%%%%%%%%%
\section{Motivation}
\label{sec:motivation}
%%%%%%%%%%%%%%%%%%%%%%%%%%%%%%%%%%%%%%%%%%%%%%%%%%%%%%%%%%%%%%%%%%%%%%%%%
As mentioned earlier, knowledge workers' decision has an effect on informal processes' sequence of activities. For example, research and development processes are of type where human decision plays very important role. Thus, sequence of activities for such processes cannot be decided in advance because such processes are characterized with changing requirements. These type of processes are highly unpredictable in nature and this makes it quite challenging, to support modeling these type of processes. This work is a part in realizing the modeling of such processes in organizations. Any approach that supports informal process modeling is required to be more autonomous because, the dynamic behavior of processes are enacted by some subjects. Thus, the existing modeling approaches available for traditional processes are not helpful in realizing the modeling of informal processes in organizations.  

Though sequence of steps to be carried out to execute informal processes cannot be determined beforehand, intentions of informal processes are known before their enactment. Achieving these intentions requires, another important driving force called \textit{resources}. These resources posses certain \textit{capabilities} to qualify for achieving an intention. This can be achieved by modeling through associated elements, i.e., associating intentions with strategies, strategies with capabilities and capabilities with resources. When the models are designed descriptively, i.e., providing only information what has to be done in order to achieve an intention rather than how to achieve an intention they serve as informal guides which preserves the information associated with informal processes to achieve an intention. Meanwhile, it also overcomes the need for predefining the sequence of execution steps. The non-existence of business logic facilitates more autonomy for human performers and enables establishment of best practices \cite{Sungur2014a}.

%%%%%%%%%%%%%%%%%%%%%%%%%%%%%%%%%%%%%%%%%%%%%%%%%%%%%%%%%%%%%%%%%%%%%%%%%
\section{Problem Statement}
\label{sec:problemstatement}
%%%%%%%%%%%%%%%%%%%%%%%%%%%%%%%%%%%%%%%%%%%%%%%%%%%%%%%%%%%%%%%%%%%%%%%%%
Though there are \textit{activity-centric} modeling such as Business Process Execution Language (BPEL) \footnote{http://docs.oasis-open.org/wsbpel/2.0/OS/wsbpel-v2.0-OS.pdf} and Business Process Model and Notation (BPMN) \footnote{http://www.omg.org/spec/BPMN/2.0/PDF/}, they are not suitable for certain type of processes whose execution steps cannot be predicted in advance \cite{Sungur2014a}. This is because of the challenges in determining the sequence of activities before enacting an informal process. Another key thing to remember is, informal processes are dynamic in nature due to the involvement of human knowledge. This dynamic nature is, one of the important challenges in developing an environment that supports informal process modeling. As mentioned earlier, there is also lack of modeling tool that creates models declaratively by providing only information required in order to enact a process. 

Every organization contains multiple entities like (1) \textit{resources}, e.g., humans, tools etc., (2) \textit{intentions}, e.g., revenue based intentions, quarterly intentions etc., (3) \textit{strategies}, e.g., improved customer help desk, expanding sales, etc., and (4) \textit{capabilities}, e.g., web application developer, sales representatives, etc. Thus, organizations need an approach to model these different organizational elements oriented to intention, because intention of an informal process can be known before their enactment. Thus, it is important to achieve an intention, by executing its strategies as an independent informal processes.  

Due to the involvement of multiple resources during modeling, there is a need for organizations to make decision regarding strategy selection based on cost calculation and achievability estimation. Moreover, associating capabilities with resources is helpful in the following example situation. There can be a situation where resources producing more accurate results for processing a task are preferred than resources which can produce higher throughput for processing a task. Thus, during modeling business expert has to specify that required capability as \textit{ability to provide high throughput} and match the resources with such capability. This is the reason, we associate organizational modeling elements of a process such as intentions, strategies, capabilities and resources with each other and facilitate strategy selection based on cost and achievability estimation. Thus, there is a need for an approach that satisfies all of the requirements of the intention-oriented modeling in organizations. 

%%%%%%%%%%%%%%%%%%%%%%%%%%%%%%%%%%%%%%%%%%%%%%%%%%%%%%%%%%%%%%%%%%%%%%%%%
\section {Contributions}
\label{sec:researchobjectives}
%%%%%%%%%%%%%%%%%%%%%%%%%%%%%%%%%%%%%%%%%%%%%%%%%%%%%%%%%%%%%%%%%%%%%%%%%
The contributions of this work can be categorized as follows:

\begin{enumerate}
 	\item Derived requirements from existing literatures and motivating scenario for supporting intention-oriented organizational modeling. Evaluated existing approaches based on derived requirements (Chapter  ~\ref{chap:analysis}).
 	\item An approach for intention-oriented organizational modeling that satisfies the derived requirements (Chapter ~\ref{chap:approach}).
 	\item Case study on a manufacturing company (Chapter ~\ref{chap:casestudy}).
\end{enumerate}
 
%%%%%%%%%%%%%%%%%%%%%%%%%%%%%%%%%%%%%%%%%%%%%%%%%%%%%%%%%%%%%%%%%%%%%%%%%
\section{Outline}
\label{sec:outline}
%%%%%%%%%%%%%%%%%%%%%%%%%%%%%%%%%%%%%%%%%%%%%%%%%%%%%%%%%%%%%%%%%%%%%%%%%
The remainder of this document is organized into following chapters:

\begin{description} [labelwidth = 0.07\textwidth]
	\item[Chapter ~\ref{chap:fundamentals} -- \nameref{chap:fundamentals}:] In this chapter, fundamental concepts and an overview of the related work that are essential to understand the work are provided.
	\item[Chapter ~\ref{chap:motivatingScenario} -- \nameref{chap:motivatingScenario}:] In this chapter, a motivating scenario has been taken and a detailed explanation for each phases of the scenario has been provided. This aids the reader to understand the concepts of intention-oriented organizational modeling clearly. 
	\item [Chapter ~\ref{chap:analysis} -- \nameref{chap:analysis}:] This chapter provides detailed requirement analysis for supporting intention-oriented organizational modeling. This chapter also provides a literature review and evaluation of existing work.
	\item[Chapter ~\ref{chap:approach} -- \nameref{chap:approach}:] This chapter discusses about the approach that realizes the requirements  of intention-oriented organizational modeling.
	\item[Chapter ~\ref{chap:casestudy} -- \nameref{chap:casestudy}:] This chapter validates the approach presented in Chapter \ref{chap:approach}. This chapter also discusses detailed system architecture and also presents the validation of the proposed approach. 	
	\item[Chapter ~\ref{chap:conclusion} -- \nameref{chap:conclusion}:] This chapter summarizes the results of the work and draws conclusion. This chapter also throws some light on the future work to be extended based on this work. 
\end{description}