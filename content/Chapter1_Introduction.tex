\chapter{Introduction}
\label{chap:introduction}
\begin{center}
	\textit{Creating a better world requires teamwork, partnerships, and collaboration, as we need an entire army of companies to work together to build a better world within the next few decades. This means corporations must embrace the benefits of cooperating with one another -  Simon Mainwaring}
\end{center}

 Every organization knows the benefits of collaborating a process in order to achieve its desired intention. Resources of an organization play an important role to collaborate and accomplish those tasks. Though organizations re-use data resources and tool resources during this collaboration work, business logics and decisions cannot be reused in certain types of processes. These type of processes are not structured like traditional processes because the process execution steps cannot be pre-defined due to its dynamic nature e.g processes that require involvement of human knowledge in deciding the execution steps. Such type of processes are called \textit{Informal Processes} \cite{Sungur2014}.

Humans play an important role in informal processes which makes the informal processes collaborative in nature. The participants of an informal processes collaborate to accomplish a task. These participants are the resources that drives towards the accomplishment of the task.  Developing an editor to create models for such \textit{resource-centric informal processes} is a part of realizing the automated execution of informal processes. In this document, we explain how we realized developing an editor that creates models for resource-centric informal processes. Along with this we also validate the developed prototype using a case study. This case study has been taken as an example scenario throughout this document that helps for better understanding of the concepts.

In this Chapter, the first section provides a detailed motivational reasons about why this work is relevant and what about this work is new.  The second section contains an overview about the problems in existing approaches and how this approach serves as an \textit{complemenatry approach} to the existing work. The third section discusses about the contributions done in this work i.e., the research objectives satisfied by this approach. The final section provides an overview about the following chapters. 

%%%%%%%%%%%%%%%%%%%%%%%%%%%%%%%%%%%%%%%%%%%%%%%%%%%%%%%%%%%%%%%%%%%%%%%%%
\section{Motivation}
\label{sec:motivation}
%%%%%%%%%%%%%%%%%%%%%%%%%%%%%%%%%%%%%%%%%%%%%%%%%%%%%%%%%%%%%%%%%%%%%%%%%
Nowadays, any task has both well defined predictable elements and less defined ambiguous elements. In tasks with less defined ambiguous elements, knowledge workers' decision plays an important role\footnote{White, Michael. "Case management: Combining knowledge with process." BPTrends, July (2009).}. For example, research and development projects are of type where \textit{what to do next} cannot be decided much in advance. These type of processes are highly unpredictable in nature and this makes it quite challenging to support and automate these type of processes. This work is a part in realizing the automation of such processes. These \textit{unstructured/informal/human-centric processes} are called as \textit{informal processes} \cite{Sungur2014}. Any approach that supports informal process automation is required to be more autonomous because of their dynamic behavior of enacting a process, so the existing approaches available for traditional processes are not helpful in realizing the execution of informal processes.  

Though the execution steps of informal processes cannot be determined beforehand, \textit{intentions} of informal processes are known before their enactment \cite{Sungur2015}. Achieving these intentions requires another important driving force called \textit{resources}. Resources can be anything from human actors, development environment, materials etc. These resources posses certain \textit{capabilities} to qualify for achieving an intention. So we need an approach that supports informal processes along with the support of intentions, resources, capabilities etc. This can be achieved by associating intentions with strategies, strategies with capabilities and capabilities with resources. Sungur et al. \cite{Sungur2014a} provided a descriptive meta-model approach called \textit{Informal Process Essentials}. This work serves as a part of the work by Sungur et al. Also, this work focuses to provide a web based editor to create resource-centric models of organizations. The reason for selecting descriptive modeling approach is to preserve the essential information associated with informal processes such as intentions, context information, resource definitions etc.  This work also provides means to initialize and acquire instances which can be further extended during enactment of resource-centric informal processes.  

The developed editor serves as an \textit{descriptive} web based editor tool, where the business experts can create models for informal processes, intentions, strategies, capabilities etc and this work does not comprise any functionality for compiling and executing the models. Instead this editor provides facility to plug-in the functionality for transforming the descriptive information of the models into deployable information. 

%%%%%%%%%%%%%%%%%%%%%%%%%%%%%%%%%%%%%%%%%%%%%%%%%%%%%%%%%%%%%%%%%%%%%%%%%
\section{Problem Statement}
\label{sec:problemstatement}
%%%%%%%%%%%%%%%%%%%%%%%%%%%%%%%%%%%%%%%%%%%%%%%%%%%%%%%%%%%%%%%%%%%%%%%%%
 Every organization contains multiple entities like \textit{resources} e.g., humans, tools etc., \textit{intentions} e.g., revenue based intentions, quarterly intentions etc., \textit{strategies} e.g., the process to achieve the intention and \textit{capabilities} e.g., a resource that can provide a particular capability. Thus an organization needs efficient mechanisms to handle and manage these different types of entities. Informal processes are collaborative in nature, which means that participants of informal process collaborate with each other to accomplish its intentions\cite{Sungur2015}. Designing these collaborations and assigning participants their respective privileges, plays an important role during modeling of the respective informal processes. The research work by Matthews et. al \cite{Matthews2011} mentions that below points are the major problems in adopting to a workspace collaboration tools.

\begin{enumerate}
	\item Lack of Methods
	\item Methods that focus on individuals
	\item Not well targeted groups
	\item Not well supported editors for executing abstract descriptions
\end{enumerate}

Though there are \textit{activity-centric} modeling and reusing of business processes such as Business Process Execution Language (BPEL) \footnote{http://docs.oasis-open.org/wsbpel/2.0/OS/wsbpel-v2.0-OS.pdf} and Business Process Model and Notation (BPMN) \footnote{http://www.omg.org/spec/BPMN/2.0/PDF/} are available, they are not suitable for certain type processes whose execution steps cannot be predicted in advance \cite{Sungur2014a}. Also complementary concepts such as automatic initialization and acquiring of interrelated resources are still missing in the existing work \cite{Sungur2015}. Another key thing to remember is informal processes are volatile in nature which is one of the important challenges in developing an environment that supports informal processes.

%%%%%%%%%%%%%%%%%%%%%%%%%%%%%%%%%%%%%%%%%%%%%%%%%%%%%%%%%%%%%%%%%%%%%%%%%
\section {Research Objectives}
\label{sec:researchobjectives}
%%%%%%%%%%%%%%%%%%%%%%%%%%%%%%%%%%%%%%%%%%%%%%%%%%%%%%%%%%%%%%%%%%%%%%%%%
The main focus of this work, is to realize the phase \textit{Informal Process Modeling} (P2) described in \textit{Executing Informal Processes} (InProXec) approach \cite{Sungur2015}. Coupled with the main focus of developing web based editor, the following research objectives provided in the Table \ref{tab:researchobjectives} are also satisfied by the developed editor. 

\label{sec:researchobj}
\begin{center}
	\begin{longtable}{p{5cm}p{11cm}} 
   	\toprule 
	\textbf{Research Objectives} & \textbf{Description} \\
	\midrule
	\endfirsthead
	\\
	R1 & \textit{Organizational intentions transparency}  \label{ro1} \\
	\\[-1.5ex]
	R2 & \textit{Organizational intention resource-based cost estimation}  \label{ro2} \\
	\\[-1.5ex]
	R3 & \textit{Organizational intention achievability estimation} \label{ro3}\\
	\\[-1.5ex]
	R4 & \textit{Intention oriented working style}  \label{ro4}\\
	\\[-1.5ex]
	R5 & \textit{Participative organizational modeling}\label{ro5}\\
	\\[-1.5ex]
	R6 & \textit{Re-use of organizational knowledge} \label{ro6}\\	
	\bottomrule
	\caption{Research Objectives}
	\label{tab:researchobjectives}
	\end{longtable}	
\end{center}

%%%%%%%%%%%%%%%%%%%%%%%%%%%%%%%%%%%%%%%%%%%%%%%%%%%%%%%%%%%%%%%%%%%%%%%%%
\section {Outline}
\label{sec:outline}
%%%%%%%%%%%%%%%%%%%%%%%%%%%%%%%%%%%%%%%%%%%%%%%%%%%%%%%%%%%%%%%%%%%%%%%%%
The remainder of this document is organized into following chapters:
\begin{description}
	\item[Chapter ~\ref{chap:fundamentals} -- \nameref{chap:fundamentals}:] In this chapter, basic fundamental concepts and an overview of the related approaches that are essential to understand the work are provided.
	\item[Chapter ~\ref{chap:motivatingScenario} -- \nameref{chap:motivatingScenario}:] In this chapter, a motivating scenario has been taken and detailed explanation for each phases of the scenario has been provided. This aids the reader to understand the concepts of organizational modeling clearly. 
	\item[Chapter ~\ref{chap:analysis} -- \nameref{chap:analysis}:] This chapter provides detailed requirement analysis based on scientific facts published in existing works. This chapter also provides literature review of existing works.
	\item[Chapter ~\ref{chap:approach} -- \nameref{chap:approach}:] This chapter discusses about the methodology followed in realizing the concepts  of resource-centric organizational.
	\item[Chapter ~\ref{chap:casestudy} -- \nameref{chap:casestudy}:] This chapter validates the approach presented in Chapter \ref{chap:approach}. This chapter also discusses detailed system architecture and also presents the validation results. The abstract concepts motivating scenario discussed in \ref{chap:motivatingScenario} is explained in a concrete way.	
	\item[Chapter ~\ref{chap:conclusion} -- \nameref{chap:conclusion}:] This chapter summarizes the results of the work and draws conclusion. This chapter also throws some light on the future work to be carried out in the approach of executing informal processes. 
\end{description}