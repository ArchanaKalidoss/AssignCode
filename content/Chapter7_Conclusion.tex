\chapter{Conclusion and Future Work}
\label{chap:conclusion}

In this document, we first provided motivational statement and problem statement. We also provided, the fundamental concepts and related work from existing literatures to aid the reader in understanding the concepts of intention-oriented organizational modeling. We then provided a motivating scenario taken from a manufacturing organization and explained it based on the guidelines and real life scenarios discussed in some previous work. This helps in understanding, the requirements of intention-oriented organizational modeling derived from existing literatures. The derived requirements were evaluated against few of the existing approaches. Since, none of the considered approaches satisfied all of the requirements, we proposed a new approach that satisfied all of the requirements. We then provided a detailed case study. The case study taken on a manufacturing sector helped to assess feasibility of the proposed approach. A web-based modeling tool was developed to realize the proposed approach. The case study also validated the web-based modeling tool. The usability of the tool was also confirmed by creating models for the motivating scenario.  

To be more precise, this work provided an approach that satisfied all of the requirements of the intention-oriented organizational modeling and realized the proposed approach as a web-based modeling tool. The models developed through this approach act as an informal guide for accomplishing intention, i.e., provides information required for intention-oriented organizational modeling. 

%%%%%%%%%%%%%%%%%%%%%%%%%%%%%%%%%%%%%%%%%%%%%%%%%%%%%%%%%%%%%%%%%%%%%%%%%
\section*{Future Work}
\label{sec:futurework}
%%%%%%%%%%%%%%%%%%%%%%%%%%%%%%%%%%%%%%%%%%%%%%%%%%%%%%%%%%%%%%%%%%%%%%%%%
The web-based modeling tool developed as a part of this master thesis work, will be integrated with back end such that it can be initialized. The future work also includes providing mobile modeling approach, enabling logging in functionality through few of the popular social network accounts and enhancing the user interface features of the modeling tool. 





