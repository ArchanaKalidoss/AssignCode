\chapter{Conclusion and Future Work}
\label{chap:conclusion}

There exists an overhead to configure, coordinate and engage different resources without automation during initialization. For example in our motivating scenario \ref{chap:motivatingScenario}, one of the sub-intention is to improve the help desk for answering consumer queries. One of the strategy to achieve this sub-intention is providing facility to automatically record and answer some of the known basic queries from the consumer.  To develop such an automatic help desk software we need different IT services and software developers need to be assigned and tasks has to be initiated. Though existing automation standards such as BPEL suggest to avoid such overhead by acquiring interrelated resources in priori, such complementary concepts of automatic initialization are still missing in this work  \cite{Sungur2015}. Due to high cost of automating in modeling execution steps, in comparison to its less benefits \cite{Sungur2015}, this work has not provided details of  formal definitions like which execution steps has to be taken by which actors. This work of resource-centric informal process modeling provides complementary \textit{informal} guides and definitions of intentions of the respective processes. 



%%%%%%%%%%%%%%%%%%%%%%%%%%%%%%%%%%%%%%%%%%%%%%%%%%%%%%%%%%%%%%%%%%%%%%%%%
\section*{Future Work}
\label{sec:futurework}
%%%%%%%%%%%%%%%%%%%%%%%%%%%%%%%%%%%%%%%%%%%%%%%%%%%%%%%%%%%%%%%%%%%%%%%%%
Each resources can be related with other resources through \textit{relationships} . This helps business experts to create models with logical resource structures. In this thesis work, we have addressed resource models without relationships and left the ones contain relationships as future work. This is due to the fact that relationships are optional entities in each model and also due to the broad context of this work \cite{Sungur2014a}. 



