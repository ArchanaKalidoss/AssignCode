\chapter{Conclusion and Future Work}
\label{chap:conclusion}

In this document, we first started Chapter \ref{chap:introduction} with motivational and problem statement followed by contributions of this work. In Chapter \ref{chap:fundamentals}, the fundamental concepts and related work from existing literature has been provided in detail. In Chapter \ref{chap:motivatingScenario}, a motivating scenario has been taken and explained based on the guidelines and real life scenarios discussed in some previous work. In Chapter \ref{chap:analysis}, a detailed requirements analysis based on existing literatures, a motivating scenario and evaluation of existing approaches has been provided. This is followed by Chapter \ref{chap:approach}, which provides an approach that satisfies all of the derived requirements. A detailed case study has been provided in Chapter \ref{chap:casestudy}, which helps to assess feasibility of the proposed approach. This chapter also validates the developed web–based modeling tool by providing examples that satisfies the derived requirements discussed in Chapter \ref{chap:analysis} and also conformance of the motivating scenario discussed in Chapter \ref{chap:motivatingScenario} with the developed system.

This work provides an approach that satisfies all of the requirements of intention-oriented organizational modeling and realized the proposed approach as a web-based modeling tool. The models developed through this approach act as a complementary informal guides and definitions required for intention-oriented organizational modeling. 

%%%%%%%%%%%%%%%%%%%%%%%%%%%%%%%%%%%%%%%%%%%%%%%%%%%%%%%%%%%%%%%%%%%%%%%%%
\section*{Future Work}
\label{sec:futurework}
%%%%%%%%%%%%%%%%%%%%%%%%%%%%%%%%%%%%%%%%%%%%%%%%%%%%%%%%%%%%%%%%%%%%%%%%%
The web-based modeling tool developed as a part of this master thesis work will be integrated with back end such that it can generate deployable entities from the current descriptive information. The future work also includes providing mobile modeling approach, enabling logging in functionality through few of the popular social network accounts and enhancing the user interface features of the modeling tool. 





