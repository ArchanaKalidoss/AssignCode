\chapter{An Approach to Intention-oriented Organizational Modeling}
\label{chap:approach}
This chapter describes in detail about the technical approach that has been taken to solve the problem mentioned in the Section \ref{sec:problemstatement} of Chapter \ref{chap:introduction} and to satisfy all of the requirements mentioned in the Section \ref{sec:requirementssupoorting} of Chapter \ref{chap:analysis}. The first section of this chapter provides an overview of the intention-oriented organizational modeling process. The second section discusses in detail second phase (P2) of the InProXec method, i.e., Model Informal Processes. The third section discusses in detail about the \textit{top-down approach}, which helps to realize the intention-oriented organizational modeling. The fourth section discusses the design methodology followed to realize this approach of developing a descriptive modeling tool. 

%%%%%%%%%%%%%%%%%%%%%%%%%%%%%%%%%%%%%%%%%%%%%%%%%%%%%%%%%%%%%%%%%%%%%%%%%
\section{Overview of the Modeling Process}
\label{sec:overviewmodelingprocess}
%%%%%%%%%%%%%%%%%%%%%%%%%%%%%%%%%%%%%%%%%%%%%%%%%%%%%%%%%%%%%%%%%%%%%%%%%
The main focus of this approach is to develop a web-based modeling tool which can be used by business experts to model the informal processes, intentions, strategies and capabilities. Also in this thesis work, the scope of modeling is limited only to the descriptive type of modeling i.e., models that describe processes declaratively by providing only information about what has to be done. As we mentioned before, the resource definitions required for the editor is made available from the first phase P1 of the InProXec approach. Business experts develop descriptive models through the developed modeling tool using these resource models to achieve main intention that contains strategies. The reason for following descriptive modeling approach is due to the fact that models reuse descriptive data and these stored models provides means of execution for the phases P3 and P4 of InProXec. The model provides necessary concepts and relations for modeling the core elements of resource-centric organizational modeling. Resources are abstract description which are made concrete during initialization of an instance. There are also participant specific views based on the participating resources' role. For example, based on the privilege provided to a participant he can view/edit/own/follow the instances. Initializing resource-centric models requires \textit{acquiring} and engaging interrelated resources \cite{Sungur2015} which is explained in a detailed way in the following sections of this chapter. 

%%%%%%%%%%%%%%%%%%%%%%%%%%%%%%%%%%%%%%%%%%%%%%%%%%%%%%%%%%%%%%%%%%%%%%%%%
\section{Second Phase of InProcXec - Model Informal Process}
\label{sec:informalprocessmodeling}
%%%%%%%%%%%%%%%%%%%%%%%%%%%%%%%%%%%%%%%%%%%%%%%%%%%%%%%%%%%%%%%%%%%%%%%%%
This approach of Informal Process Modeling is directed towards modeling the informal process based on their intentions rather than their activities.  Since this phase is part of InProXec method, the properties and requirements of informal process described in previous approaches \cite{Sungur2014a,Sungur2015} also applies to informal process modeling phase. The developed system serves as an holistic web based modeling tool to create, view and update all the associated elements of informal process like contexts, intentions, capabilities, strategies and resources. Also from our detailed explanation in previous sections about the importance of resources in organizational modeling and along with the fact that phase P2 receives resource definitions as input from phase P1 of InProXec method, we can apprehend that resource definitions are the lowest level in the hierarchy of intention-oriented organizational modeling approach. The sequence of steps to be carried out using the developed modeling tool has been shown in the Figure \ref{fig:processdiagram}. 

\begin{figure}
	\centering
	\includegraphics[width = \textwidth]{processmodeling.pdf}
	\caption{Steps of the Informal Process Modeling}
	\label{fig:processdiagram}
\end{figure}

\subsubsection{Model Context Definitions (M1)}  
The first step is to model context definitions, where we can model both basic properties like name and namespace of a context definition and entity specific properties like contained contexts, entity definitions, etc., of a context definition.  

\subsubsection{Model Intentions (M2)}  
Similar to context definition modeling (M1), the second step (M2) is to model intentions. The context definitions created in step M1 can be used to specify initial and final contexts of an intention. Intentions can contain sub-intentions and contradicting intentions. These type of sub intentions and contradicting intentions are also modeled as intentions in this step and their type of relation to specific intention are mentioned. Intentions are defined hierarchically, which can contain and extend sub-intentions.It is depicted by a double circle in organizational notations. The sub-intentions are refined starting from main intentions. Intentions are associated with strategies.

\subsubsection{Model Strategies (M3)}  
Once intentions are identified and modeled, the third step is modeling of strategy to achieve a specific intention. As mentioned earlier in Section \ref{sec:entitytypesrepresentation}, an intention can have multiple strategies.  A  strategy is a method or plan chosen to bring  desired results, such as achievement of an intention or solution to a problem. Strategies are associated with capabilities. 

\subsubsection{Model Required Capabilities (M4)}  
After modeling of strategies, capabilities required to achieve an intention in a specific strategy is modeled. A strategy can require multiple capabilities which has been explained in detail with a suitable example in the following Chapter \ref{chap:motivatingScenario}. 
A capability is the ability to provide business values like software applications, resources and potential of the actor to make decisions even in changing situations \cite{Stirna2012}. Capability describes the ability provided by a resource or required by an intention. The performers of an informal process should posses certain skills and roles to achieve the intention. These type of required skills are modeled during this step.

\subsubsection{Create Resource Models (M6)}  
After matching the resources and capabilities i.e after finding the correct resource that has the capability to carry out the process, the resource models are created. The need for modeling a new intention may arise in parallel during modeling of resources. This has been explained with a suitable example in the following Chapter \ref{chap:motivatingScenario}.  A resource can be a people or tool those/that drive towards the successful execution of the process. It is key for achieving specified process intentions. In the context of this work, the definition of organizational resources refers not only the entities that are capable of doing work but also entities that have an impact on the outcome of the processes, e.g., software tools, human performers, data etc.      

\subsubsection{Extract as an IPE Model (M7)}  
After the completion of above mentioned steps, the modeled entities can be extracted as an IPE model which can be reused. 

The other steps denoted in dashed round edged rectangle are not part of developed web editor. The steps are matching of required organizational capabilities (M5) that are satisfied by resource models and integration of required resources (P1). If there is no suitable matching capability then phase P1 of InProXec can be carried out again until a matching capability is found. If capabilities are satisfied resource models can be created. The created resource models(M6) along with modeled capabilities can be extracted as an IPE Model(M7) which will be provided as input for the next step execution of intentions (M8). After the execution of an intention, the status of an intention is updated inside the specific intentions's property. 

%%%%%%%%%%%%%%%%%%%%%%%%%%%%%%%%%%%%%%%%%%%%%%%%%%%%%%%%%%%%%%%%%%%%%%%%%
\section{A Top-down Modeling Approach}
\label{sec:topdownapproach}
%%%%%%%%%%%%%%%%%%%%%%%%%%%%%%%%%%%%%%%%%%%%%%%%%%%%%%%%%%%%%%%%%%%%%%%%%
As we mentioned earlier, the modeling approach in our context is descriptive modeling approach which starts from the top level intention and refines modeling until the operational bottom level is reached. Hence, it is called top-down modeling approach. The purpose of selecting top-down modeling approach is because based on the suggestion provided in the literature \cite{Mandic2010, Bider2005,Sungur2016} that the value of intention in the top of hierarchy propagates till the lower levels and helps in making investment-related decisions while at same time integrating cost and benefits estimates from all levels. Moreover, by creating declarative models using top-down modeling approach. Models are easily changeable as they are decoupled from their operational terms. Such declarative approaches provide more flexibility and enable easier change of the business process models. The integration of declarative models using top-down modeling approach also provides a coupling of cost-benefit and strategy achieve-ability estimation with operationally measurable business intention and supports the evaluation of business intention success and the effectiveness of the chosen strategies. In the Figure \ref{fig:topdownapproach}, it has been shown that how this modeling approach starts modeling from top level intentions and does modeling until the operational lower level is reached and how the organizational modeling elements are associated with each other. 

\begin{figure}
	\centering
	\includegraphics[width=\textwidth]{TopDownApproach.pdf}
	\caption{Intention-oriented Organizational Modeling: A Top down Modeling Approach}
	\label{fig:topdownapproach}
\end{figure}

The approach is evaluated based on the derived requirements in the Section \ref{sec:requirementssupoorting} of Chapter \ref{chap:analysis} as follows :

\textit{Organizational Intention Transparency} (R1) : From the Figure \ref{fig:topdownapproach}, one could understand that (1) intentions are refinable and as per the current design of the approach (2) organizational members can view the intentions at different levels. Thus, requirement R1 is satisfied by the approach as it satisfies all of the pre-requisites. 

\textit{Organizational Strategy-based Cost Estimation} (R2) : During modeling phase itself business executive can make decisions based on the strategy cost estimation. In this approach (1) resources are associated with cost and (2) the cost estimation of strategies include all its low level structures. Thus, requirement R2 is satisfied by the approach as it satisfies all of the pre-requisites. 

\textit{Organizational Strategy Achieve-ability Estimation} (R3) : Similar to requirement R2 cost calculation, strategy achieve-ability estimation based on strategy's association with valid capability is also estimated during modeling phase itself. In this approach (1) a capability is considered as valid only when it is associated with matching resource and (2) from the Figure \ref{fig:topdownapproach}, one could understand how independent informal process realizes strategy. Thus, requirement R3 is satisfied by the approach as it satisfies all of the pre-requisites.

\textit{Intention Oriented Working Style} (R4) : This approach (1) satisfies requirement R1 and (2) when modeling through this approach members require understanding of intention and its associated elements for successfully achieving the main intention. Thus, requirement R4 is satisfied by the approach as it satisfies all of the pre-requisites.

\textit{Participative Organizational Modeling} (R5) : This approach (1) satisfies requirement R1 and  (2) intentions are modeled collaboratively based on input received from different members of the organization. Thus, requirement R4 is satisfied by the approach as it satisfies all of the pre-requisites. 

%%%%%%%%%%%%%%%%%%%%%%%%%%%%%%%%%%%%%%%%%%%%%%%%%%%%%%%%%%%%%%%%%%%%%%%%%
\section{Design Methodology}
\label{sec:designmethodology}
%%%%%%%%%%%%%%%%%%%%%%%%%%%%%%%%%%%%%%%%%%%%%%%%%%%%%%%%%%%%%%%%%%%%%%%%%
This section discusses in detail the method of designing the web-based modeling tool that realizes the approach proposed in the Section \ref{sec:topdownapproach}. When designing the user interface components and functionalities required to develop the tool, most of the similar functionalities are designed as common functions for re-using the functions. This reduced unnecessary functional redundancies and overhead. Some of the important methodologies followed with respect to user interface components design are 1. multiple items to be selected from multiple list items are displayed as \textit{list group} and 2. selecting single item from multiple items are displayed as \textit{drop down}. For example, to select multiple strategies from a list of strategies, available strategies are displayed as a list from which the user can select desired number of strategies. Another important methodology followed during user interface design is, for every entity the properties should be displayed only under the respective properties tab. For example, in the Figure \ref{fig:samplescreen}, the basic properties such as name, target namespace and process type of an informal process model should be displayed only under the respective basic properties tab and similarly for all other tabs. This methodology is followed uniformly throughout the design of all the entity types such as intention definitions, strategy definitions, capability definitions, context definitions, instance definitions and informal process definitions. 

All data are stored only under the data artifact. This applies to the labels and text fields of all user interface elements and this data can be updated only through the handler function. Through the \textit{settings} option, user can add new namespace and intention relation type. From the Figure \ref{fig:samplescreen}, it is clear that a standard design methodology has been followed to display the list of available entity types such as intentions, strategies, capabilities etc., and to display their respective properties such as basic, entity specific, instance data, etc., properties. Though the top-down modeling approach \ref{sec:topdownapproach}, shows that definition of each entity type is contained within another entity type, as per the user interface design, separate entities references each other using the unique reference identifier but does not contain all properties of referenced entity. For instance, a strategy containing an intention should contain only the intention's unique reference identifier but not the actual intention itself. Later, in the view of strategy, actual intention properties are fetched and displayed based on the unique reference identifier. 

\begin{figure}
	\centering
	\includegraphics[width=\textwidth,angle=0]{samplescreen.png}
	\caption{User Interface Design of the Editor}
	\label{fig:samplescreen}
\end{figure}



