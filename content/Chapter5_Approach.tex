\chapter{An Approach to Resource-centric Organizational Modeling}
\label{chap:approach}

%%Description of the approach you have taken to solve the scientific or technical problem which you were posed.
%%Outline the design, the methodology and overall structure of your experinmental approach.

%%%%%%%%%%%%%%%%%%%%%%%%%%%%%%%%%%%%%%%%%%%%%%%%%%%%%%%%%%%%%%%%%%%%%%%%%
\section{Overview of Modeling Process}
\label{sec:overviewmodelingprocess}
%%%%%%%%%%%%%%%%%%%%%%%%%%%%%%%%%%%%%%%%%%%%%%%%%%%%%%%%%%%%%%%%%%%%%%%%%
 The Organizational Modeling element notation has been selected as per the guidelines mentioned in the paper by Moody \cite{Moody2009}. Also by observing  the fact that business process modelers are already well-known with the present process modeling notations such as Business Process Modeling Notation 2.0 (BPMN) \cite{bpm2011} and ArchiMate notation \cite{arc2013}, the shape depiction of organizational model elements are designed similar to those existing process notations. 




%%%%%%%%%%%%%%%%%%%%%%%%%%%%%%%%%%%%%%%%%%%%%%%%%%%%%%%%%%%%%%%%%%%%%%%%%
\section{Evaluation of the Approach}
\label{sec:evaluationoftheapproach}
%%%%%%%%%%%%%%%%%%%%%%%%%%%%%%%%%%%%%%%%%%%%%%%%%%%%%%%%%%%%%%%%%%%%%%%%%

The Table \ref{tab:evaluationoftheapproach}, provides an evaluation of the approaches. The description of each symbol used in the  Table \ref{tab:evaluationoftheapproach} is given as a legend.


\begin{center}
	\begin{longtable}{p{5cm}p{2cm}p{2cm}p{2cm}p{2cm}p{2cm}} 
		\toprule 
		\textbf{Approach} & \textbf{R1}  & \textbf{R2}  & \textbf{R3}  & \textbf{R3}  & \textbf{R5} \\
		\midrule
		\endfirsthead
		\\
	   	Strategy-Driven & -  & -  & +  & -  & + \\
	   	Activity-centric System  \\
	   	Activity-oriented System  \\
	   	Artifact-centric System  \\
	   	Capability-driven Development \\
	   	ArchiMate \\
	    Subject-Oriented System \\
		
		\bottomrule
		\caption{Evaluation of the Approach}
		\label{tab:evaluationoftheapproach}
	\end{longtable}	
\end{center}

Legend :
\begin{description}
	\item[+]  Addressed in the approach
	\item[-]  Not addressed in the approach
	\item[\~] Partially addressed in the approach
\end{description}

%%%%%%%%%%%%%%%%%%%%%%%%%%%%%%%%%%%%%%%%%%%%%%%%%%%%%%%%%%%%%%%%%%%%%%%%%
\section{Design Methodology}
\label{sec:designmethodology}
%%%%%%%%%%%%%%%%%%%%%%%%%%%%%%%%%%%%%%%%%%%%%%%%%%%%%%%%%%%%%%%%%%%%%%%%%

On the left list we should have all organizational context definitions and on the right one only ones that are contained in an informal process. The dropdown box of the initial and final context defines the selection inside of an informal process, thus right side.

So, in db.cljs, we should have only a list of context definitions no :initial-contexts and :desired-final-contexts. Only :organizational-contexts and under this all available contexts. Under the left list, we present these elements. Right list should refer to the initial-context, final-contexts, etc. of the informal process model depending on the selection of the dropbox button. For instance if we have initial contexts selected on the dropdown box, we should present the initial contexts in the right list.

I have changed the code accordingly and provided you an example how you should change data from views.cljs. All data should be stored in db.cljs. This applies to the text fields of all elements. Whenever, we want to update something we need to update the map in db.cljs and this will be propagated to the views.


On the left side of each list item, you should present all available items of context definitions or intentions whatever type is selected there. On the right side only the ones contained in the respective informal process model. Inside of another entity, you should refer to other entities using their ids and these ids should be resolved using, for instance, intentions vector. You check each intention in the intentions vector, if it’s id is the same as the id you are looking for it, you found it and you use the information about it.  


Please align it with the structure and names of the IPSM.xsd. Each variableName like this is written like variable-name. Each complex type is a map each attribute is a key value pair and each element in another element is another key value pair.


%%%%%%%%%%%%%%%%%%%%%%%%%%%%%%%%%%%%%%%%%%%%%%%%%%%%%%%%%%%%%%%%%%%%%%%%%
\subsection{Specifications}
\label{subsec:specifications}
%%%%%%%%%%%%%%%%%%%%%%%%%%%%%%%%%%%%%%%%%%%%%%%%%%%%%%%%%%%%%%%%%%%%%%%%%
In order to realize the web editor of Intention-centric Organizational Modeling, a formal inquiry has been done and concluded with the below specifications.

\begin{enumerate}   
	\item \textbf{Clojurescript} as the programming language
	\item \textbf{IntelliJIDEA} as the IDE
	\item \textbf{MVC} as the architecture pattern
	\item \textbf{Re-frame} as the pattern for writing SPAs in ClojureScript, using Reagent	
\end{enumerate}

%%%%%%%%%%%%%%%%%%%%%%%%%%%%%%%%%%%%%%%%%%%%%%%%%%%%%%%%%%%%%%%%%%%%%%%%%
\subsection{MVC Architecture}
\label{subsec:mvcarch}
%%%%%%%%%%%%%%%%%%%%%%%%%%%%%%%%%%%%%%%%%%%%%%%%%%%%%%%%%%%%%%%%%%%%%%%%%
 The architecture of the UI editor is based on the \textbf{Model-View-Control (MVC)} design pattern. The MVC paradigm allows to separate business logic from the code that controls presentation and event handling \cite{Oracle2016}.Each entity view in the web page is made up of combination of at least on Model and View, and one or more Controls. The individual files which acts an Model, View and Controller has been shown in the Figure \ref{fig:mvc_arch}

\begin{itemize}
	\item \textbf{Model} artifact stores the required data structure for web-editor. In the developed model artifact, the four main types of data stored inside the artifact are intentions, strategies, capabilities and informal process instances. 
	\item \textbf{View} artifact contains HTML elements and HTML constructs that describe the way of displaying the data from Model to the user.
	\item \textbf{Control} artifact contains the handler functions which can only change the model. Even the initial values of the model are put inside the control. 
\end{itemize}


\begin{figure}
	\centering
	\includegraphics [width= 0.75\textwidth]{mvc_arch.pdf}
	\caption{Relationship between developed web editor artifacts and MVC architecture components}
	\label{fig:mvc_arch}
\end{figure}


%%%%%%%%%%%%%%%%%%%%%%%%%%%%%%%%%%%%%%%%%%%%%%%%%%%%%%%%%%%%%%%%%%%%%%%%%
\subsubsection{Example: Component using MVC Pattern }
%%%%%%%%%%%%%%%%%%%%%%%%%%%%%%%%%%%%%%%%%%%%%%%%%%%%%%%%%%%%%%%%%%%%%%%%%
 The Figure \ref{fig:mvc_arch} below shows the simplifed version of how the components interact with each other using the Model-View-Control (MVC) pattern, for the functionality adding new entity data. This functionality is same for all the types intentions, strategies, capabilities and informal proceess instances and below is the detailed explanation of each interaction.

\begin{enumerate}
	\item User clicks the tab \textbf{Add New} in the web editor.
	\item View, in response to the user click displays the UI component for entering the new entity data details.
	\item User enters the required basic details for adding new entity data and clicks save button.
	\item View dispatches the data to Control, which can only modify the Model.
	\item Control inserts/updates data into the model.
	\item View displays the updated model as it has been subscribed to the model.
\end{enumerate}

\begin{figure}
	\centering
	\includegraphics [width= \textwidth]{mvc_pattern.pdf}
	\caption{MVC Pattern of adding new entity}
	\label{fig:mvc_pattern}
\end{figure}


%%%%%%%%%%%%%%%%%%%%%%%%%%%%%%%%%%%%%%%%%%%%%%%%%%%%%%%%%%%%%%%%%%%%%%%%%
\subsection{Using Reagent Framework}
\label{subsec:reagent}
%%%%%%%%%%%%%%%%%%%%%%%%%%%%%%%%%%%%%%%%%%%%%%%%%%%%%%%%%%%%%%%%%%%%%%%%%

The Reagent Framework architecture has been reused Fig. \ref{fig:mvc_pattern23} \footnote{Source: https://github.com/Day8/re-frame}


\begin{figure}
	\centering
	\includegraphics [width= \textwidth]{mvc_pattern.pdf}
	\caption{MVC Pattern of adding new entity }
	\label{fig:mvc_pattern23}
\end{figure}


%%%%%%%%%%%%%%%%%%%%%%%%%%%%%%%%%%%%%%%%%%%%%%%%%%%%%%%%%%%%%%%%%%%%%%%%%
\subsection{User Interface Diagram}
\label{sec:uidiagram}
%%%%%%%%%%%%%%%%%%%%%%%%%%%%%%%%%%%%%%%%%%%%%%%%%%%%%%%%%%%%%%%%%%%%%%%%%




%%%%%%%%%%%%%%%%%%%%%%%%%%%%%%%%%%%%%%%%%%%%%%%%%%%%%%%%%%%%%%%%%%%%%%%%%
\section{Relationship between Entity Types}
\label{sec:enttyperelation}
%%%%%%%%%%%%%%%%%%%%%%%%%%%%%%%%%%%%%%%%%%%%%%%%%%%%%%%%%%%%%%%%%%%%%%%%%

%%%%%%%%%%%%%%%%%%%%%%%%%%%%%%%%%%%%%%%%%%%%%%%%%%%%%%%%%%%%%%%%%%%%%%%%%
\subsection{Context Intention Relationship}
\label{sec:ctxintrel}
%%%%%%%%%%%%%%%%%%%%%%%%%%%%%%%%%%%%%%%%%%%%%%%%%%%%%%%%%%%%%%%%%%%%%%%%%
Intentions connect initial context definitions with final context definitions. \ref{fig:CtxIntRel}


\begin{figure}
	\centering
	\includegraphics [width= 0.5\textwidth]{CtxIntRel.pdf}
	\caption{Context Intentions Relationship}
	\label{fig:CtxIntRel}
\end{figure}

%%%%%%%%%%%%%%%%%%%%%%%%%%%%%%%%%%%%%%%%%%%%%%%%%%%%%%%%%%%%%%%%%%%%%%%%%
\subsection{Capabilities Intention Relationship}
\label{sec:capIntRel}
%%%%%%%%%%%%%%%%%%%%%%%%%%%%%%%%%%%%%%%%%%%%%%%%%%%%%%%%%%%%%%%%%%%%%%%%%

 Each organizational capability must be provided by a resource in the organization. Resource models are optional to make precise definitions of resources needed. The relationship between organizational capabilities and organizational intentions has been provided in the Figure 
 
 \begin{figure}
 	\centering
 	\includegraphics[width=\textwidth]{CapIntRel.pdf}
 	\caption{Relation between organizational capabilities and intentions}
 	\label{fig:orgcapabilities}
 \end{figure}


