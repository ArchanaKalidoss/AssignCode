\chapter{Introduction}
\label{chap:introduction}
This work is structured in the following way:

\section*{Outline}
This work is structured in the following way:
\begin{description}
\item[Chapter ~\ref{chap:motivatingScenario} -- \nameref{chap:motivatingScenario}:] In this chapter a motivating scenario has been taken which helps reader in realizing the Organizational Modeling throughout the document. 
\item[Chapter ~\ref{chap:fundamentals} -- \nameref{chap:fundamentals}:] In this chapter previous work carried in organizations to model the goals has been discussed.
\item[Chapter ~\ref{chap:requirements} -- \nameref{chap:requirements}:] Detailed requirements based on scientific facts and introduction about some properties of the organizations.
\item[Chapter ~\ref{chap:orgModeling} -- \nameref{chap:orgModeling}:] In this chapter basics of Organizational Modeling has been discussed.
\item[Chapter ~\ref{chap:modelingNotation} -- \nameref{chap:modelingNotation}:] In this chapter, notations used to realise the Organization Modeling has been introduced. 
\item[Chapter ~\ref{chap:realization} -- \nameref{chap:realization}:] 
\item[Chapter ~\ref{chap:evaluation} -- \nameref{chap:evaluation}:] Comparison of the Intention-centric Modeling of Organizations approach with the other approaches.
\item[Chapter ~\ref{chap:conclusion} -- \nameref{chap:conclusion}:] 
\end{description}


%%%%%%%%%%%%%%%%%%%%%%%%%%%%%%%%%%%%%%%%%%%%%%%%%%%%%%%%%%%%%%%%%%%%%%%%%
\section{Motivation}
%%%%%%%%%%%%%%%%%%%%%%%%%%%%%%%%%%%%%%%%%%%%%%%%%%%%%%%%%%%%%%%%%%%%%%%%%


%%%%%%%%%%%%%%%%%%%%%%%%%%%%%%%%%%%%%%%%%%%%%%%%%%%%%%%%%%%%%%%%%%%%%%%%%
\section{Problem Definition}
%%%%%%%%%%%%%%%%%%%%%%%%%%%%%%%%%%%%%%%%%%%%%%%%%%%%%%%%%%%%%%%%%%%%%%%%%

%%%%%%%%%%%%%%%%%%%%%%%%%%%%%%%%%%%%%%%%%%%%%%%%%%%%%%%%%%%%%%%%%%%%%%%%%
\section{Contributions and Outline}
%%%%%%%%%%%%%%%%%%%%%%%%%%%%%%%%%%%%%%%%%%%%%%%%%%%%%%%%%%%%%%%%%%%%%%%%%