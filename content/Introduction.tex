\chapter{Introduction}
\label{chap:introduction}
This work is structured in the following way:


\section*{Outline}
The remainder of the document is organized into following chapters
\begin{description}
\item[Chapter ~\ref{chap:motivatingScenario} -- \nameref{chap:motivatingScenario}:] In this chapter a motivating scenario has been taken which helps reader in realizing the Organizational Modeling throughout the document. 
\item[Chapter ~\ref{chap:fundamentals} -- \nameref{chap:fundamentals}:] In this chapter previous work carried in organizations to model the goals has been discussed.
\item[Chapter ~\ref{chap:requirements} -- \nameref{chap:requirements}:] Detailed requirements based on scientific facts and introduction about some properties of the organizations.
\item[Chapter ~\ref{chap:orgModeling} -- \nameref{chap:orgModeling}:] In this chapter basics of Organizational Modeling has been discussed and notations used to realize the Organization Modeling has been discussed.
\item[Chapter ~\ref{chap:design} -- \nameref{chap:design}:] In this chapter, proposed design is discussed. 
 
\item[Chapter ~\ref{chap:realization} -- \nameref{chap:realization}:] This chapter concentrates on detailed system architecture and also presents the experimental results.
\item[Chapter ~\ref{chap:evaluation} -- \nameref{chap:evaluation}:] Comparison of the Intention-centric Modeling of Organizations approach with the other approaches.
\item[Chapter ~\ref{chap:conclusion} -- \nameref{chap:conclusion}:] Summaries the results of this thesis work and draws conclusion. 
\end{description}


%%%%%%%%%%%%%%%%%%%%%%%%%%%%%%%%%%%%%%%%%%%%%%%%%%%%%%%%%%%%%%%%%%%%%%%%%
\section{Motivation}
%%%%%%%%%%%%%%%%%%%%%%%%%%%%%%%%%%%%%%%%%%%%%%%%%%%%%%%%%%%%%%%%%%%%%%%%%


%%%%%%%%%%%%%%%%%%%%%%%%%%%%%%%%%%%%%%%%%%%%%%%%%%%%%%%%%%%%%%%%%%%%%%%%%
\section{Problem Definition}
%%%%%%%%%%%%%%%%%%%%%%%%%%%%%%%%%%%%%%%%%%%%%%%%%%%%%%%%%%%%%%%%%%%%%%%%%

%%%%%%%%%%%%%%%%%%%%%%%%%%%%%%%%%%%%%%%%%%%%%%%%%%%%%%%%%%%%%%%%%%%%%%%%%
\section{Contributions and Outline}
%%%%%%%%%%%%%%%%%%%%%%%%%%%%%%%%%%%%%%%%%%%%%%%%%%%%%%%%%%%%%%%%%%%%%%%%%

%%%%%%%%%%%%%%%%%%%%%%%%%%%%%%%%%%%%%%%%%%%%%%%%%%%%%%%%%%%%%%%%%%%%%%%%%
\section {Research Objectives}
%%%%%%%%%%%%%%%%%%%%%%%%%%%%%%%%%%%%%%%%%%%%%%%%%%%%%%%%%%%%%%%%%%%%%%%%%
\label{sec:researchobj}

\begin{tabular}{p{5cm}p{11cm}} 
	\textbf{Research Objective} & \textbf{Description} \\
	\\
	RO. 1 & \textit{Intentions are traceble in the different levels of the organizational hierarchy. } \label{ro1} \\
	\\[-1.5ex]
	RO. 2 & \textit{Linking intentions with capabilities and at the with resources enable us a cost estimation for each intention. Cost is estimated in a recursive manner.} \label{ro2} \\
	\\[-1.5ex]
	RO. 3 & \textit{Validity of an organizational intentions is achieveable when the intentions can be refined by defining sub-intentions, which can then be defined recursively as independent informal processes.} \label{ro3}\\
	\\[-1.5ex]
	RO. 4 & \textit{As each member of the organization aware of the higher level and lower level intentions. He can engage for these explicit intentions. } \label{ro4}\\
	\\[-1.5ex]
	RO. 5 & \textit{Different members of an organization participate to create organizational intentions, as a result intentions are shaped based on all members but directed by the executives.} \label{ro5}\\
	\\[-1.5ex]
	RO. 6 & \textit{Intention-specific solutions can be extracted as abstract re-usable entities, organizational strategy patterns and can be re-used in muliple context definitions.} \label{ro6}\\
\end{tabular}
