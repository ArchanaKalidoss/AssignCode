% !TeX spellcheck = de_DE
% Dieses Dokument muss mit PDFLatex gesetzt werden
% Vorteil: Grafiken koennen als jpg, png, ... verwendet werden
%          und die Links im Dokument sind auch gleich richtig
%
%Ermöglicht \\ bei der Titelseite (z.B. bei supervisor)
%Siehe https://github.com/latextemplates/uni-stuttgart-cs-cover/issues/4
\RequirePackage{kvoptions-patch}
%Warns about outdated packages and missing caption delcarations
%See https://www.ctan.org/pkg/nag
\RequirePackage[l2tabu, orthodox]{nag}
%Neue deutsche Trennmuster
%Siehe http://www.ctan.org/pkg/dehyph-exptl und http://projekte.dante.de/Trennmuster/WebHome
%Nur für pdflatex, nicht für lualatex
%\RequirePackage[ngerman=ngerman-x-latest]{hyphsubst}
\documentclass[
               fontsize=12pt, %Default: 11pt, bei Linux Libertine zu klein zum Lesen
               paper=a4,
               twoside, % fuer die Betrachtung am Schirm ungeschickt
% BEGINN: Optionen für typearea
               BCOR=3mm, % Hack für BCOR (1.92 o.ä.), da bei BCOR2mm die Fuellpunkte beim Inhaltsverzeichnis falsch sind. Hack aber nicht mehr nötig: microtype für Verzeichnisse ausschalten hilft.
               DIV=13,   % je höher der DIV-Wert, desto mehr geht auf eine Seite. Gute werde sind zwischen DIV=12 und DIV=15
               headinclude=true,
               footinclude=false,
% ENDE: Optionen für typearea
%               titlepage,
               bibliography=totoc,
%               idxtotoc,   %Index ins Inhaltsverzeichnis
%                liststotoc, %List of X ins Inhaltsverzeichnis, mit liststotocnumbered werden die Abbildungsverzeichnisse nummeriert
               headsepline,
               cleardoublepage=empty,
               parskip=half,
               pointlessnumbers, %f"ur englische Texte, dann unten \ifdeutsch und \ifenglisch anpassen.
%               draft    % um zu sehen, wo noch nachgebessert werden muss - wichtig, da Bindungskorrektur mit drin
               final   % ACHTUNG! - in pagestyle.tex noch Seitenstil anpassen
               ]{scrbook}

%Englisch:
\let\ifdeutsch\iffalse
\let\ifenglisch\iftrue

%Deutsch:
%\let\ifdeutsch\iftrue
%\let\ifenglisch\iffalse


\input{preambel/packages_and_options}

%Der untere Rand darf "flattern"
\raggedbottom

%%%
% Wie tief wird das Inhaltsverzeichnis aufgeschlüsselt
% 0 --\chapter
% 1 --\section % fuer kuerzeres Inhaltsverzeichnis verwenden - oder minitoc benutzen
% 2 --\subsection
% 3 --\subsubsection
% 4 --\paragraph
\setcounter{tocdepth}{1}
%
%%%

\makeindex

%Angaben in die PDF-Infos uebernehmen
\makeatletter
\hypersetup{
            pdftitle={}, %Titel der Arbeit
            pdfauthor={}, %Author
            pdfkeywords={}, % CR-Klassifikation und ggf. weitere Stichworte
            pdfsubject={}
}
\makeatother

\begin{document}
%%%%%%%%%%%%%%%%%%%%%%%%%%%%%%%%%%%%%%%%%%%
%Erzeugung des Titelblatts
%%%%%%%%%%%%%%%%%%%%%%%%%%%%%%%%%%%%%%%%%%%
% Deckblatt zentrieren
%\newlength\oddsidemarginorig
%\oddsidemarginorig=\oddsidemargin
%\oddsidemargin 1.05cm
%\newgeometry{a4paper,left=20.05mm,right=10mm, top=29.7mm, bottom=29.7mm}
%\thispagestyle{plain}
\pagestyle{plain}
\begin{titlepage}
	\begin{sffamily}
		\begin{center}
			Institute of Architecture of Application Systems\\
			University of Stuttgart\\
			Universitätsstraße 38\\
			D-70569 Stuttgart\\
		\end{center}
		
		\vspace{3.5cm}
		
		\begin{center}
			{Master's Thesis No. MCS-0003 }\\
			\vspace{0.5cm}
			\begin{minipage}{8.5cm}
				\begin{center}
					
					\Large \textbf{Intention-oriented Organizational Modeling - A Top-down Approach}
					
				\end{center}
			\end{minipage}
			\\
			\vspace{1cm}
			{Archana Kalidoss}
		\end{center}
		
		\vspace{1.0cm}
		
		\begin{center}
			\begin{minipage}{3cm}
				\begin{center}
					\includegraphics[width=0.9\textwidth]{./gfx/unilogo.pdf}
					%\includegraphics{./gfx/uni_logo.png}
				\end{center}
			\end{minipage}
			\begin{minipage}{3cm}
				\begin{center}
					\includegraphics{./gfx/iaas.jpg}
				\end{center}
			\end{minipage}
		\end{center}
		%
		\vspace{1.0cm}
		%
		\begin{center}
			\begin{tabular}{ll}
				\textbf{Course of Study:} & Computer Science M.Sc\\
				&\\&\\
				\textbf{Examiner:}   & Prof. Dr. Dr. h. c. Frank Leymann\\
				\textbf{Supervisor:}   & M.Sc. C. Timurhan Sungur\\
				
				\textbf{Commenced:} & 2nd November 2015\\
				\textbf{Completed:}  & 2nd August 2016\\
				&\\
				\textbf{CR-Classification:} & H.4.1, H.5.3 \\
				
			\end{tabular}
		\end{center}
	\end{sffamily}
\end{titlepage}
%\oddsidemargin=\oddsidemarginorig
%%%%%%%%%%%%%%%%%%%%%%%%%%%%%%%%%%%%%%%%%%%
%Ende Titelblatt
%%%%%%%%%%%%%%%%%%%%%%%%%%%%%%%%%%%%%%%%%%%
	
\cleardoublepage
	
%tex4ht-Konvertierung verschönern
\iftex4ht
% tell tex4ht to create picures also for formulas starting with '$'
% WARNING: a tex4ht run now takes forever!
\Configure{$}{\PicMath}{\EndPicMath}{} 
%$ % <- syntax highlighting fix for emacs
\Css{body {text-align:justify;}}

%conversion of .pdf to .png
\Configure{graphics*}  
         {pdf}  
         {\Needs{"convert \csname Gin@base\endcsname.pdf  
                               \csname Gin@base\endcsname.png"}%  
          \Picture[pict]{\csname Gin@base\endcsname.png}%  
         }  
\fi

%Tipp von http://goemonx.blogspot.de/2012/01/pdflatex-ligaturen-und-copynpaste.html
%siehe auch http://tex.stackexchange.com/questions/4397/make-ligatures-in-linux-libertine-copyable-and-searchable
%
%ONLY WORKS ON MiKTeX
%On other systems, download glyphtounicode.tex from http://pdftex.sarovar.org/misc/
%
\input glyphtounicode.tex
\pdfgentounicode=1

\VerbatimFootnotes %verbatim text in Fußnoten erlauben. Geht normalerweise nicht.

\input{macros/commands}
\pagenumbering{arabic}
%\Titelblatt

%Eigener Seitenstil fuer die Kurzfassung und das Inhaltsverzeichnis
\deftripstyle{preamble}{}{}{}{}{}{\pagemark}
%Doku zu deftripstyle: scrguide.pdf
\pagestyle{preamble}
\renewcommand*{\chapterpagestyle}{preamble}

\setlength{\parindent}{0.0em}


%Kurzfassung / abstract
%auch im Stil vom Inhaltsverzeichnis
%\ifdeutsch
%\section*{Abstract}
%\else
\section*{Abstract}
%State clearly what problem has been studied and/or what is the goal of the %thesis/paper. Give a  brief  statement  on  existing  solutions  and  their  %drawbacks.  List  major  contributions  of  the  thesis.  State bri                                                                                                                                                                                                                                                                                                                                                                                                                                                                                                                                                                                                                                                                                                                                                                                                                                                                                efly  %assumptions  and  limitations.  The  abstract  should  also  include  major  %idea(s), the type (e.g. performance, complexity) and result of analysis done. 

%\fi
The involvement of human resources is a necessity in many organizations. In these organizations, there are processes that may require decisions taken by the human resources. The processes that are accomplished through human knowledge have irregular sequence of execution steps, i.e., the series of activities to be carried out are not structured. It is also important to guide such unstructured organizational processes and resources because they work towards the achievement of an organizational intention. Thus, designing models that serve as guide in order to achieve the organizational intentions are of prime importance. The intentions play a critical role in organizations because they motivate the organizational resources to work towards the overall development of an organization. Thus, supporting modeling of the intentions, strategies to achieve the intentions, capabilities required by the strategies, resources that provide the capabilities and processes that implement the strategies everything together in a holistic way is vitally important for any organizational modeling approach. The holistic way of modeling is required, because each modeling element requires modeling of its associated element. 

Traditional modeling approaches that are oriented to the sequence of activities, are not suitable when the sequence of activities cannot be determined in advance. Hence, there is a need for a modeling approach that enables creating models as guide in order to achieve an intention rather than providing sequence of steps required to achieve an intention. This master thesis work, proposes a modeling approach based on the derived requirements of the intention-oriented organizational modeling. The proposed approach allows creating organizational models that acts as a descriptive guide, e.g., providing information about the required strategies to achieve an intention. 

In the proposed modeling approach, the intentions are realized through the strategies which are associated with the capabilities that are satisfied by the resources. As a result, the unstructured organizational processes realizes the strategies that are associated with the capabilities, resources and intentions. A motivating scenario from an organization that belongs to the manufacturing sector is provided to help the reader in easily acquiring the concepts of the proposed approach. The approach is realized as a web-based modeling tool through which organizational models can be created. To assess, feasibility of the proposed approach and usability of the developed modeling tool, we also provide a case study centered around the motivating scenario.  

\textbf{Keywords:} Intention-oriented modeling, informal processes, top-down approach, descriptive guides 

% BEGIN: Verzeichnisse

\iftex4ht
\else
\microtypesetup{protrusion=false}
\fi

%%%
% Literaturverzeichnis ins TOC mit aufnehmen, aber nur wenn nichts anderes mehr hilft!
% \addcontentsline{toc}{chapter}{Literaturverzeichnis}
%
% oder zB
%\addcontentsline{toc}{section}{Abkürzungsverzeichnis}
%\section*{Abkürzungsverzeichnis}
%
%%%

%Produce table of contents
%
%In case you have trouble with headings reaching into the page numbers, enable the following three lines.
%Hint by http://golatex.de/inhaltsverzeichnis-schreibt-ueber-rand-t3106.html
%
%\makeatletter
%\renewcommand{\@pnumwidth}{2em}
%\makeatother
%


\tableofcontents

% Bei einem ungünstigen Seitenumbruch im Inhaltsverzeichnis, kann dieser mit
% \addtocontents{toc}{\protect\newpage}
% an der passenden Stelle im Fließtext erzwungen werden.

%listof* untereinandergesetzt
%ACHTUNG! Falls ein anderer Kapitelstil gewählt wird, muss der Code hier evtl.
%  angepasst werden
\begingroup 
\makeatletter
  \def\@makeschapterhead#1{%
  \vspace*{10\p@}%
  {\parindent \z@ \raggedright \reset@font
            \normalfont \vphantom{\@chapapp{} \thechapter}
        \par\nobreak\vspace*{10\p@}%
        \interlinepenalty\@M
    {\huge \bfseries %
    %
    %Default-Schrift: Serifenhaft (fuer englische Dokumente)
    % Dann sowohl A als auch B deaktivieren
    %A) Fuer serifenlose Schrift folgende Zeile aktivieren:
    \ifdeutsch
    \fontfamily{phv}\selectfont
    \fi
    %B) Fuer Kapitaelchen folgende Zeile aktivieren:
    %\fontseries{m}\fontshape{sc}\selectfont
    %
    #1\par\nobreak}
    %\vspace*{1\p@}%
\makebox[\textwidth]{\hrulefill}%    \hrulefill alone does not work
    \par\nobreak
    \vskip 5\p@
  }}
\makeatother
\cleardoublepage
\listoffigures
%\cleardoublepage
%\newpage
%\let\cleardoublepage\relax
\cleardoublepage
\listoftables

%\newpage
%Wird nur bei Verwendung von der lstlisting-Umgebung mit dem "caption"-Parameter benoetigt
%\lstlistoflistings 
%ansonsten:
%\cleardoublepage
%\ifdeutsch
%\listof{Listing}{List of Listings}
%\else
%\listof{Listing}{List of Listings}
%\fi

%mittels \newfloat wurde die Algorithmus-Gleitumgebung definiert.
%Mit folgendem Befehl werden alle floats dieses Typs ausgegeben
%\ifdeutsch
%\listof{Algorithmus}{List of Algorithms}
%\else
%\listof{Algorithmus}{List of Algorithms}
%\fi
%\listofalgorithms %Ist nur für Algorithmen, die mittels \begin{algorithm} umschlossen werden, nötig

\endgroup

\cleardoublepage

\iftex4ht
\else
%Optischen Randausgleich und Grauwertkorrektur wieder aktivieren
\microtypesetup{protrusion=true}
\fi

% END: Verzeichnisse


\renewcommand*{\chapterpagestyle}{scrplain}
\pagestyle{scrheadings}
\input{preambel/pagestyle}
%
%
% ** Hier wird der Text eingebunden **
%
\cleardoublepage
\chapter{Introduction}
\label{chap:introduction}

Resources of an organization play an important role to accomplish its intentions. Though organizations reuse resources, stored execution procedure of successfully completed processes, etc., the decisions that involve human knowledge cannot be reused in certain types of processes. These type of processes are not structured like traditional processes because the sequence of activities to be carried out in order to execute a process cannot be pre-defined due to its dynamic changing nature e.g research and development process.

The processes whose required activities and order of their execution cannot be determined beforehand are called \textit{Informal Processes} \cite{Sungur2014}. These type of processes are human-centric as their dynamic nature is due to the involvement of human knowledge. The next section provides a detailed motivational statement of this master thesis work followed by problem statement section which is then followed by contributions of this work. The final section provides an outline about the following chapters of the document. 

%%%%%%%%%%%%%%%%%%%%%%%%%%%%%%%%%%%%%%%%%%%%%%%%%%%%%%%%%%%%%%%%%%%%%%%%%
\section{Motivation}
\label{sec:motivation}
%%%%%%%%%%%%%%%%%%%%%%%%%%%%%%%%%%%%%%%%%%%%%%%%%%%%%%%%%%%%%%%%%%%%%%%%%
Nowadays, any task has both well defined predictable traditional processes and less defined ambiguous informal processes. In tasks with less defined ambiguous processes, knowledge workers' decision plays an important role \cite{BPTrends2009}. For example, research and development projects are of type where \textit{what to do next} cannot be decided in advance. These type of processes are highly unpredictable in nature and this makes it quite challenging to support and automate these type of processes. This work is a part in realizing the automation of such processes. Any approach that supports informal process automation is required to be more autonomous because of their dynamic behavior of processes are enacted by some subjects, so the existing approaches available for traditional processes are not helpful in realizing the execution of informal processes.  

Though sequence of steps to be carried out to execute informal processes cannot be determined beforehand, \textit{intentions} of informal processes are known before their enactment \cite{Sungur2015}. Achieving these intentions requires another important driving force called \textit{resources}. Resources can be anything from human actors, development environments, materials etc. These resources posses certain \textit{capabilities} to qualify for achieving an intention. So we need an approach that supports informal processes along with the support of intentions, resources, capabilities etc. This can be achieved by associating intentions with strategies, strategies with capabilities and capabilities with resources, i.e., modeling oriented to organizational intentions. When the models are designed descriptively, i.e., providing only information what has to be done in order to achieve an intention rather than how to achieve an intention they serve as informal guides which preserves the essential information associated with informal processes to achieve an intention. Meanwhile, it also overcomes the need for pre-defining the sequence of execution steps.

%%%%%%%%%%%%%%%%%%%%%%%%%%%%%%%%%%%%%%%%%%%%%%%%%%%%%%%%%%%%%%%%%%%%%%%%%
\section{Problem Statement}
\label{sec:problemstatement}
%%%%%%%%%%%%%%%%%%%%%%%%%%%%%%%%%%%%%%%%%%%%%%%%%%%%%%%%%%%%%%%%%%%%%%%%%
 Every organization contains multiple entities like \textit{resources} e.g., humans, tools etc., \textit{intentions} e.g., revenue based intentions, quarterly intentions etc., \textit{strategies} e.g., the process to achieve the intention and \textit{capabilities} e.g., a resource that can provide a particular capability. Thus an organization needs efficient mechanisms to handle and manage these different types of entities. Informal processes are collaborative in nature, which means that participants of informal process collaborate with each other to accomplish its intentions\cite{Sungur2015}. Designing these collaborations and assigning participants their respective privileges, plays an important role during modeling of the respective informal processes. The research work by Matthews et. al \cite{Matthews2011} mentions that below points are the major problems in adopting to a workspace collaboration tools.

\begin{enumerate}
	\item Lack of Methods
	\item Methods that focus on individuals
	\item Not well targeted groups
	\item Not well supported editors for executing abstract descriptions
\end{enumerate}

Though there are \textit{activity-centric} modeling and reusing of business processes such as Business Process Execution Language (BPEL) \footnote{http://docs.oasis-open.org/wsbpel/2.0/OS/wsbpel-v2.0-OS.pdf} and Business Process Model and Notation (BPMN) \footnote{http://www.omg.org/spec/BPMN/2.0/PDF/} are available, they are not suitable for certain type processes whose execution steps cannot be predicted in advance \cite{Sungur2014a}. Also complementary concepts such as automatic initialization and acquiring of interrelated resources are still missing in the existing work \cite{Sungur2015}. Another key thing to remember is informal processes are volatile in nature which is one of the important challenges in developing an environment that supports informal processes.

%%%%%%%%%%%%%%%%%%%%%%%%%%%%%%%%%%%%%%%%%%%%%%%%%%%%%%%%%%%%%%%%%%%%%%%%%
\section {Research Objectives}
\label{sec:researchobjectives}
%%%%%%%%%%%%%%%%%%%%%%%%%%%%%%%%%%%%%%%%%%%%%%%%%%%%%%%%%%%%%%%%%%%%%%%%%
The main focus of this work, is to realize the phase \textit{Informal Process Modeling} (P2) described in \textit{Executing Informal Processes} (InProXec) method \cite{Sungur2015}. Coupled with the main focus of developing web based editor, the following research objectives provided in the Table \ref{tab:researchobjectives} are also satisfied by the developed editor. 

\label{sec:researchobj}
\begin{center}
	\begin{longtable}{p{5cm}p{11cm}} 
   	\toprule 
	\textbf{Research Objectives} & \textbf{Description} \\
	\midrule
	\endfirsthead
	\\
	R1 & \textit{Organizational intentions transparency}  \label{ro1} \\
	\\[-1.5ex]
	R2 & \textit{Organizational intention resource-based cost estimation}  \label{ro2} \\
	\\[-1.5ex]
	R3 & \textit{Organizational intention achieve-ability estimation} \label{ro3}\\
	\\[-1.5ex]
	R4 & \textit{Intention oriented working style}  \label{ro4}\\
	\\[-1.5ex]
	R5 & \textit{Participative organizational modeling}\label{ro5}\\
	\\[-1.5ex]
	R6 & \textit{Re-use of organizational knowledge} \label{ro6}\\	
	\bottomrule
	\caption{Research Objectives}
	\label{tab:researchobjectives}
	\end{longtable}	
\end{center}

%%%%%%%%%%%%%%%%%%%%%%%%%%%%%%%%%%%%%%%%%%%%%%%%%%%%%%%%%%%%%%%%%%%%%%%%%
\section {Outline}
\label{sec:outline}
%%%%%%%%%%%%%%%%%%%%%%%%%%%%%%%%%%%%%%%%%%%%%%%%%%%%%%%%%%%%%%%%%%%%%%%%%
The remainder of this document is organized into following chapters:
\begin{description}
	\item[Chapter ~\ref{chap:fundamentals} -- \nameref{chap:fundamentals}:] In this chapter, fundamental concepts and an overview of the related approaches that are essential to understand the work are provided.
	\item[Chapter ~\ref{chap:motivatingScenario} -- \nameref{chap:motivatingScenario}:] In this chapter, a motivating scenario has been taken and detailed explanation for each phases of the scenario has been provided. This aids the reader to understand the concepts of organizational modeling clearly. 
	\item[Chapter ~\ref{chap:analysis} -- \nameref{chap:analysis}:] This chapter provides detailed requirement analysis based on scientific facts published in existing works. This chapter also provides literature review of existing works.
	\item[Chapter ~\ref{chap:approach} -- \nameref{chap:approach}:] This chapter discusses about the methodology followed in realizing the concepts  of resource-centric organizational.
	\item[Chapter ~\ref{chap:casestudy} -- \nameref{chap:casestudy}:] This chapter validates the approach presented in Chapter \ref{chap:approach}. This chapter also discusses detailed system architecture and also presents the validation results. The abstract concepts motivating scenario discussed in \ref{chap:motivatingScenario} is explained in a concrete way.	
	\item[Chapter ~\ref{chap:conclusion} -- \nameref{chap:conclusion}:] This chapter summarizes the results of the work and draws conclusion. This chapter also throws some light on the future work to be carried out in the approach of executing informal processes. 
\end{description}
\cleardoublepage
\chapter{Fundamentals and Related Work}
\label{chap:fundamentals}
%background - What do I need to know to understand the thesis?
This chapter provides the fundamental concepts and related work that are required to understand the approach to be discussed in following Chapter \ref{chap:approach}. The first section introduces definitions of terms that are used throughout this document. The second section provides a brief introduction about the basic concepts of this thesis work.  The third and fourth sections present description about  works related to the presented approach. The last section focuses on concepts of Resource-centric Organizational Modeling and its notations. Though modeling notations are not part of implementation, it has been introduced to assist the reader in better understanding. The last section also discusses about the process and entity representation of the organizational modeling.  

%%%%%%%%%%%%%%%%%%%%%%%%%%%%%%%%%%%%%%%%%%%%%%%%%%%%%%%%%%%%%%%%%%%%%%%%%
\section{Definitions of Terms}
\label{sec:termdefinitions}
%%%%%%%%%%%%%%%%%%%%%%%%%%%%%%%%%%%%%%%%%%%%%%%%%%%%%%%%%%%%%%%%%%%%%%%%%
In this section, the definitions of terminologies that are used throughout this document has been provided.

\textit{Business Process} -  A business process has been defined as the set of activities and tasks whose final output is accomplishment of a goal. These activities are performed in an organizational and technical environment \cite{Weske2012}.  Based on the type of input and the operation of tasks, these processes can be categorized as management process, support process, research process, development process, etc., \cite{Sungur2015}.   

\textit{Business Logic} - Business logic refers to the activities that need to be done to execute the corresponding process. 

\textit{Business Process Models} - Business process models are models to capture recurring procedures during a business process execution and enact them in a automated fashion for re-using those stored knowledge. 

\textit{Business Process Engines} - Business process engines can enact the business process models automatically once the configuration of necessary infrastructure has been carried out \cite{Sungur2015}.	

\textit{Business Process Management} - Business process management (BPM) includes concepts, methods, and techniques to support the design, administration, configuration, enactment, and analysis of business processes \cite{Weske2012}.

\textit{Business Process Management Life Cycle} - Business process management life cycle is the series of phases such as modeling, configuring, executing, and improving business process. These series phases are conducted as a cycle \cite{Weske2012}.

\textit{Informal Process} - The processes that human participate and create knowledge are called unstructured/informal/human-centric processes. In informal process, execution steps cannot be modeled or are not feasible to model before their enactments. This is because due to the dynamic changing behavior of execution steps of the informal processes.  For Example, software development process is an informal process, where required activities and order of their execution cannot be determined beforehand \cite{Sungur2015}. The four characteristic properties are: implicit business logic, varying relationships among resources, resource participation in multiple informal processes and changing resources \cite{Sungur2014a}.    

\textit{Informal Process Essentials} - Informal Process Essentials (IPE) is an intention-based approach that enables describing process declartively, i.e., without describing how the intention is achieved, and providing only information about what has to be achieved \cite{Sungur2014a}. 
 
\textit{Intention} - Intentions are defined hierarchically, which can contain and extend sub-intentions.It is depicted by a double circle in organizational notations. The sub-intentions are refined starting from main intentions. Intentions are associated with capabilities or resources. An accomplishment of an intention changes state. An intention can extend another intention.        

\textit{Resource} - A resource can be a people or tool those/that drive towards the successful execution of the process. It is key for achieving specified process intentions. In the context of our work, the definition of organizational resources refers not only the entities that are capable of doing work but also entities that have an impact on the outcome of the processes, e.g., software tools, human performers, data etc.      

\textit{Capability} - A capability is the ability to provide business values like software applications, resources, and potential of the actor to make decisions even in changing situations \cite{Stirna2012}.Describes a capability provided by a resource or required by an intention. The performers of an informal process have certain skills and roles to achieve the intention.   

\textit{Strategy} -  A  strategy is a method or plan chosen to bring  desired results, such as achievement of an intention or solution to a problem \footnote{http://www.businessdictionary.com/definition/strategy.html}. 

%%%%%%%%%%%%%%%%%%%%%%%%%%%%%%%%%%%%%%%%%%%%%%%%%%%%%%%%%%%%%%%%%%%%%%%%%
\section{Organizational Modeling Notations}
\label{sec:resourcecentricorganizationalmodeling}
%%%%%%%%%%%%%%%%%%%%%%%%%%%%%%%%%%%%%%%%%%%%%%%%%%%%%%%%%%%%%%%%%%%%%%%%%
The Organizational Modeling element notation has been selected as per the guidelines mentioned in the literature \cite{Moody2009} and these notations are taken from the thesis work by the author Sierra\cite{Sierr2015}. Though these notations modeling are not part of this diploma thesis, this has been mentioned in this section for the sole purpose of aiding the reader to understand the concepts much better through pictorial representations. Also by observing  the fact that business process modelers are already well-known with the present process modeling notations such as Business Process Modeling Notation 2.0 (BPMN) \cite{bpm2011} and ArchiMate notation\cite{arc2013}, the shape depiction of organizational model elements has been designed in the previous work \cite{Sierr2015} similar to those existing process notations. 

Due to the importance of shapes in expressing the information visually , the notations are chosen in such a way that each element of Organizational Modeling  differ by shape. Also a legend will be always shown in the modeling notation to denote the meaning of each shape \cite{Moody2009}. The description of each element in the Organizational Model Notation is shown in the Table \ref{tab:notations}. 

\begin{center}
	\begin{longtable}{p{3cm}p{10cm}p{3cm}}
		\toprule 
		\textbf{Element} & \textbf{Definition} & \textbf{Notation} \\
		\midrule
		\endfirsthead
		Intentions 			& Intentions are purposeful concrete steps taken by organizations or individuals to achieve an expected outcome. & \begin{center} \includegraphics[width= 0.07\textwidth]{Intention.pdf}  \end{center}  \\
		
		Capabilities	&  Capability is an ability that should be possessed by a resource that work towards achievement of one or several intentions.   & \begin{center} \includegraphics[width= 0.07\textwidth]{Capability.pdf} \end{center}  \\
		
		Context				& The environment that forms the setting for an event, statement, or idea, and in terms of which it can be fully understood. There are two Contexts: Initial and Final. The Initial Context is the situation which describes the driving forces that trigger the process to start. The Final Context is the expected situation once the process has finished. Both initial and final context are represented by an hexagonal shape except the final context has thick edges than initial context.  & \begin{center} \includegraphics[width= 0.07\textwidth]{Context.pdf} \end{center}   \\
			
		Strategy		&  A method or plan chosen to bring about a desired future, such as accomplishment of an intention.   & \begin{center} \includegraphics[width= 0.07\textwidth]{Strategy.pdf} \end{center}  \\
		
		Resources					& The people or tools those/that needed to fulfill the middle objectives or work towards the achievement of intentions . & \begin{center} \includegraphics[width= 0.07\textwidth]{Resource.pdf} \end{center}  \\
		
		Relationship				& A relationship is used specify the fixed links between the elements of the model.  & \begin{center} \includegraphics[width= 0.07\textwidth]{Relationship.pdf} \end{center}   \\
		
		\bottomrule
		\caption{Organizational Modeling Notations}
		\label{tab:notations}		
	\end{longtable}	
\end{center}

%%%%%%%%%%%%%%%%%%%%%%%%%%%%%%%%%%%%%%%%%%%%%%%%%%%%%%%%%%%%%%%%%%%%%%%%%
\section{Entity Types Representation}
\label{sec:entitytypesrepresentation}
%%%%%%%%%%%%%%%%%%%%%%%%%%%%%%%%%%%%%%%%%%%%%%%%%%%%%%%%%%%%%%%%%%%%%%%%%

The conceptual model of entity types in organizational modeling are shown in the Figure \ref{fig:entitymodel}. This model shows that among all the entity types, intentions are in the top level of hierarchy which can be further divided into \textit{sub-intentions} and/or \textit{strategies}. An intention can either contradict or be a sub intention of another intention. These type of sub-intention and contradicting intention has been explained in detail with a suitable example in Section \ref{sec:intentions}.  An intention can be achieved through a strategy, which is a plan of action designed to meet the intention. An intention can be achieved through none or many strategies. Strategies also describe none or many capabilities and processes required to achieve intention. The capabilities and processes can be further resolved into resources or resource models. Thus starting from defining intentions, we define strategies then required capabilities and process models. The capabilities and process models define the required resources. 
 
As reported by Sungur et al. \cite{Sungur2014a}, the concept of IPE provides an agent-based approach i.e., human performers are considered as agents who execute the processes autonomously. Organizational Process Modeling is \textit{Resource-centric} approach as they support processes by providing required resources and thrives to successfully execute the processes by using qualified autonomous agents, i.e., actors under certain \textit{context definitions}.  As we mentioned before, in our context resources can be anything like people, IT tools, data that are used to accomplish the objectives. Emerging intentions can result in the requirement of new capabilities, i.e., resources. Resource models are also provided in the developed prototype to make precise definitions of resources needed.


\begin{figure}
	\centering
	\includegraphics[width=\textwidth]{entity.pdf}
	\caption{Organizational Modeling Entity Types Representation}
	\label{fig:entitymodel}
\end{figure}

In Sungur et al \cite{Sungur2014a} work, the concept of \textit{Informal Process Support Model} (IPSM) has been introduced which is to make use of existing knowledge of human performers. Here the initial creator of the model is experienced human performers. Based on their experience, they add relevant  resources of an informal process. The models are generated at runtime based on the interactions and activities of corresponding human performers. An informal process targets for accomplishment of an intention. The intentions can be refined by defining sub-intentions and/or strategies, which can then be further refined recursively as independent informal processes. The intention-based approach enables describing processes declaratively, i.e., without describing \textit{how} the intention is achieved, and providing only information about \textit{what} is achieved. As the author \cite{Sungur2014a} suggests that this avoids need of predefined business logic in the representations of informal processes. Each resource can be related to another resource in the context of an informal process using predefined or custom \textit{Relationships}. Informal Process Essentials are realized through strategies. Each informal process starts from an initial context, i.e., \textit{initial context} and aims to achieve an intention. After accomplishing the intention, there is a resulting context called as \textit{final context}. The beginning state before achieving intention is called as initial context and the end state after achieving intention is called as final context. On completion of intention execution, the process state changes from one state to another.

%%%%%%%%%%%%%%%%%%%%%%%%%%%%%%%%%%%%%%%%%%%%%%%%%%%%%%%%%%%%%%%%%%%%%%%%%
\section{Overview of Informal Process Essentials}
\label{sec:basicconcepts}
%%%%%%%%%%%%%%%%%%%%%%%%%%%%%%%%%%%%%%%%%%%%%%%%%%%%%%%%%%%%%%%%%%%%%%%%%
In this section, we provide an overview about the concepts introduced in the approach Informal Process Essentials (IPE) \cite{Sungur2014a}. Models are used in various fields like manufacturing, scientific, IT, etc. These models are mainly useful in re-using the predefined regular, intelligible and field-tested solutions. Such models has numerous benefits \footnote{http://www.nomagic.com/getting-started/modeling-benefits.html} like performance improvement, reduced cost of operation and design, etc. Besides these processes there are processes which requires participation of human and performance of these processes depend on human knowledge, i.e., they are subject to change and carried out based on experience of previous knowledge. These processes are called \textit{Informal Processes} and they do not have formal structured execution of steps for the enactment of processes. 

The work by Sungur et. al \cite{Sungur2014a} gives a comprehensive account of challenges in defining the business logic of informal processes as below:

\begin{itemize}
	\item The structure of informal processes are not known before enactment of the processes
	\item Results in less flexible and less efficient solutions
	\item The cost of creation of well-defined business logic is too high
\end{itemize}

This thesis work realizes the concept of \textit{resource-centric modeling of informal processes}, specified in the Informal Process Essentials(IPE) approach by Sungur et al \cite{Sungur2014a}.  As mentioned in the Section \ref{sec:termdefinitions}, resources are drivers to achieve intentions in the informal processes. In this approach author states that, when the desired process result is repeated the same set of resources can be selected and engaged towards collective intention of the informal processes. Also it has been mentioned that each IPE model contains the list of necessary resources to accomplish the main intention of the respective informal process.

In IPE approach, the author differentiates the resources based on the time. The resources that are needed in the informal processes are below :
  \begin{itemize}
  	\item \textit{Initial resources} which are required during the start of informal processes.
  	\item \textit{On-demand resources} that are required based on intentions during process enactment.
  	\item \textit{Actors} are the resources in IPE meta-model, that drive process execution autonomously.
  	\item \textit{Knowledge resources} resources that contain important information required for the enactment of a process. These are critical for guiding actors.  
  \end{itemize}
 
It has been mentioned in the approach \cite{Sungur2014} that Informal Process Essentials (IPE) describes the following about informal process  

\begin{itemize}
	\item Describes the constituents informal process such as performers, data and software tools
	\item Describes how to make core element ready for the enactment of the informal process i.e resource providers
\end{itemize}

Predefining business logic would results in higher cost compared to making decisions by human performers during enactment \cite{Sungur2014}. Sometimes, a process team may require participation of  new resources with different roles and relationships from a different team \cite{Matthews2011,Matthews2012}. For example, in our motivating scenario we have two teams  software development team and help desk team. To improve the user feedback portal, help desk team may require resources from software development team with a role of user interface web developer. Thus to satisfy requirement changes, resources are also changeable during process execution. 

%%%%%%%%%%%%%%%%%%%%%%%%%%%%%%%%%%%%%%%%%%%%%%%%%%%%%%%%%%%%%%%%%%%%%%%%%
\section{Human Centric Process}
\label{sec:humancentric}
%%%%%%%%%%%%%%%%%%%%%%%%%%%%%%%%%%%%%%%%%%%%%%%%%%%%%%%%%%%%%%%%%%%%%%%%%
The role of humans in organizations has been evolving over time. The shift from "personnel" to "human resources" acknowledges the importance of humans as organizational resources. There are incredible number of pressure on today's organizations \footnote{http://www.siop.org/tip/backissues/tipjan98/may.aspx} due to varying dynamic nature of organizations. For example, organizational changes like addition of new organizational alliances, new structures and hierarchies, new ways of assigning work, and a very high rate of changes like changes in the workforce, including employees' priorities, capabilities, and demographic characteristics. Thus it is impossible to do one hundred percent perfect forecasting of dynamically changing activities or processes in an organization.

In order to manage such a dynamic environment, organizations need skilled human resources with previous knowledge of handling unforeseen scenarios. Thus human resources are vital part of any organizations as they have skills of acute future orientation to understand changing organizational environment. Humans in an organizations carry out many important activities. Managers and Human Resource (HR) professionals organizes jobs of each and every human in the organization so that they can effectively perform these jobs. Thus humans in any organization are viewed as resources of the organization which is a contemporary part of Human Resource Management \footnote{http://smallbusiness.chron.com/role-human-resource-management-organizations-21077.html}.

When there are multiple human resources working for a process, then there should be some sort of co-ordination and understanding between the humans which is called \textit{collaboration} at an organizational level. Collaboration exists in every levels of an organization. For example at management levels of an organization, managers and HR professionals work together to assign employees their roles and task in the organization. This helps the employees of the organization to adapt to its environment. In a flexible organization, employees roles and responsiblities changes dynamically based on the requirements and business priorities. Thus the need for network of representations between the human resources which sets up an environment to support collaborative work of business related process has been realized in the work by the author Canko\cite{Canko2015}. The concept of \textit{virtual human representation} is an extension of  actor-concept described in  \textit{Informal Process Essentials} \cite{Sungur2014a}. The developed prototype \textit{Human Resource Representation} in the work by the author Canko\cite{Canko2015} saves the information such as capabilities, roles, responsibilities etc.  as a virtual human web ontology instance which can be re-used in web based environments.

These kind of human representation are highly helpful to organizations with dynamically changing processes. These representations can describe and match resources with their capabilities based on the requirements. As we have mentioned in Chapter \ref{chap:introduction}, in our context of resource-centric modeling humans are also considered as resources and we associate \textit{capabilities} with every resources. Moreover, associating capabilities, with resources is helpful in following situation. There can be a situation where resources producing more accurate results for a processing task are preferred than resources which can produce higher throughput for a processing task. Thus we need to associate capabilities with each resources and need to automate the process of discovering and matching the resources with their capabilities based on their process. 

%%%%%%%%%%%%%%%%%%%%%%%%%%%%%%%%%%%%%%%%%%%%%%%%%%%%%%%%%%%%%%%%%%%%%%%%%
\section{Second Phase of InProcXec}
\label{sec:inproxec}
%%%%%%%%%%%%%%%%%%%%%%%%%%%%%%%%%%%%%%%%%%%%%%%%%%%%%%%%%%%%%%%%%%%%%%%%%
In this section, we present an overview about the \textit{Executing Informal Processes} (InProXec) method, proposed by Sungur et al. \cite{Sungur2015}. Since this thesis work is realizing resource-centric modeling of organizations, the main focus of this section is on the second phase of InProXec which is \textit{Informal Process Modeling}(P2). The method described in Figure \ref{fig:inprocxec_steps}, initializes informal process models in an automated fashion. In the following paragraphs, we present short overview about different phases of the InProXec method which has been decribed in detail in the article \cite{Sungur2015} and with a detailed description about the second phase of the \textit{InProXec} method. 

\begin{figure}
	\centering
	\includegraphics[width= \textwidth]{InProXec_Steps.pdf}
	\caption{Steps of InProXec method \cite{Sungur2015}}
	\label{fig:inprocxec_steps}
\end{figure} 

As shown in the Figure \ref{fig:inprocxec_steps}, the InProcXec method consists of four different phases.

\textit{Integrating Resources of Informal Processes (P1)} - In order to model an informal process, we need information about the resources, these information are collected beforehand during process execution. There exist many services to acquire information about informal processes resources automatically. The final output of this phase is integrated resources which are required as an input to next modeling phase. Thus this phase sets up an environment required for modeling and execution of informal processes.  

\textit{Informal Process Modeling (P2)} - This phase receives the resource definitions made available in the first phase P1 as an input.  Based on this, business experts model informal processes and associated entities like strategies, intentions, capabilities etc., using our developed web editor. This phase has been explained in detail in the following sub section \ref{subsec:informalprocessmodeling}    

\textit{Informal Process Compilation (P3)} - The previous phase P2, describes only the intentions required to be achieved, corresponding required resources etc. But in phase P2, the functionality to instantiate acquirable entities are not included. Thus in third phase P3, the output of phase P2 is taken i.e IPE models and are transformed into intializable self-contained \textit{Deployable Informal Process Essentials Archives(DIPEA)} \cite{Sungur2015} takes place. This results in DIPEAs enacting required informal process. To realize, phase P3 an \textit{IPE Model Compiler} also been introduced. 

\textit{Informal Process Model Deployment and Runtime (P4)} This phase employs \textit{IPE Runtime} which parses DIPEAs and runs the executables contained in those archives. During this phase, the autonomous actors work towards intentions of informal processes using acquired resources and other involved resources.  

%%%%%%%%%%%%%%%%%%%%%%%%%%%%%%%%%%%%%%%%%%%%%%%%%%%%%%%%%%%%%%%%%%%%%%%%%
\subsection{Informal Process Modeling (P2)}
\label{subsec:informalprocessmodeling}
%%%%%%%%%%%%%%%%%%%%%%%%%%%%%%%%%%%%%%%%%%%%%%%%%%%%%%%%%%%%%%%%%%%%%%%%%
This approach of Informal Process Modeling is directed towards modeling the informal process based on their intentions rather than their activities. The developed prototype serves as an holistic web based editor to create, view and update all the associated entities of informal process like intentions, capabilities, strategies etc., along with informal process. Also from our detailed explanation in previous sections about the importance of resources in organizational modeling  and along with the fact that phase P2 receives resource defintions as input from phase P1 of InProXec method we can apprehend that resource definitions are the lowest level in the hierarchy of resource-centric organizational modeling approach. The sequence of steps to be carried out using the developed editor has been shown in the Figure  \ref{fig:processdiagram}. It is important to note that in the figure, only solid round edged rectangles are part of the developed editor. The tasks to be carried out in each of the steps in developed editor is described as below:

\begin{figure}
		\centering
		\includegraphics[width=\paperwidth,angle=90]{processmodeling.pdf}
		\caption{Organizational Process Modeling}
		\label{fig:processdiagram}
\end{figure}

\textit{Model Context Definitions} (M1) -  The first step is to model context definitions, where we can model both basic properties like name and namespace of a context definition and entity specific properties like contained contexts, entity definitions etc., of a context definition.  

\textit{Model Intentions} (M2) -  Similar to context definition modeling (M1), the second step (M2) is to model intentions. The context definitions created in step M1 can be used to specify initial and final contexts of an intention. Intentions can contain sub-intentions and contradicting intentions. These type of sub intentions and contradicting intentions are also modeled as intentions in this step and their type of relation to specific intention are mentioned. 

\textit{Model Strategies} (M3) -  Once intentions are identified and modeled, the third step is modeling of strategy to achieve a specific intention. As mentioned earlier in Section \ref{sec:entitytypesrepresentation}, an intention can have multiple strategies. 

\textit{Model Required Capabilities} (M4) - After modeling of strategies, capabilities required to achieve an intention in a specific strategy is modeled. A strategy can require multiple capabilities which has been explained in detail with a suitable example in the following Chapter \ref{chap:motivatingScenario}. 

\textit{Create Resource Models} (M6) -  After matching the resources and capabilities i.e after finding the correct resource that has the capability to carry out the process, the resource models are created. The need for modeling a new intention may arise in parallel this has been explained with a suitable example in the following Chapter \ref{chap:motivatingScenario}.   

\textit{Extract as an IPE Model} (M7) -  After the completion of above mentioned steps, the modeled entities can be extracted as an IPE model which can be reused. 

The other steps denoted in dashed round edged rectangle which are not part of developed web editor includes matching of required organizational capabilities that are satisfied by resource models  to enable the achievement of organizational intentions in certain context definitions through a strategy. If there is no suitable matching capability then phase P1 of InProXec can be carried out again until a matching capability is found. If Capabilites are satisfied resource models can be created. The created resource models(M6) along with modeled capabilities can be extracted as an IPE Model(M7) which will be provided as input for the next step execution of intentions (M8). After the execution of an intention, the status of an intention is updated inside the specific intentions's property. 











\cleardoublepage
\chapter{Motivating Scenario}
\label{chap:motivatingScenario}
In order to help in understanding the concepts of organizational modeling, a motivating scenario has been taken and explained through the modeling notations mentioned in the Section \ref{sec:resourcecentricorganizationalmodeling} of Chapter \ref{chap:fundamentals}. This scenario also helps to validate the developed web-based modeling tool in the Section \ref{sec:validation} of Chapter \ref{chap:casestudy}. The motivating scenario has been chosen based on the collected real life scenarios provided in another thesis work \cite{Sierr2015}. The motivating scenario is taken from the context of manufacturing sector. 

In this chapter, the first section provides a brief introduction about the motivating scenario. The last section provides an explanation about the organizational modeling elements discussed in the motivating scenario. 
%%%%%%%%%%%%%%%%%%%%%%%%%%%%%%%%%%%%%%%%%%%%%%%%%%%%%%%%%%%%%%%%%%%%%%%%%
\section{Intention-oriented Organizational Modeling Example}
\label{sec:scenario}
%%%%%%%%%%%%%%%%%%%%%%%%%%%%%%%%%%%%%%%%%%%%%%%%%%%%%%%%%%%%%%%%%%%%%%%%%
The concept of intention oriented organizational modeling can be explained with the following scenario taken from a manufacturing organization. Consider a budding manufacturing company which designs, develops, manufactures and sells personal computers and laptops. The CEO's main intention of the quarter is \textit{to increase the revenue and number of unit sales}. Intentions connect initial context definitions with final context definitions \cite{Sungur2014a}. There are also low level intentions other than the main intention which helps in achieving main intention as a collection of several intentions in a measurable form. 
 
\begin{figure}
  	\centering
  	\includegraphics[width=\textwidth,angle=0]{MotivatingScenario.pdf}
  	\caption{Intention-oriented Organizational Modeling - Example Scenario}
  	\label{fig:motivatingscenario}
\end{figure}
  
The Figure \ref{fig:motivatingscenario} provides the details of organizational intentions, strategies, capabilities and resources. There can be multiple strategies followed to achieve a main intention. The main intention in the motivating scenario can be achieved by following all of the below mentioned strategies. These strategies require resources with matching capabilities.
 
 \begin{enumerate}
 	\item Through increasing the revenue by expanding the market sales. 
 	\item Through increasing advertisements, which helps the customer to know about the product.
 	\item Through improving the existing customer help desk portal, as it helps to maintain good customer relationship.
 \end{enumerate}
 
%%%%%%%%%%%%%%%%%%%%%%%%%%%%%%%%%%%%%%%%%%%%%%%%%%%%%%%%%%%%%%%%%%%%%%%%%
\section{Intention-oriented Organizational Modeling Elements}
\label{sec:entities}
%%%%%%%%%%%%%%%%%%%%%%%%%%%%%%%%%%%%%%%%%%%%%%%%%%%%%%%%%%%%%%%%%%%%%%%%%
It is important to explain each of the organizational modeling element using an example, as it helps in understanding the requirements of intention-oriented organizational modeling discussed in the Section \ref{sec:requirementssupoorting}. Before we proceed with detailed description of each modeling element, we provide an example scenario to know the dynamic nature of the organizational modeling. For example, in our above mentioned motivating scenario in the Section \ref{sec:scenario} one of the intention is to \textit{expand sales geographically}. To achieve this intention successfully, few ground works like collection of laptop usage statistics such as average buying capacity of the consumers, average computer knowledge of the people in new geographic location has to be done. Thus, the main intention, i.e., \textit{increase revenue and number of unit sales}, requires collaboration of people with different skills and expertise. For example, resources with capability to do market analysis are required. If in case none of the organizational resources provide required capability, then the organization can get it served from external resources or further modularize the intention so that it can be provided by internal resources itself. This makes to emerge new intentions dynamically. The team working towards achievement of main intention should also be ready to accommodate new resources with new capabilities and skills. For example, there is a software development team, which work towards achievement of the intention \textit{improve customer help desk portal}, i.e., this team develops software that automatically attends and records user queries. Suppose, if there arise a new requirement of \textit{supporting help desk through mobile applications} then the system should accommodate new resource with \textit{mobile application developer} capability. 

%%%%%%%%%%%%%%%%%%%%%%%%%%%%%%%%%%%%%%%%%%%%%%%%%%%%%%%%%%%%%%%%%%%%%%%%%
\subsection{Contexts} 
\label{sec:contexts}
%%%%%%%%%%%%%%%%%%%%%%%%%%%%%%%%%%%%%%%%%%%%%%%%%%%%%%%%%%%%%%%%%%%%%%%%%
The execution of the manufacturing processes such as the one provided in Figure \ref{fig:motivatingscenario} are not similar to execution of typical business processes. This is because, the execution of manufacturing processes mostly depends on the information collected from the real world, i.e., the execution context \cite{Sungur2016}. A context definition provides mechanism to act adaptively based on the current situation. This is achieved in the production environment by describing each process with a specific context definition \cite{Sungur2016}. For example, in our motivating scenario the initial context provides details about status before achievement of the main intention, i.e., it specifies the situation of the organization which triggers the execution of main-intention. The actual problem context is, the revenue of the previous quarter was lesser than the estimated revenue. Hence, the initial context for next quarter is set as \textit{quarterly goal of increasing the revenue and number of unit sales}. The initial context helps to decide the main intention and its related low level associates. On successful achievement of main-intention, the intention reaches desired state which is called as final context. Along with successful reaching of the final context, this also provides tools such as web-based help desk portals, automated ad software etc., that are developed as part of this intention achievement. When the final context definition has been reached the process completion starts. This process final state can be stored and same set of resources can be re-used in future executions with similar contexts and intentions \cite{Sungur2015}.
 
%%%%%%%%%%%%%%%%%%%%%%%%%%%%%%%%%%%%%%%%%%%%%%%%%%%%%%%%%%%%%%%%%%%%%%%%%
\subsection{Intentions} 
\label{sec:intentions}
%%%%%%%%%%%%%%%%%%%%%%%%%%%%%%%%%%%%%%%%%%%%%%%%%%%%%%%%%%%%%%%%%%%%%%%%%
The intentions are defined hierarchically in the motivating scenario. The intentions are located at top level of the hierarchy, which are refined until concrete lower level of the hierarchy is reached. In the motivating scenario, intentions are not associated with capabilities directly, instead intentions are associated with strategies which are then associated with capabilities. For example, in our motivating scenario the main intention is to increase revenue and number of unit sales which also has other low level intentions such as \textit{improving the customer help desk portal} and strategies such as (1) through expanding sales and (2) through advertisements. The relation between strategies and intentions are denoted by the term \textit{contains} in Figure \ref{fig:motivatingscenario}. This because through strategies, intentions can be achieved. There can be situation where an intention can be related to another intention. There can be custom relationships between intentions such as contains, contradicts, etc. For example, consider in our motivating scenario the intention \textit{implement an automated ad software} can also contain an intention \textit{implement a mobile application}.  

%%%%%%%%%%%%%%%%%%%%%%%%%%%%%%%%%%%%%%%%%%%%%%%%%%%%%%%%%%%%%%%%%%%%%%%%%
\subsection{Strategies} 
\label{sec:strategies}
%%%%%%%%%%%%%%%%%%%%%%%%%%%%%%%%%%%%%%%%%%%%%%%%%%%%%%%%%%%%%%%%%%%%%%%%%
As mentioned earlier, a strategy is an approach, a manner or a means to achieve an intention \cite{Bider2005}. Strategies are associated with both intentions and capabilities. Each strategy needs certain capabilities to successfully accomplish an intention. We need to associate strategy with a capability that has matching resource. The resources are the potential holder of the capability, i.e., to satisfy a capability we need resources. The capability and its associated resources are also shown in the Figure \ref{fig:motivatingscenario}. In our motivating scenario, the main intention can be achieved through two strategies \textit{through expansion} and \textit{through advertisements}. These two strategies further contain the intentions such as \textit{expand geographically}, \textit{expand based on target customers} and \textit{implement an automated ad software}. Since, strategies contain intentions they are related through the term \textit{contains} in the Figure \ref{fig:motivatingscenario}. As mentioned earlier, the informal process models are realized through strategies. This is achieved through strategy containing capabilities and resources. For example, consider a small part in our motivating scenario of achieving an intention \textit{expand geographically} through strategy \textit{product sales distribution}. This strategy is chosen because the products will reach customer only if it is effectively distributed. To achieve this intention, through a specified strategy we need resources with the product sales distribution capability, i.e., resources that has an ability to effectively distribute the products. For example, sales agents, wholesalers or other kinds of sales distributors. 

%%%%%%%%%%%%%%%%%%%%%%%%%%%%%%%%%%%%%%%%%%%%%%%%%%%%%%%%%%%%%%%%%%%%%%%%%
\subsection{Capabilities}
\label{sec:capabilities}
%%%%%%%%%%%%%%%%%%%%%%%%%%%%%%%%%%%%%%%%%%%%%%%%%%%%%%%%%%%%%%%%%%%%%%%%%
The organizational resources posses certain capabilities to work towards the achievement of an intention. Each organizational capability must be provided by a resource in the organization. In our context, capabilities that are associated with resources are called as \textit{functional capabilities}. The type of capability that contains functional capabilities are called as \textit{cross-functional capabilities}. Strategies are associated with cross-functional capabilities, which contains functional capabilities out of which resources are created. In our motivating scenario to achieve a main intention, we need several capabilities such as product sales distribution capability, front end developer capability etc. For example, we need front end developer capability to execute the strategy \textit{through application development}, i.e., resources that has ability to develop an application's front end. In the Figure \ref{fig:motivatingscenario}, strategies and associated capabilities are related through the term \textit{requires}. This is because strategies require capabilities for execution. 

%%%%%%%%%%%%%%%%%%%%%%%%%%%%%%%%%%%%%%%%%%%%%%%%%%%%%%%%%%%%%%%%%%%%%%%%%
\subsection{Resources} 
\label{sec:resources}
%%%%%%%%%%%%%%%%%%%%%%%%%%%%%%%%%%%%%%%%%%%%%%%%%%%%%%%%%%%%%%%%%%%%%%%%%
The organizational resources of an organization can be anything that satisfies required capability to achieve an intention. Each resource have different types of relationship with other resources based on how they communicate with other resources \cite{Sungur2015}. For example, in our motivating scenario described in Section \ref{sec:scenario}, has an intention to \textit{improve customer help desk portal}. This intention can be achieved by providing skills improvement training to the existing employees or by recruiting newly skilled employee. Here the manager of HR department has permissions to decide whether to improve skills of existing employee or recruit new employee. But the team lead has only restricted permission like what type of skills are required for the project and also decision of team lead depends on decision of manager. Thus, manager and team lead are related in this simple example. 

\cleardoublepage
\chapter{Analysis of Resource-centric Organizational Modeling}
\label{chap:analysis}

This chapter positions the thesis work with respect to the other existing approach. The first section provides detailed requirement analysis about the research objectives described in Chapter \ref{chap:introduction}. The second section provides a detailed literature review about the existing approaches. This literature review is used to evaluate the proposed approach in the following Chapter \ref{chap:approach}.

%%%%%%%%%%%%%%%%%%%%%%%%%%%%%%%%%%%%%%%%%%%%%%%%%%%%%%%%%%%%%%%%%%%%%%%%%
\section{Requirement Analysis}
\label{sec:requirementssupoorting}
%%%%%%%%%%%%%%%%%%%%%%%%%%%%%%%%%%%%%%%%%%%%%%%%%%%%%%%%%%%%%%%%%%%%%%%%%
This section provides a detailed requirement analysis of research objectives mentioned in Chapter \ref{chap:introduction}. The below mentioned requirements are part of the functioning system proposed in the following Chapter \ref{chap:approach}. Also the exact phase when each of the requirements get satisfied are provided in Table \ref{tab:subrequirements}.

\textit{Organizational intention transparency (R1)}:  An intention can be broken down into definitive actionable components, or sub-intentions, upon which individual resources can act. When these lower level sub-intentions are made  achievable for individual resources, they can be combined to provide successful execution of higher level intention. Different organizational members can observe lower level and higher level intentions in their organizations. Intentions are traceable in the different levels of the organizational hierarchy. This means that the status of each intention can be accessed by members in different levels of the organizations. This kind of transparency within an organization reduces inefficiencies in intention execution, and is a key factor in attracting and retaining high  performers in the labor market \cite{McManus2007}. Requirement R1 has to be satisfied in the modeling phase itself as the designing of intentions, sub-intentions, strategies are done during the modeling time. 

\textit{Organizational intention resource-based cost estimation (R2)}:Linking intentions with strategies enable us a cost estimation for each intention. This is because intentions are realized through some strategies, strategies are associated with organizational capabilities which in turn has been associated with organizational resources. Cost is estimated in a recursive manner which has been explained in detail with an example in the following Chapter \ref{chap:approach}. To incorporate the cost estimation of intentions, we have to understand the recursive structure of the intentions associated with strategies. Since intentions are defined hierarchically, they can contain and extend intentions. Here strategy represents a means for achieving the intention. Further on, the cost of a strategy can be analyzed using the costs of derived sub-intentions, process definitions and so on. Including resources cost in intention cost calculation is important. This is achieved by associating resource models' cost with process models' cost. The recursion is stopped when the intention derivation process reaches the operational level. At the moment a  intention is achieved, some resources should be allocated to maintain the desired state (intention maintenance costs)\cite{Mandic2010}. Allocation of resources is mainly done at the operational level,hence requirement R2 has to be satisfied during the deployment phase.

\textit{Organizational intention achievability estimation (R3)}: The sub-intentions are projections of their super intentions, and satisfaction of the sub-intentions ensures satisfaction of the super intentions. Hence validity of an organizational intention is achievable when the intentions can be refined by defining sub-intentions, which can then be defined recursively as strategy and then to independent informal process models. Lower-level requirements can be validated against higher-level intentions, thus enabling validation of strategic alignment of  higher level intentions. The objectives of business strategy are found in the highest levels of the intention model.\cite{Bleistein2006}.Requirement R3  can be found during the modeling time of the process models as intention achieveability estimations are done before starting the execution of the intention based on the related intentions.

\textit{Intention oriented working style (R4)}: As each member of the organization is aware of the higher level and lower level intentions and he can engage for these explicit intentions. Intention orientation is the degree to which a person or organization focuses on tasks and the end results of those tasks. Strong intention orientation advocates a focus on the ends that the tasks are made for instead of the tasks themselves and how those ends will affect either the person or the entire company. Those with strong intention orientation will be able to accurately judge the effects of reaching the intention as well as the ability to fulfill that particular intention with current resources and skills \cite{Lacom}. The distinction between explicit knowledge of each sub intentions should not be seen as a division but rather as a continuum which aligns towards achieving the higher level intention . Though Requirement R4 itself has sub-requirement of R1, R4 has to be done at the run time which makes it distinct from the Requirement 1.

 \textit{Participative organizational modeling (R5)}: Different members of an organization participate to create organizational intentions, as a result intentions are shaped based on all members but directed by the executives. The  social  extension  of  a  business  process  can  be  regarded  as  a  process optimization phase, where the organization seeks efficiency  by  extending  the  reach  of  a  business  process  to  a  broader  class  of  stakeholders\cite{Brambilla2012}.Requirement R5 would be done at the run time as the input from different members of the organization provided during the process execution. But the list of participants who can have the priviliges such as own/edit/follow/view access to the models can be determined beforehand.
 
 \textit{Re-use of organizational knowledge (R6)}: Intentions specific solutions can be extracted as abstract re-usable entities, organizational strategy patterns and can be re-used in multiple context definitions. These field tested solutions are made as descriptive model informations which can be re-used.  Re-using the informations as models from the previous executions trims out the model designing time \cite{Yu2000}.


\begin{table} [htbp]
	\centering
	\begin{tabular} {p{4cm}p{3cm}p{9cm}}
		\toprule
		\textbf{Requirements}                                                      & \textbf{Requirement Satisfaction Phase} & \textbf{Pre-requisites}    \\
		\midrule                                                                                                               
		Requirement 1 (R1)                    & Modeling phase                 &\begin{tabular}[c]{@{}l@{}}1. Main intention can be refinable into sub-intentions.\\ 2. Organizational members can view the intentions \\ at different levels.\end{tabular}                \\ 
		
		Requirement 2 (R2)                      & Deployment phase               &\begin{tabular}[c]{@{}l@{}}1. intention cost estimation that includes all recursive \\ sub-intentions and resources.\\ 2. Cost estimation including the strategy. \end{tabular}                \\         
		
		
		Requirement 3 (R3)                     & Modeling phase            &\begin{tabular}[c]{@{}l@{}}1. Each sub-intention should be achievable and valid.     \end{tabular}                \\      
		
		Requirement 4 (R4)                     & Deployment phase               &\begin{tabular}[c]{@{}l@{}}1. Satisfaction of R1.\\  2.  Understanding of the intentions and \\ how they can be reached.   \end{tabular}                \\                         
		
		
		
		Requirement 5 (R5)                      &Both Modeling phase and Deployment phase              &\begin{tabular}[c]{@{}l@{}}1. Satisfaction of R1.\\  2. The output of intention is based on the inputs \\ provided by different members of the organization.   \end{tabular}                \\ 
		
		Requirement 6 (R6)                      & Modeling phase              &\begin{tabular}[c]{@{}l@{}}1. Satisfaction of R1,R2,R3,R4 and R5\\     \end{tabular}                \\     
		
		\bottomrule
	\end{tabular}
	\caption{Requirements Satisfaction Phase}
	\label{tab:subrequirements}
\end{table}

%%%%%%%%%%%%%%%%%%%%%%%%%%%%%%%%%%%%%%%%%%%%%%%%%%%%%%%%%%%%%%%%%%%%%%%%%
\section{Literature Review}
\label{sec:literaturereview}
%%%%%%%%%%%%%%%%%%%%%%%%%%%%%%%%%%%%%%%%%%%%%%%%%%%%%%%%%%%%%%%%%%%%%%%%%
In the literature, several work has been done in order to support and automate the business processes such as strategy-driven \cite{bider2005strategy}, activity-centric\cite{Yarosh2009}, activity-oriented \cite{Reijers2006}, artifact-centric \cite{Cohn2009}, capability-driven \cite{Stirna2012}, archimate \cite{Aldea2015} and subject-oriented \cite{Fleischmann2013}.  A detailed description about these approaches and their degree of satisfying the requirements mentioned in Section \ref{sec:requirementssupoorting} has also been provided. 

\textit{Strategy-Driven} : This approach \cite{bider2005strategy} defines business process in terms of goals and strategies in order to achieve the goals. It also uses map representation system that  conforms goals and strategies. In this approach the details regarding visibility of goals has not been addressed, hence requirement R1 is not satisfied. Also details about cost of achieving a goal through a strategy has also been not addressed, hence requirement R2 is also not satisfied. The requirement R3 contradicts with the process rule of this approach which states that "There is no goal/strategy in the map that can be considered as the subset
of another one". Requirement R4 is also not satisified due to the fact that this approach follows strategy driven modeling. Information about participation of different members has also not been metioned, hence requirement R5 is also not met. This approach addresses the requirement R6, as it supports reusability of the map components in different maps. 

\textit{Activity-Centric} : The activity-centric approach \cite{Yarosh2009} also supports knowledge workers by providing shared activity constructs as a computational unit for organizing the work. This approach provides team level view of past and ongoing work
and also supports propogation of completed activities to the existing activities. Hence requirement R1 is partially met. The information about cost of achieving a goal or activity has not been mentioned. Thus requirement R2 is not satisfied. In this approach, activities support objectives at various levels of granularity and thus requirement R4 is met. Since the main focus of this activity, it has
not provided any information regarding working style based on goals instead working style is based on activities. Hence requirement R5 is also met. Cross activity overview pattern is one of the pattern described in this approach which does unifying work across the team members, hence reequirement R5 is addressed. Since this approach also supports reusable activity patterns requirement R6 is also met. 
 
\textit{Activity-Oriented} :  The activity-oriented approach \cite{Reijers2006} is traditional workflow management approach where the main focus unit is business process' activity rather than strategy. In traditional workflows the concept of "process view" from different levels of an organisation not addressed, thus requirement R1 is not addressed. The details about cost calculation is not mentioned, hence requirement R2 is also not satisfied. Though this approach does not support sub-processes directly, it provides support for plugging in sub process extensions, this satisfies requirement R3. From goal oriented to activity oriented working style, so the working style is based on activities and not on goals. Thus requirement R4 is not addressed. Traditional workflow models like BPMN do not support participative modeling, but extensions like social-BPM supports social interactions. Hence requirement R5 is addressed. Re-using of existing activities is not addressed, thus requirement R6 is not addressed.  

\textit{Artifact-Centric} :  The artifact-centric approach \cite{Cohn2009} combines business data as artifacts and business process in a holistic way. This approach clearly states that artifacts "views" is not addressed, thus requirement R1 is not satisfied. The requirement R2 which is about cost calculation is not addressed. This approach allows modularity and componentization of business operations at various levels, hence requirement R3 is satisfied. Requirement R4  is partially met as the process evolves through a series of
intermediate goals. Requirement R5 is not met due to the fact that concept of social organizational modeling is not addressed. Requirement R6 is partially met because, only the concepts of modularization, componentization at various levels of abstraction has been discussed but reuse of components has not been addressed.

\textit{Capability-Driven} :  The capability driven approach \cite{Stirna2012} also proposes to support the changing environment of organizations. But in this approach there is no information about the visibility of goals has been addressed. Hence requirement R1 is not met. This approach claims that, it overcomes the challenge of high cost in developing applications but there is no clear details about how cost calculation is done, hence requirement R2 is partially addressed. In this approach, the top goal is refined into a number of sub-goals, then each sub-goal is lined to one or several KPIs. Thus requirement R3 is met. Since visibility of goals is not addressed, the details about explicit goals which is requirement R4 is not addressed. Requirement R5 is not addressed as the concept of multiple resources working together is not described. Reuse and execution of capability delivery pattern has been addressed, this meets requirement R6. 

\textit{ArchiMate} : This approach \cite{Aldea2015}, investigates if ArchiMate modeling language tool can be used to model strategies and also addresses the properties of Archimate. This apporach provides visibility of whole process, supports "viewpoints" in different levels of modeling. Thus requirement R1 is addressed. Requirement R2 which is cost of achieving each goal is not addressed. The approach provides three levels of modeling i.e., business, application, and technology, thus requirement R3 is partially met. This approach provides visibility of whole process which supports explicit goals but ArchiMate modeling language is not very easy to use to model multiple strategies with goal concept, thus requirement R4 is partially met. Requirement R5 is addressed as it supports different resources participation at different levels. Requirement R6 is also addressed as re-use of existing models is supported. 

\textit{Subject-Oriented} : This approach \cite{Fleischmann2013} supports multi-agent business process models that improves efficiency of the business logics. In this approach, requirement R1 is partially addressed  as S-BPM shows process view of who communicates with whom
but not how the process advances or if it terminates at all. Requirement R2 which is cost of achieving each goal is not addressed. Requirement R3 is addressed as the main process net can be divided into sub-nets. Requirement R4 is not addressed, as  the concept of explicit goals is not described. S-BPM emphasises "subjects" in a process as a decenrtalised, interacting entities thus requirement R4 is addressed. Requirement R6 which is about re-using of existing knowledge is not addressed. 

\begin{center}
	\begin{longtable}{p{6cm}p{1.5cm}p{1.5cm}p{1.5cm}p{1.5cm}p{1.5cm}p{2cm}} 
		\toprule 
		\textbf{Approach} & \textbf{R1}  & \textbf{R2}  & \textbf{R3}  & \textbf{R4}  & \textbf{R5} & \textbf{R6} \\
		\midrule
		\endfirsthead
		
		Strategy-Driven & No  & No  & No  & Partial  & No  & Yes\\
		Activity-Centric   & Partial   & No  & Yes  & Partial  & Yes  & Yes \\
		Activity-Oriented    & No  & No  & Yes  & Partial   & Partial  & No \\
		Artifact-Centric    & No  & No  & Yes  & Partial & No  & No \\ %R4 -partial , R5 -No
		Capability-Driven   & No  & No  & Yes  & No  & No  & Yes\\
		ArchiMate  & Yes  & No  & Partial  & Partial  & Yes  & Yes \\
		Subject-Oriented   &Partial  & No  & Yes  & No   & Yes   & No\\
		
		\bottomrule
		\caption{Evaluation of the Approach}
		\label{tab:evaluationoftheapproach}
	\end{longtable}	
\end{center}

Legend :
\begin{description}
	\item[Yes]      Requirement is addressed
	\item[No]       Requirement is not addressed
	\item[Partial]  Requirement is partially addressed
\end{description}

The approach \textit{Adaptive Case Management}, proposed by Hermann et. al \cite{Herrmann2011} bridges the gap between business processes management and flexibility in adapting knowledge intensive processes by defining activities and re-using created activity structure. When the required activities changes dynamically, capturing them for re-use are not helpful \cite{Sungur2015}. Though the approach \textit{Ad-hoc and Collaborative Processes} proposed by Dustdar et. al. overcomes the challenges in process aware collaborations, defining activities in a ad-hoc fashion does not support human actor in various cases \cite{Sungur2015}. Also the approach proposed in Chapter \ref{chap:approach} serves as a complementary to the above discussed  approaches.  This is because every existing approach satisfies one or few of the requirements even though not all requirements are satisfied. The Table \ref{tab:evaluationoftheapproach}, shows the number and extent of requirements satisfied by the existing approaches. 

 



\cleardoublepage
\chapter{An Approach to Resource-centric Organizational Modeling}
\label{chap:approach}

%%Description of the approach you have taken to solve the scientific or technical problem which you were posed.
%%Outline the design, the methodology and overall structure of your experinmental approach.

%%%%%%%%%%%%%%%%%%%%%%%%%%%%%%%%%%%%%%%%%%%%%%%%%%%%%%%%%%%%%%%%%%%%%%%%%
\section{Overview of Modeling Process}
\label{sec:overviewmodelingprocess}
%%%%%%%%%%%%%%%%%%%%%%%%%%%%%%%%%%%%%%%%%%%%%%%%%%%%%%%%%%%%%%%%%%%%%%%%%
 The Organizational Modeling element notation has been selected as per the guidelines mentioned in the paper by Moody \cite{Moody2009}. Also by observing  the fact that business process modelers are already well-known with the present process modeling notations such as Business Process Modeling Notation 2.0 (BPMN) \cite{bpm2011} and ArchiMate notation \cite{arc2013}, the shape depiction of organizational model elements are designed similar to those existing process notations. 




%%%%%%%%%%%%%%%%%%%%%%%%%%%%%%%%%%%%%%%%%%%%%%%%%%%%%%%%%%%%%%%%%%%%%%%%%
\section{Evaluation of the Approach}
\label{sec:evaluationoftheapproach}
%%%%%%%%%%%%%%%%%%%%%%%%%%%%%%%%%%%%%%%%%%%%%%%%%%%%%%%%%%%%%%%%%%%%%%%%%

The Table \ref{tab:evaluationoftheapproach}, provides an evaluation of the approaches. The description of each symbol used in the  Table \ref{tab:evaluationoftheapproach} is given as a legend.


\begin{center}
	\begin{longtable}{p{5cm}p{2cm}p{2cm}p{2cm}p{2cm}p{2cm}} 
		\toprule 
		\textbf{Approach} & \textbf{R1}  & \textbf{R2}  & \textbf{R3}  & \textbf{R3}  & \textbf{R5} \\
		\midrule
		\endfirsthead
		\\
	   	Strategy-Driven & -  & -  & +  & -  & + \\
	   	Activity-centric System  \\
	   	Activity-oriented System  \\
	   	Artifact-centric System  \\
	   	Capability-driven Development \\
	   	ArchiMate \\
	    Subject-Oriented System \\
		
		\bottomrule
		\caption{Evaluation of the Approach}
		\label{tab:evaluationoftheapproach}
	\end{longtable}	
\end{center}

Legend :
\begin{description}
	\item[+]  Addressed in the approach
	\item[-]  Not addressed in the approach
	\item[\~] Partially addressed in the approach
\end{description}

%%%%%%%%%%%%%%%%%%%%%%%%%%%%%%%%%%%%%%%%%%%%%%%%%%%%%%%%%%%%%%%%%%%%%%%%%
\section{Design Methodology}
\label{sec:designmethodology}
%%%%%%%%%%%%%%%%%%%%%%%%%%%%%%%%%%%%%%%%%%%%%%%%%%%%%%%%%%%%%%%%%%%%%%%%%

On the left list we should have all organizational context definitions and on the right one only ones that are contained in an informal process. The dropdown box of the initial and final context defines the selection inside of an informal process, thus right side.

So, in db.cljs, we should have only a list of context definitions no :initial-contexts and :desired-final-contexts. Only :organizational-contexts and under this all available contexts. Under the left list, we present these elements. Right list should refer to the initial-context, final-contexts, etc. of the informal process model depending on the selection of the dropbox button. For instance if we have initial contexts selected on the dropdown box, we should present the initial contexts in the right list.

I have changed the code accordingly and provided you an example how you should change data from views.cljs. All data should be stored in db.cljs. This applies to the text fields of all elements. Whenever, we want to update something we need to update the map in db.cljs and this will be propagated to the views.


On the left side of each list item, you should present all available items of context definitions or intentions whatever type is selected there. On the right side only the ones contained in the respective informal process model. Inside of another entity, you should refer to other entities using their ids and these ids should be resolved using, for instance, intentions vector. You check each intention in the intentions vector, if it’s id is the same as the id you are looking for it, you found it and you use the information about it.  


Please align it with the structure and names of the IPSM.xsd. Each variableName like this is written like variable-name. Each complex type is a map each attribute is a key value pair and each element in another element is another key value pair.


%%%%%%%%%%%%%%%%%%%%%%%%%%%%%%%%%%%%%%%%%%%%%%%%%%%%%%%%%%%%%%%%%%%%%%%%%
\subsection{Specifications}
\label{subsec:specifications}
%%%%%%%%%%%%%%%%%%%%%%%%%%%%%%%%%%%%%%%%%%%%%%%%%%%%%%%%%%%%%%%%%%%%%%%%%
In order to realize the web editor of Intention-centric Organizational Modeling, a formal inquiry has been done and concluded with the below specifications.

\begin{enumerate}   
	\item \textbf{Clojurescript} as the programming language
	\item \textbf{IntelliJIDEA} as the IDE
	\item \textbf{MVC} as the architecture pattern
	\item \textbf{Re-frame} as the pattern for writing SPAs in ClojureScript, using Reagent	
\end{enumerate}

%%%%%%%%%%%%%%%%%%%%%%%%%%%%%%%%%%%%%%%%%%%%%%%%%%%%%%%%%%%%%%%%%%%%%%%%%
\subsection{MVC Architecture}
\label{subsec:mvcarch}
%%%%%%%%%%%%%%%%%%%%%%%%%%%%%%%%%%%%%%%%%%%%%%%%%%%%%%%%%%%%%%%%%%%%%%%%%
 The architecture of the UI editor is based on the \textbf{Model-View-Control (MVC)} design pattern. The MVC paradigm allows to separate business logic from the code that controls presentation and event handling \cite{Oracle2016}.Each entity view in the web page is made up of combination of at least on Model and View, and one or more Controls. The individual files which acts an Model, View and Controller has been shown in the Figure \ref{fig:mvc_arch}

\begin{itemize}
	\item \textbf{Model} artifact stores the required data structure for web-editor. In the developed model artifact, the four main types of data stored inside the artifact are intentions, strategies, capabilities and informal process instances. 
	\item \textbf{View} artifact contains HTML elements and HTML constructs that describe the way of displaying the data from Model to the user.
	\item \textbf{Control} artifact contains the handler functions which can only change the model. Even the initial values of the model are put inside the control. 
\end{itemize}


\begin{figure}
	\centering
	\includegraphics [width= 0.75\textwidth]{mvc_arch.pdf}
	\caption{Relationship between developed web editor artifacts and MVC architecture components}
	\label{fig:mvc_arch}
\end{figure}


%%%%%%%%%%%%%%%%%%%%%%%%%%%%%%%%%%%%%%%%%%%%%%%%%%%%%%%%%%%%%%%%%%%%%%%%%
\subsubsection{Example: Component using MVC Pattern }
%%%%%%%%%%%%%%%%%%%%%%%%%%%%%%%%%%%%%%%%%%%%%%%%%%%%%%%%%%%%%%%%%%%%%%%%%
 The Figure \ref{fig:mvc_arch} below shows the simplifed version of how the components interact with each other using the Model-View-Control (MVC) pattern, for the functionality adding new entity data. This functionality is same for all the types intentions, strategies, capabilities and informal proceess instances and below is the detailed explanation of each interaction.

\begin{enumerate}
	\item User clicks the tab \textbf{Add New} in the web editor.
	\item View, in response to the user click displays the UI component for entering the new entity data details.
	\item User enters the required basic details for adding new entity data and clicks save button.
	\item View dispatches the data to Control, which can only modify the Model.
	\item Control inserts/updates data into the model.
	\item View displays the updated model as it has been subscribed to the model.
\end{enumerate}

\begin{figure}
	\centering
	\includegraphics [width= \textwidth]{mvc_pattern.pdf}
	\caption{MVC Pattern of adding new entity}
	\label{fig:mvc_pattern}
\end{figure}


%%%%%%%%%%%%%%%%%%%%%%%%%%%%%%%%%%%%%%%%%%%%%%%%%%%%%%%%%%%%%%%%%%%%%%%%%
\subsection{Using Reagent Framework}
\label{subsec:reagent}
%%%%%%%%%%%%%%%%%%%%%%%%%%%%%%%%%%%%%%%%%%%%%%%%%%%%%%%%%%%%%%%%%%%%%%%%%

The Reagent Framework architecture has been reused Fig. \ref{fig:mvc_pattern23} \footnote{Source: https://github.com/Day8/re-frame}


\begin{figure}
	\centering
	\includegraphics [width= \textwidth]{mvc_pattern.pdf}
	\caption{MVC Pattern of adding new entity }
	\label{fig:mvc_pattern23}
\end{figure}


%%%%%%%%%%%%%%%%%%%%%%%%%%%%%%%%%%%%%%%%%%%%%%%%%%%%%%%%%%%%%%%%%%%%%%%%%
\subsection{User Interface Diagram}
\label{sec:uidiagram}
%%%%%%%%%%%%%%%%%%%%%%%%%%%%%%%%%%%%%%%%%%%%%%%%%%%%%%%%%%%%%%%%%%%%%%%%%




%%%%%%%%%%%%%%%%%%%%%%%%%%%%%%%%%%%%%%%%%%%%%%%%%%%%%%%%%%%%%%%%%%%%%%%%%
\section{Relationship between Entity Types}
\label{sec:enttyperelation}
%%%%%%%%%%%%%%%%%%%%%%%%%%%%%%%%%%%%%%%%%%%%%%%%%%%%%%%%%%%%%%%%%%%%%%%%%

%%%%%%%%%%%%%%%%%%%%%%%%%%%%%%%%%%%%%%%%%%%%%%%%%%%%%%%%%%%%%%%%%%%%%%%%%
\subsection{Context Intention Relationship}
\label{sec:ctxintrel}
%%%%%%%%%%%%%%%%%%%%%%%%%%%%%%%%%%%%%%%%%%%%%%%%%%%%%%%%%%%%%%%%%%%%%%%%%
Intentions connect initial context definitions with final context definitions. \ref{fig:CtxIntRel}


\begin{figure}
	\centering
	\includegraphics [width= 0.5\textwidth]{CtxIntRel.pdf}
	\caption{Context Intentions Relationship}
	\label{fig:CtxIntRel}
\end{figure}

%%%%%%%%%%%%%%%%%%%%%%%%%%%%%%%%%%%%%%%%%%%%%%%%%%%%%%%%%%%%%%%%%%%%%%%%%
\subsection{Capabilities Intention Relationship}
\label{sec:capIntRel}
%%%%%%%%%%%%%%%%%%%%%%%%%%%%%%%%%%%%%%%%%%%%%%%%%%%%%%%%%%%%%%%%%%%%%%%%%

 Each organizational capability must be provided by a resource in the organization. Resource models are optional to make precise definitions of resources needed. The relationship between organizational capabilities and organizational intentions has been provided in the Figure 
 
 \begin{figure}
 	\centering
 	\includegraphics[width=\textwidth]{CapIntRel.pdf}
 	\caption{Relation between organizational capabilities and intentions}
 	\label{fig:orgcapabilities}
 \end{figure}



\cleardoublepage
\chapter{Case Study on Intention-oriented Organizational Modeling}
\label{chap:casestudy}

In this chapter, the first two sections provide implementation details along with the reason for making certain decisions regarding the implementation of web-based modeling tool. The third section provides an architecture of the functioning system and fourth section provides application flow of the functioning system. The fifth section explains how motivating scenario has been realized using the proposed modeling approach. Successful modeling of the motivating scenario using the developed editor serves as a proof for usability of the web-based modeling tool. Hence, the final section validates the system by evaluating it with the requirements for supporting intention-oriented organizational modeling. 

%%%%%%%%%%%%%%%%%%%%%%%%%%%%%%%%%%%%%%%%%%%%%%%%%%%%%%%%%%%%%%%%%%%%%%%%%
\section{Technologies and Frameworks}
\label{subsec:specifications}
%%%%%%%%%%%%%%%%%%%%%%%%%%%%%%%%%%%%%%%%%%%%%%%%%%%%%%%%%%%%%%%%%%%%%%%%%
In order to realize the web-based modeling tool of intention-oriented organizational modeling, a formal inquiry was done to choose suitable technologies and frameworks required. The below specifications were finalized and \textit{client-side scripting} \cite{Sierra2012} was chosen, due to the fact that developed tool is web-based single page application (SPA) \cite{Mikowski2013}. 

\begin{enumerate}   
	\item \textit{ClojureScript}\footnote{http://clojure.org/about/clojurescript} as the programming language
	\item \textit{Model-view-controller (MVC)} \cite{Deacon2009}  as the architecture pattern
	\item \textit{Re-frame}\footnote{https://github.com/Day8/re-frame} as the pattern for writing SPA in ClojureScript, using Reagent\footnote{http://reagent-project.github.io/}	
\end{enumerate}

Other than the above listed frameworks and technologies, frameworks like \textit{react-bootstrap}\footnote{https://react-bootstrap.github.io/}, jquery\footnote{https://jquery.com/} were also used to provide more optimal view of the tool. Along with this, we have also used libraries like bidi\footnote{https://github.com/juxt/bidi} and pushy\footnote{https://github.com/kibu-australia/pushy}, to handle page navigation from current location to the desired location in the URL (Uniform Resource Locator) of the browser. \textit{Clojure}https://clojure.org/ is a dynamic, general-purpose programming language, combining the approachability and interactive development of a scripting language. \textit{ClojureScript} is a compiler for Clojure that targets JavaScript which has been designed to emit JavaScript code. In our implementation, we have used both Clojure and Clojurescript artifacts. We also used \textit{Reagent} which provides minimalistic interface between ClojureScript and React\footnote{https://facebook.github.io/react/}. \textit{Re-frame}\footnote{https://github.com/Day8/re-frame} is a pattern for writing applications in ClojureScript, using Reagent.

%%%%%%%%%%%%%%%%%%%%%%%%%%%%%%%%%%%%%%%%%%%%%%%%%%%%%%%%%%%%%%%%%%%%%%%%%
\subsection{MVC Architecture}
\label{subsec:mvcarch}
%%%%%%%%%%%%%%%%%%%%%%%%%%%%%%%%%%%%%%%%%%%%%%%%%%%%%%%%%%%%%%%%%%%%%%%%%
The architecture of the developed user interface is based on the \textit{Model-View-Control (MVC)} design pattern. The MVC paradigm allows to separate business logic from the code that controls presentation of user interface and event handling \cite{Oracle2016}. Each entity view in the web page is made as a combination of at least one Model, View and one or more Controls. 

\textit{Model} artifact stores the required data structure for web-based modeling tool. In the developed model artifact, the data structure of modeling elements with their values are stored. 

\textit{View} artifact contains HTML (HyperText Markup Language) elements and HTML constructs that describe the way of displaying the data from Model to the user. Most of the common functionalities that render user interface components are re-used. 

\textit{Control} artifact contains the handler functions which can only change the model. Even the initial values of the model are put inside the control. This artifact has functions that updates default database, which then causes a re-render of view that makes the user to see a new view.

Apart from the above artifacts, there is another important artifact that registers subscription functions, i.e., query layer of the data. As view components never source data directly from default model, we use \textit{subscription} functions. Subscription functions returns values that change over time, i.e., based on user events.

%%%%%%%%%%%%%%%%%%%%%%%%%%%%%%%%%%%%%%%%%%%%%%%%%%%%%%%%%%%%%%%%%%%%%%%%%
\subsubsection{Example: Component using MVC Pattern }
%%%%%%%%%%%%%%%%%%%%%%%%%%%%%%%%%%%%%%%%%%%%%%%%%%%%%%%%%%%%%%%%%%%%%%%%%
The Figure \ref{fig:mvc_pattern} below shows how components interact with each other using the MVC pattern with a simple example of adding new modeling entity data. This functionality is same for all the types such as intentions, strategies, capabilities and informal processes.  

\begin{figure}
	\centering
	\includegraphics [width= \textwidth]{mvc_pattern.pdf}
	\caption{MVC Pattern of Adding New Entity}
	\label{fig:mvc_pattern}
\end{figure}

\begin{enumerate}
	\item User clicks the \textit{Add New} button in the developed editor.
	\item Responding to the user click, view displays the respective user interface component for entering the new entity data details.
	\item User enters the required basic details for adding new entity data and clicks save button.
	\item View dispatches data to control, as control can only modify the model.
	\item Control inserts/updates data into the model.
	\item View displays the updated model as it has been subscribed to the model.
\end{enumerate}

%%%%%%%%%%%%%%%%%%%%%%%%%%%%%%%%%%%%%%%%%%%%%%%%%%%%%%%%%%%%%%%%%%%%%%%%%
\section{Architecture of the Functioning System}
\label{sec:architectureofthefunctioningsystem}
%%%%%%%%%%%%%%%%%%%%%%%%%%%%%%%%%%%%%%%%%%%%%%%%%%%%%%%%%%%%%%%%%%%%%%%%%
Also from the Figure \ref{fig:architectureofthecasestudy}, it is clear that we followed the MVC architecture to design the user interface. Business experts can use the web-based modeling tool to view/update the descriptive entity details. Whenever a change in the model data is detected respective handler function is \textit{dispatched} and the corresponding handler function can only \textit{update} the model. Since we associate every modeling element with another modeling element, model data of an element is required by another element which are resolved using the unique reference identifier. For example, intention model's unique reference identifier of intention \textit{improve help customer help portal} is required by the strategy \textit{through application development}. This is because, for strategy (through application development), intention (improve help customer help portal) is the target intention. 

\begin{figure}
	\centering
	\includegraphics [width= \textwidth]{architectureofthecasestudy.pdf}
	\caption{Architecture of the Functioning System}
	\label{fig:architectureofthecasestudy}
\end{figure}


%%%%%%%%%%%%%%%%%%%%%%%%%%%%%%%%%%%%%%%%%%%%%%%%%%%%%%%%%%%%%%%%%%%%%%%%%
\subsection{Application Flow}
\label{subsec:applicationflow}
%%%%%%%%%%%%%%%%%%%%%%%%%%%%%%%%%%%%%%%%%%%%%%%%%%%%%%%%%%%%%%%%%%%%%%%%%
In this sub-section we provide an overview about how page navigation from current location to the desired location happens in URL of the browser. The external libraries used for route navigation, parses URL into data structures and generates URL from data structure defined as required routes. We call a function to dispatch route, with the matched route. Then we also have another function that parses the URL, to turn URL into data structure representing it. From the Figure \ref{fig:UIArchitecture}, it is clear that route navigation for each entity items happens based on their entity type and its own unique reference identifier.

\begin{figure}
	\centering
	\includegraphics [width= \textwidth]{UIArchitecture.pdf}
	\caption{Implementation of URL Navigation}
	\label{fig:UIArchitecture}
\end{figure} 

Each entity item has basic properties such as \textit{name} and \textit{target namespace}. The entities are identified using their unique id which is generated using the unique combination of name and target namespace. The entities that are associated with a particular entity are resolved through unique identifier. For example, in our motivating scenario consider the intention \textit{improve the customer help desk portal} when creating model for this intention, business expert provide name and namespace for this intention and add it to the database. A unique identifier is generated for the intention model using the combination of name and namespace by the system. For example, the strategy in the motivating scenario \textit{through application  development} that is associated with an intention, contains only the unique identifier of intention as reference. 

%%%%%%%%%%%%%%%%%%%%%%%%%%%%%%%%%%%%%%%%%%%%%%%%%%%%%%%%%%%%%%%%%%%%%%%%%
\section{Realization of Motivating Scenario}
\label{sec:realization}
%%%%%%%%%%%%%%%%%%%%%%%%%%%%%%%%%%%%%%%%%%%%%%%%%%%%%%%%%%%%%%%%%%%%%%%%%
The realization of motivating scenario is explained by integrating the concepts discussed in Chapter \ref{chap:motivatingScenario} and the informal process modeling approach discussed in the Chapter \ref{chap:approach}. From the Figure \ref{fig:realizationofmotivatingscenario}, it is clear that to realize the motivating scenario using the proposed approach it is important to model them step by step as mentioned in the informal process modeling approach. The developed modeling tool also supports dynamic changes in the models whenever there is a need to add new models. As each models are designed in individual modeling step, details of individual modeling steps are provided in the following sub sections. 

\begin{figure}
	\centering
	\includegraphics[width=\textwidth]{PhasesofMotivatingScenario.pdf}
	\caption{Realization of Motivating Scenario}
	\label{fig:realizationofmotivatingscenario}
\end{figure}

%%%%%%%%%%%%%%%%%%%%%%%%%%%%%%%%%%%%%%%%%%%%%%%%%%%%%%%%%%%%%%%%%%%%%%%%%
\subsection{Realization of Context Definitions}
%%%%%%%%%%%%%%%%%%%%%%%%%%%%%%%%%%%%%%%%%%%%%%%%%%%%%%%%%%%%%%%%%%%%%%%%%
In the informal process modeling approach, the first modeling step is to model the context definitions (M1). Each informal process starts from an initial context and aims to achieve an intention \cite{Sungur2014a}. After reaching an intention, there is resulting IPE Context. To realize the motivating scenario, user can add new contexts by providing basic properties such as name of the context and target namespace of the context as they serve as unique reference identifier for these contexts. After successfully adding the basic properties, user can provide entity specific properties such as contained contexts inside the main context, entity definition details about the contexts and participant list with respective privileges for each participant are also provided. The required context definitions are modeled first because these definition are required for modeling intention definitions and process definitions.  

%%%%%%%%%%%%%%%%%%%%%%%%%%%%%%%%%%%%%%%%%%%%%%%%%%%%%%%%%%%%%%%%%%%%%%%%%
\subsection{Realization of Intention Definitions}
%%%%%%%%%%%%%%%%%%%%%%%%%%%%%%%%%%%%%%%%%%%%%%%%%%%%%%%%%%%%%%%%%%%%%%%%%
After modeling context definitions(M1), the second step of the modeling is to model the intentions (M2). For example, in our motivating scenario we have main intention as "increase revenue and number of unit sales" and other low level intentions that emerged out of main intention and strategies of the main intention. The user can provide descriptive information about particular intention as intention definition. Similar to context modeling, the user has to provide basic properties such as name and target namespace required for unique identification of the entity. After providing basic properties, the user has to provide entity specific details of the intention such as due date and time for intention completion, priority of the intention, cost of the intention, other related intentions that are contained under this particular intention. The strategies to achieve this intention and contexts of the intention are also provided as entity specific properties. The participant list with respective privileges for each participant are also provided when an entity is of type interactive acquirable entity. 

%%%%%%%%%%%%%%%%%%%%%%%%%%%%%%%%%%%%%%%%%%%%%%%%%%%%%%%%%%%%%%%%%%%%%%%%%
\subsection{Realization of Strategy Definitions}
%%%%%%%%%%%%%%%%%%%%%%%%%%%%%%%%%%%%%%%%%%%%%%%%%%%%%%%%%%%%%%%%%%%%%%%%%
After modeling context definitions (M1) and intention definitions (M2) user can proceed to model the strategies (M3) which is third step of the modeling process. For example, in our motivating scenario user can model the strategies such as \textit{through expansion}, \textit{through advertisements} and other required strategies as third step of the modeling process. Similar to earlier modeling steps, during modeling of strategy user required to provide basic properties such as name and target namespace. After providing the basic properties, entity specific properties such as target intentions of the strategy, capabilities and process definitions associated with strategy are also provided. Since, strategy is also an interactive acquirable entity similar to intention, participant list details are also provided during modeling of strategies

%%%%%%%%%%%%%%%%%%%%%%%%%%%%%%%%%%%%%%%%%%%%%%%%%%%%%%%%%%%%%%%%%%%%%%%%%
\subsection{Realization of Capability Definitions}
%%%%%%%%%%%%%%%%%%%%%%%%%%%%%%%%%%%%%%%%%%%%%%%%%%%%%%%%%%%%%%%%%%%%%%%%%
Modeling of capability (M4) is the fourth step in intention-oriented organizational modeling. There are two types of capabilities. Functional capabilities and cross-functional capabilities. Functional capabilities are the capabilites that associated with other entity types. Cross-functional capabilities contains multiple functional capabilities. Similar to earlier entity types' basic properties such as name and target namespace are added to get the unique reference identifier and entity specific properties for capabilities are added. Since cross functional capability contains functional capabilities, it holds the identifiers of the functional capabilities contained in it. Functional capability definitions also has participant list details similar to intention definitions and strategy definitions. 

%%%%%%%%%%%%%%%%%%%%%%%%%%%%%%%%%%%%%%%%%%%%%%%%%%%%%%%%%%%%%%%%%%%%%%%%%
\subsection{Realization of Process Definitions}
%%%%%%%%%%%%%%%%%%%%%%%%%%%%%%%%%%%%%%%%%%%%%%%%%%%%%%%%%%%%%%%%%%%%%%%%%
By modeling the business processes based on resources that work towards certain intentions, informal processes are modeled without predefining their business logic \cite{Sungur2014a}. Also as mentioned earlier each informal process starts from an initial context and aims to achieve an intention that results in a final context. Thus, we require context definitions and intention definitions before modeling process definitions. Similar to earlier modeling of entity types, process modeling also requires basic properties such as name, namespace  and entity specific properties such as associated intentions, contexts and resources. Process definition also has participant list similar to other entity types. 

%%%%%%%%%%%%%%%%%%%%%%%%%%%%%%%%%%%%%%%%%%%%%%%%%%%%%%%%%%%%%%%%%%%%%%%%%
\subsection{Realization of Resource Definitions}
%%%%%%%%%%%%%%%%%%%%%%%%%%%%%%%%%%%%%%%%%%%%%%%%%%%%%%%%%%%%%%%%%%%%%%%%%
Each resource that provides certain capability can be related to another resource which are defined using predefined or custom \textit{relationships} \cite{Sungur2014a}. These resources are managed through \textit{resource organizers}, this is because resource organizers are used to bring together the relevant interrelated resources that work towards to achieve an intention. TOSCA \cite{Binz2014} can be used to model all nodes and relationship among them. In this work, authors consider resources as nodes to make use of the TOSCA's service. In the developed modeling tool, the resource models are managed by embedding the open source modeling tool Winery web page \cite{Kopp2013} in the modeling tool's web page. This is because it creates a new service template that contains an application topology by using the topology modeler. Winery also offers all available node types in a palette. From there, user can drag the desired node type and drop it into the editing area. There the node type becomes a node template i.e., a node in the topology graph. Node templates can be annotated with requirements and capabilities, property values, and policies. The screen shot of modeling sample resource has been provided in the Figure \ref{fig:realizationofresourcemodel}. 

In order to achieve this, we use TOSCA repository URL referring to winery and topology modeler of the winery. Using these values we create corresponding URL required for our modeling based on the name and namespace properties of an entity. The functionality to generate resource model page, using TOSCA repository URL and topology modeler URL is provided below.

\fbox{
	\begin{minipage}{\textwidth}
		\{topology-modeler-url\}?repositoryURL=\{encoded-tosca-repository-url\}\&ns=\{encoded-target-namepsace\}\&id=\{encoded-identifier\}\#
	\end{minipage}
	}
			
			
\begin{figure}
	\centering
	\includegraphics[width=\textwidth,angle=0]{ResourceModel_UI.png}
	\caption{User Interface Screen of Resource Model}
	\label{fig:realizationofresourcemodel}
\end{figure}
		
%%%%%%%%%%%%%%%%%%%%%%%%%%%%%%%%%%%%%%%%%%%%%%%%%%%%%%%%%%%%%%%%%%%%%%%%%
\subsection{Realization of Instance creation}
%%%%%%%%%%%%%%%%%%%%%%%%%%%%%%%%%%%%%%%%%%%%%%%%%%%%%%%%%%%%%%%%%%%%%%%%%
Initializing resource-centric processes requires acquiring and engaging interrelated resources \cite{Sungur2015}. As mentioned earlier, the phases of compiling and initializing of informal process models are not within scope of this work. Only the functionalities such as creating instances, extracting instances and editing instances are part of the functioning tool. This is because initializing informal process models starts after the initial context defined in an IPE model \cite{Sungur2015}. Thus, it is important to discuss realization of instance creation which are required for phase P3 of Executing Informal Processes (InProXec) method. Acquirable entity types' models can be converted into instances. For example, process definition is converted into \textit{process instance} when the model is compiled and engaged with resources. A model instance contains additional meta-data about the executed processes such as the information about the start date and time, end date and time, instance status, cost, source model etc. From the screen-shot image \ref{fig:realizationofinstances2} it is clear that these properties of an instance can be edited through the developed tool. Only when a acquirable model is successfully initialized it can be engaged to adapt the process execution of emerging requirements \cite{Sungur2015}. 

\begin{figure}
	\centering
	\includegraphics[width=\textwidth,angle=0]{InstanceDescriptor_UI.png}
	\caption{User Interface Screen of Instance Descriptor}
	\label{fig:realizationofinstances2}
\end{figure}

The developed tool supports creation and updation of descriptive information about instances. Each instance belong to any one of the acquirable entity type such strategies, intentions and informal processes. Any entity that has instances are also listed inside the \textit{Instance data} tab of each entity. From the user interface screen Figure \ref{fig:realizationofinstances}, it is clear that the editor has ability to add, remove and extract instance descriptors for any entity type. An instance descriptor of a functional capability refers to a resource definition meaning that a capability is provided by a resource.
 
\begin{figure}
	\centering
	\includegraphics[width=\textwidth,angle=0]{AcquirableEntities_UI.png}
	\caption{User Interface Screen of Acquirable Entities}
	\label{fig:realizationofinstances}
\end{figure}

		
%%%%%%%%%%%%%%%%%%%%%%%%%%%%%%%%%%%%%%%%%%%%%%%%%%%%%%%%%%%%%%%%%%%%%%%%%
\section{Validation of Requirements}
\label{sec:validation}
%%%%%%%%%%%%%%%%%%%%%%%%%%%%%%%%%%%%%%%%%%%%%%%%%%%%%%%%%%%%%%%%%%%%%%%%%	
This section evaluates the current functioning tool against the derived requirements of intention-oriented organizational modeling. 

\textit{Organizational Intention Transparency} (R1):  Using the modeling tool intentions at different levels can be modeled which satisfies first pre-requisite of R1. With the current functioning system any user can view intention at different levels which satisfies second pre-requisite of R2. Thus, requirement R1 is satisfied by the functioning system as both of its pre-requisites are satisfied .

\textit{Organizational Strategy-based Cost Estimation} (R2): The modeling tool itself calculates and displays the strategy cost calculation based on strategy implementation and resource cost. This cost information helps the business experts to make certain decision based on cost calculation during modeling. The functioning tool satisfies requirement R2 as both of its pre-requisites are met by the tool. 

\textit{Organizational Strategy Achievability Estimation} (R3): Similar to cost calculation, strategy achievability estimation based on its association with valid capability is also determined and displayed during modeling phase itself. The functioning tool satisfies requirement R3 as both of its pre-requisites are met by the tool.

\textit{Intention Oriented Working Style} (R4): Any user can create intention models, strategy models, informal process models etc., through the developed tool provided the user has understanding about main intention and its recursive structure. The requirement R4 is also satisfied by the functioning tool as it satisfies both the pre-requisites of R4.

\textit{Participative Organizational Modeling} (R5): Each entity type that can be interactively acquirable has list of participants with their corresponding privileges. The users can edit, view, own or follow based on their privilege. Thus, the tool satisfies the requirement R5.


\cleardoublepage
\chapter{Conclusion and Future Work}
\label{chap:conclusion}

In this document, we first started Chapter \ref{chap:introduction} with motivational and problem statement followed by research objectives of this work. In Chapter \ref{chap:fundamentals}, the fundamental concepts from existing literature has been provided in a detailed way. In Chapter \ref{chap:motivatingScenario}, a motivating scenario has been taken and explained based on the guidelines and real life scenarios discussed in some previous work. In Chapter \ref{chap:analysis}, a detailed requirements analysis and literature review from the existing approaches has been provided. This is followed by Chapter \ref{chap:approach}, which provides detailed description about the methodology and characteristics of the modeling process. A detailed case study has been provided in Chapter \ref{chap:casestudy}, which helps to explain the abstract concepts discussed in the earlier chapter in a concrete way. This chapter also validates the developed web–based editor by providing examples that satisfies the research objectives discussed in Chapter \ref{chap:introduction} and also conformance of the motivating scenario discussed in Chapter \ref{chap:motivatingScenario} with the developed system.

 \begin{figure}
 	\centering
 	\includegraphics [width= \textwidth]{conclusionimage.pdf}
 	\caption{Contribution of the thesis work}
 	\label{fig:conclusionimage}
 \end{figure}
 
The place of this thesis work in the execution steps of informal process is shown in the Figure \ref{fig:conclusionimage}. The developed editor serves the purpose of creating descriptive models for informal process, intentions, strategies, capabilities, resources and contexts. This created models are used by the next phases of compilation (P3) and execution (P4). This work has not provided details of formal definitions like which execution steps has to be taken by which resources. This work of resource-centric informal process modeling provides complementary \textit{informal} guides and definitions of intentions, strategies, capabilities and resources of a process. 

%%%%%%%%%%%%%%%%%%%%%%%%%%%%%%%%%%%%%%%%%%%%%%%%%%%%%%%%%%%%%%%%%%%%%%%%%
\section*{Future Work}
\label{sec:futurework}
%%%%%%%%%%%%%%%%%%%%%%%%%%%%%%%%%%%%%%%%%%%%%%%%%%%%%%%%%%%%%%%%%%%%%%%%%
Each resources can be related with other resources through \textit{relationships}. This helps business experts to create models with logical resource structures. In this thesis work, we have addressed resource models without relationships and left the ones contain relationships as future work. This is due to the fact that relationships are optional entities in each model and also due to the broad context of this work \cite{Sungur2014a}. 

As discussed in Chapter \ref{chap:introduction}, the web based editor developed as part of this master thesis work, will be further extended such that it can generate deployable entities from the current descriptive information. These deployable entities will be further developed as compilable and executable entities in phases P3 and P4 of the InProXec\cite{Sungur2015}. Also extension of providing mobile support to this web editor are also part of future work. 






%
%
%\renewcommand{\appendixtocname}{Anhang}
%\renewcommand{\appendixname}{Anhang}
%\renewcommand{\appendixpagename}{Anhang}
\appendix
%\input{content/latex-tipps}

%\printindex

\printbibliography

\ifdeutsch
All links were last followed on \today.
\else
All links were last followed on \today.
\fi

\pagestyle{empty}
\renewcommand*{\chapterpagestyle}{empty}
%\Versicherung
\pagestyle{empty}
\vspace{9cm}
\begin{center}
	\begin{minipage}{11cm}
		\vspace{6cm}
		
		\textbf{\Large Declaration}\\\\
		\vspace{0.4cm}
		
		I hereby declare that the work presented in this thesis is entirely my own. 
		I did not use any other sources and references than the listed ones. I have marked all direct or indirect statements from other sources contained therein as quotations. 
		Neither this work nor significant parts of it were part of another examination procedure. I have not published this work in whole or in part before. 
		The electronic copy is consistent with all submitted copies.
		\vspace{1cm}
		
		
		
		
\hspace{2.1cm}-----------------------------------------------------------\\
\vspace{1cm} \hspace{2.1cm} place,date,signature
	\end{minipage}
\end{center}

\end{document}
