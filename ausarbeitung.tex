% !TeX spellcheck = de_DE
% Dieses Dokument muss mit PDFLatex gesetzt werden
% Vorteil: Grafiken koennen als jpg, png, ... verwendet werden
%          und die Links im Dokument sind auch gleich richtig
%
%Ermöglicht \\ bei der Titelseite (z.B. bei supervisor)
%Siehe https://github.com/latextemplates/uni-stuttgart-cs-cover/issues/4
\RequirePackage{kvoptions-patch}
%Warns about outdated packages and missing caption delcarations
%See https://www.ctan.org/pkg/nag
\RequirePackage[l2tabu, orthodox]{nag}
%Neue deutsche Trennmuster
%Siehe http://www.ctan.org/pkg/dehyph-exptl und http://projekte.dante.de/Trennmuster/WebHome
%Nur für pdflatex, nicht für lualatex
%\RequirePackage[ngerman=ngerman-x-latest]{hyphsubst}
\documentclass[
               fontsize=12pt, %Default: 11pt, bei Linux Libertine zu klein zum Lesen
               paper=a4,
               twoside, % fuer die Betrachtung am Schirm ungeschickt
% BEGINN: Optionen für typearea
               BCOR=3mm, % Hack für BCOR (1.92 o.ä.), da bei BCOR2mm die Fuellpunkte beim Inhaltsverzeichnis falsch sind. Hack aber nicht mehr nötig: microtype für Verzeichnisse ausschalten hilft.
               DIV=13,   % je höher der DIV-Wert, desto mehr geht auf eine Seite. Gute werde sind zwischen DIV=12 und DIV=15
               headinclude=true,
               footinclude=false,
% ENDE: Optionen für typearea
%               titlepage,
               bibliography=totoc,
%               idxtotoc,   %Index ins Inhaltsverzeichnis
%                liststotoc, %List of X ins Inhaltsverzeichnis, mit liststotocnumbered werden die Abbildungsverzeichnisse nummeriert
               headsepline,
               cleardoublepage=empty,
               parskip=half,
               pointlessnumbers, %f"ur englische Texte, dann unten \ifdeutsch und \ifenglisch anpassen.
%               draft    % um zu sehen, wo noch nachgebessert werden muss - wichtig, da Bindungskorrektur mit drin
               final   % ACHTUNG! - in pagestyle.tex noch Seitenstil anpassen
               ]{scrbook}

%Englisch:
\let\ifdeutsch\iffalse
\let\ifenglisch\iftrue

%Deutsch:
%\let\ifdeutsch\iftrue
%\let\ifenglisch\iffalse


\input{preambel/packages_and_options}

%Der untere Rand darf "flattern"
\raggedbottom

%%%
% Wie tief wird das Inhaltsverzeichnis aufgeschlüsselt
% 0 --\chapter
% 1 --\section % fuer kuerzeres Inhaltsverzeichnis verwenden - oder minitoc benutzen
% 2 --\subsection
% 3 --\subsubsection
% 4 --\paragraph
\setcounter{tocdepth}{1}
%
%%%

\makeindex

%Angaben in die PDF-Infos uebernehmen
\makeatletter
\hypersetup{
            pdftitle={}, %Titel der Arbeit
            pdfauthor={}, %Author
            pdfkeywords={}, % CR-Klassifikation und ggf. weitere Stichworte
            pdfsubject={}
}
\makeatother

\begin{document}
%%%%%%%%%%%%%%%%%%%%%%%%%%%%%%%%%%%%%%%%%%%
%Erzeugung des Titelblatts
%%%%%%%%%%%%%%%%%%%%%%%%%%%%%%%%%%%%%%%%%%%
% Deckblatt zentrieren
%\newlength\oddsidemarginorig
%\oddsidemarginorig=\oddsidemargin
%\oddsidemargin 1.05cm
%\newgeometry{a4paper,left=20.05mm,right=10mm, top=29.7mm, bottom=29.7mm}
%\thispagestyle{plain}
\pagestyle{plain}
\begin{titlepage}
	\begin{sffamily}
		\begin{center}
			Institute of Architecture of Application Systems\\
			University of Stuttgart\\
			Universitätsstraße 38\\
			D-70569 Stuttgart\\
		\end{center}
		
		\vspace{3.5cm}
		
		\begin{center}
			{Master's Thesis No. }\\
			\vspace{0.5cm}
			\begin{minipage}{8.5cm}
				\begin{center}
					
					\Large \textbf{Intention-centric Modeling of Organizations}
					
				\end{center}
			\end{minipage}
			\\
			\vspace{1cm}
			{Archana Kalidoss}
		\end{center}
		
		\vspace{1.0cm}
		
		\begin{center}
			\begin{minipage}{3cm}
				\begin{center}
					\includegraphics[width=0.9\textwidth]{./gfx/unilogo.pdf}
					%\includegraphics{./gfx/uni_logo.png}
				\end{center}
			\end{minipage}
			\begin{minipage}{3cm}
				\begin{center}
					\includegraphics{./gfx/iaas.jpg}
				\end{center}
			\end{minipage}
		\end{center}
		%
		\vspace{1.0cm}
		%
		\begin{center}
			\begin{tabular}{ll}
				\textbf{Course of Study:} & Computer Science M.Sc\\
				&\\&\\
				\textbf{Examiner:}   & Prof. Dr. Dr. h. c. Frank Leymann\\
				\textbf{Supervisor:}   & M.Sc. C. Timurhan Sungur\\
				
				\textbf{Commenced:} & 2nd November 2015\\
				\textbf{Completed:}  & \\
				&\\
				\textbf{CR-Classification:} & \\
				
			\end{tabular}
		\end{center}
	\end{sffamily}
\end{titlepage}
%\oddsidemargin=\oddsidemarginorig
%%%%%%%%%%%%%%%%%%%%%%%%%%%%%%%%%%%%%%%%%%%
%Ende Titelblatt
%%%%%%%%%%%%%%%%%%%%%%%%%%%%%%%%%%%%%%%%%%%
	
\cleardoublepage
	
%tex4ht-Konvertierung verschönern
\iftex4ht
% tell tex4ht to create picures also for formulas starting with '$'
% WARNING: a tex4ht run now takes forever!
\Configure{$}{\PicMath}{\EndPicMath}{} 
%$ % <- syntax highlighting fix for emacs
\Css{body {text-align:justify;}}

%conversion of .pdf to .png
\Configure{graphics*}  
         {pdf}  
         {\Needs{"convert \csname Gin@base\endcsname.pdf  
                               \csname Gin@base\endcsname.png"}%  
          \Picture[pict]{\csname Gin@base\endcsname.png}%  
         }  
\fi

%Tipp von http://goemonx.blogspot.de/2012/01/pdflatex-ligaturen-und-copynpaste.html
%siehe auch http://tex.stackexchange.com/questions/4397/make-ligatures-in-linux-libertine-copyable-and-searchable
%
%ONLY WORKS ON MiKTeX
%On other systems, download glyphtounicode.tex from http://pdftex.sarovar.org/misc/
%
\input glyphtounicode.tex
\pdfgentounicode=1

\VerbatimFootnotes %verbatim text in Fußnoten erlauben. Geht normalerweise nicht.

\input{macros/commands}
\pagenumbering{arabic}
%\Titelblatt

%Eigener Seitenstil fuer die Kurzfassung und das Inhaltsverzeichnis
\deftripstyle{preamble}{}{}{}{}{}{\pagemark}
%Doku zu deftripstyle: scrguide.pdf
\pagestyle{preamble}
\renewcommand*{\chapterpagestyle}{preamble}

\setlength{\parindent}{0.0em}


%Kurzfassung / abstract
%auch im Stil vom Inhaltsverzeichnis
%\ifdeutsch
%\section*{Abstract}
%\else
\section*{Abstract}
%\fi
\hspace{4ex}Every organization thrives to achieve its intentions, these intentions can be in level of the organization like technical intentions that focus to satisfy technical level requirements, management intentions that focus to satisfy the management level requirements, and financial intentions to achieve financial level requirements. Intentions play critical role in many organizations because they motivate organizations towards the overall development. Therefore supporting and automating organizational intentions and associated components are absolute necessary for any organization. Current technologies and literature focus on diverse components  like activity, strategy, artifact and capability but no or little focus on \textbf{Intention}, which is the starting gate to reach the trailing gates like activity, strategy and capability.

\hspace{4ex}This Master thesis aims at providing means to design and realize the Intention-centric organizational modeling. We propose a motivating scenario to help the reader in easily acquiring the concepts and usability of developed web editor. The purpose of the web editor is to view/update existing intentions, strategies, capabilities and informal process instances and to add new data of type intentions, strategies, capabilities and informal process instances. 

\textbf{Key words:} Intentions, Capabilities, Strategies, Informal Process Instances and Informal Process Models

% BEGIN: Verzeichnisse

\iftex4ht
\else
\microtypesetup{protrusion=false}
\fi

%%%
% Literaturverzeichnis ins TOC mit aufnehmen, aber nur wenn nichts anderes mehr hilft!
% \addcontentsline{toc}{chapter}{Literaturverzeichnis}
%
% oder zB
%\addcontentsline{toc}{section}{Abkürzungsverzeichnis}
%\section*{Abkürzungsverzeichnis}
%
%%%

%Produce table of contents
%
%In case you have trouble with headings reaching into the page numbers, enable the following three lines.
%Hint by http://golatex.de/inhaltsverzeichnis-schreibt-ueber-rand-t3106.html
%
%\makeatletter
%\renewcommand{\@pnumwidth}{2em}
%\makeatother
%
\tableofcontents

% Bei einem ungünstigen Seitenumbruch im Inhaltsverzeichnis, kann dieser mit
% \addtocontents{toc}{\protect\newpage}
% an der passenden Stelle im Fließtext erzwungen werden.

%listof* untereinandergesetzt
%ACHTUNG! Falls ein anderer Kapitelstil gewählt wird, muss der Code hier evtl.
%  angepasst werden
\begingroup 
\makeatletter
  \def\@makeschapterhead#1{%
  \vspace*{10\p@}%
  {\parindent \z@ \raggedright \reset@font
            \normalfont \vphantom{\@chapapp{} \thechapter}
        \par\nobreak\vspace*{10\p@}%
        \interlinepenalty\@M
    {\huge \bfseries %
    %
    %Default-Schrift: Serifenhaft (fuer englische Dokumente)
    % Dann sowohl A als auch B deaktivieren
    %A) Fuer serifenlose Schrift folgende Zeile aktivieren:
    \ifdeutsch
    \fontfamily{phv}\selectfont
    \fi
    %B) Fuer Kapitaelchen folgende Zeile aktivieren:
    %\fontseries{m}\fontshape{sc}\selectfont
    %
    #1\par\nobreak}
    %\vspace*{1\p@}%
\makebox[\textwidth]{\hrulefill}%    \hrulefill alone does not work
    \par\nobreak
    \vskip 5\p@
  }}
\makeatother
\let\cleardoublepage\clearpage
\listoffigures
\newpage
%\let\cleardoublepage\relax
\listoftables
\newpage
%Wird nur bei Verwendung von der lstlisting-Umgebung mit dem "caption"-Parameter benoetigt
%\lstlistoflistings 
%ansonsten:
\ifdeutsch
\listof{Listing}{List of Listings}
\else
\listof{Listing}{List of Listings}
\fi

%mittels \newfloat wurde die Algorithmus-Gleitumgebung definiert.
%Mit folgendem Befehl werden alle floats dieses Typs ausgegeben
%\ifdeutsch
%\listof{Algorithmus}{List of Algorithms}
%\else
%\listof{Algorithmus}{List of Algorithms}
%\fi
%\listofalgorithms %Ist nur für Algorithmen, die mittels \begin{algorithm} umschlossen werden, nötig

\endgroup

\cleardoublepage

\iftex4ht
\else
%Optischen Randausgleich und Grauwertkorrektur wieder aktivieren
\microtypesetup{protrusion=true}
\fi

% END: Verzeichnisse


\renewcommand*{\chapterpagestyle}{scrplain}
\pagestyle{scrheadings}
\input{preambel/pagestyle}
%
%
% ** Hier wird der Text eingebunden **
%
\cleardoublepage
\chapter{Introduction}
\label{chap:introduction}
This work is structured in the following way:


\section*{Outline}
The remainder of the document is organized into following chapters
\begin{description}
\item[Chapter ~\ref{chap:motivatingScenario} -- \nameref{chap:motivatingScenario}:] In this chapter a motivating scenario has been taken which helps reader in realizing the Organizational Modeling throughout the document. 
\item[Chapter ~\ref{chap:fundamentals} -- \nameref{chap:fundamentals}:] In this chapter previous work carried in organizations to model the goals has been discussed.
\item[Chapter ~\ref{chap:requirements} -- \nameref{chap:requirements}:] Detailed requirements based on scientific facts and introduction about some properties of the organizations.
\item[Chapter ~\ref{chap:orgModeling} -- \nameref{chap:orgModeling}:] In this chapter basics of Organizational Modeling has been discussed and notations used to realize the Organization Modeling has been discussed.
\item[Chapter ~\ref{chap:design} -- \nameref{chap:design}:] In this chapter, proposed design is discussed. 
 
\item[Chapter ~\ref{chap:realization} -- \nameref{chap:realization}:] This chapter concentrates on detailed system architecture and also presents the experimental results.
\item[Chapter ~\ref{chap:evaluation} -- \nameref{chap:evaluation}:] Comparison of the Intention-centric Modeling of Organizations approach with the other approaches.
\item[Chapter ~\ref{chap:conclusion} -- \nameref{chap:conclusion}:] Summaries the results of this thesis work and draws conclusion. 
\end{description}


%%%%%%%%%%%%%%%%%%%%%%%%%%%%%%%%%%%%%%%%%%%%%%%%%%%%%%%%%%%%%%%%%%%%%%%%%
\section{Motivation}
%%%%%%%%%%%%%%%%%%%%%%%%%%%%%%%%%%%%%%%%%%%%%%%%%%%%%%%%%%%%%%%%%%%%%%%%%


%%%%%%%%%%%%%%%%%%%%%%%%%%%%%%%%%%%%%%%%%%%%%%%%%%%%%%%%%%%%%%%%%%%%%%%%%
\section{Problem Definition}
%%%%%%%%%%%%%%%%%%%%%%%%%%%%%%%%%%%%%%%%%%%%%%%%%%%%%%%%%%%%%%%%%%%%%%%%%

%%%%%%%%%%%%%%%%%%%%%%%%%%%%%%%%%%%%%%%%%%%%%%%%%%%%%%%%%%%%%%%%%%%%%%%%%
\section{Contributions and Outline}
%%%%%%%%%%%%%%%%%%%%%%%%%%%%%%%%%%%%%%%%%%%%%%%%%%%%%%%%%%%%%%%%%%%%%%%%%

%%%%%%%%%%%%%%%%%%%%%%%%%%%%%%%%%%%%%%%%%%%%%%%%%%%%%%%%%%%%%%%%%%%%%%%%%
\section {Research Objectives}
%%%%%%%%%%%%%%%%%%%%%%%%%%%%%%%%%%%%%%%%%%%%%%%%%%%%%%%%%%%%%%%%%%%%%%%%%
\label{sec:researchobj}

\begin{tabular}{p{5cm}p{11cm}} 
	\textbf{Research Objective} & \textbf{Description} \\
	\\
	RO. 1 & \textit{Intentions are traceble in the different levels of the organizational hierarchy. } \label{ro1} \\
	\\[-1.5ex]
	RO. 2 & \textit{Linking intentions with capabilities and at the with resources enable us a cost estimation for each intention. Cost is estimated in a recursive manner.} \label{ro2} \\
	\\[-1.5ex]
	RO. 3 & \textit{Validity of an organizational intentions is achieveable when the intentions can be refined by defining sub-intentions, which can then be defined recursively as independent informal processes.} \label{ro3}\\
	\\[-1.5ex]
	RO. 4 & \textit{As each member of the organization aware of the higher level and lower level intentions. He can engage for these explicit intentions. } \label{ro4}\\
	\\[-1.5ex]
	RO. 5 & \textit{Different members of an organization participate to create organizational intentions, as a result intentions are shaped based on all members but directed by the executives.} \label{ro5}\\
	\\[-1.5ex]
	RO. 6 & \textit{Intention-specific solutions can be extracted as abstract re-usable entities, organizational strategy patterns and can be re-used in muliple context definitions.} \label{ro6}\\
\end{tabular}

\cleardoublepage
\chapter{Motivating Scenario}
\label{chap:motivatingScenario}

In order to realize the Organizational Modeling  below motivating scenario has been taken and realized using the developed UI editor.
This scenario also helps in testing the UI editor along with realizing the Organizational Notations. The motivating scenario has been chosen based on the advice provided in works ..... The figure of this example was taken from the context of manufacturing sector. The abstract intention of the organization is to increase the quarterly revenue and number of unit sales. In order to achieve this intention through Organizational Modeling Approach, as a first step we need to break the abstract intention into several strategies like : 1. Increasing the revenue through expanding the market sales. 2. Through improving the excellence of the product which in turn brings back old and new customers. and 3. Through increasing the advertisement which helps in customer knowing about the product.  
\cleardoublepage
\chapter{Fundamentals of Organizational Modeling}
\label{chap:fundamentals}
\cleardoublepage
\input{content/Requirements}
\cleardoublepage
\chapter{Intention-centric Organizational Modeling}
\label{chap:orgModeling}

%%%%%%%%%%%%%%%%%%%%%%%%%%%%%%%%%%%%%%%%%%%%%%%%%%%%%%%%%%%%%%%%%%%%%%%%%
\section{Overview of Modeling Process}
\label{sec:overviewmodelingprocess}
%%%%%%%%%%%%%%%%%%%%%%%%%%%%%%%%%%%%%%%%%%%%%%%%%%%%%%%%%%%%%%%%%%%%%%%%%
\hspace{4ex} The Organizational Modeling element notation has been selected as per the guidelines mentioned in the paper by Moody \cite{Moody2009}. Also by observing  the fact that business process modelers are already well-known with the present process modeling notations such as Business Process Modeling Notation 2.0 (BPMN) \cite{bpm2011} and ArchiMate notation \cite{arc2013}, the shape depiction of organizational model elements are designed similar to those existing process notations. 





%%%%%%%%%%%%%%%%%%%%%%%%%%%%%%%%%%%%%%%%%%%%%%%%%%%%%%%%%%%%%%%%%%%%%%%%%
\section{Organizational Modeling Notation}
\label{sec:orgmodelingnotation}
%%%%%%%%%%%%%%%%%%%%%%%%%%%%%%%%%%%%%%%%%%%%%%%%%%%%%%%%%%%%%%%%%%%%%%%%%
\hspace{4ex} Due to the importance of shapes in expressing the information visually , the notations are chosen in such a way that each element of Organizational Modeling  differ by shape. A legend has been provided with the modeling notation as per the guidelines mentioned in the article \cite{Moody2009}. The same study \cite{Moody2009} presents the importance of shapes in playing a primary role to discriminate between different elements. Hence the organizational model notations are represented through individual shapes like rectangle, double circle, elliptic etc.,. The description of each element in the Organizational Model Notation is shown in the Table \ref{tab:notations}. 

\begin{center}
	\begin{longtable}{p{3cm}p{10cm}p{3cm}}
		\toprule 
		\textbf{Element} & \textbf{Definition} & \textbf{Notation} \\
		\midrule
		\endfirsthead
		Intention 			& Intentions are purposeful concrete steps taken to achieve expected outcomes . They reflect the actual intention of an organization.Intentions are defined hierarchically, which can contain and extend intentions.It is depicted by a double circle. The sub-intentions are refined starting from main intention. intentions associated with capabilities are concrete intentions and intentions that are not associated with capabilities are abstract intentions. Abstract intentions are represented by dashed double circle and concrete intentions are depicted by a solid double circle. & \begin{center} \includegraphics[width= 0.07\textwidth]{intentions.png}  \end{center}  \\  
		
		Capabilities	& Organizational capability is the ability to provide business values like software applications, resources, and potential of the actor to make decisions even in changing situations \cite{Stirna2012}. Capabilites are represented by a elliptical circle. Capability is an ability that should be possessed by an Actor or a Resource that work towards achievement of intention.   & \begin{center} \includegraphics[width= 0.1\textwidth]{capabilities.png} \end{center}   \\
		
		Context				& The environment that forms the setting for an event, statement, or idea, and in terms of which it can be fully understood. There are two Contexts: Initial and Final. The Initial Context is the situation which describes the driving forces that trigger the process to start. The Final Context is the expected situation once the process has finished.Both initial and final context are represented by an hexagonal shape except the final context has thick edges than initial context.  & \begin{center} \includegraphics[width= 0.1\textwidth]{context.png} \end{center}  \\
		
		
		Strategy		&  A method or plan chosen to bring about a desired future, such as accomplishment of a intention. Strategies are expressed by rectangles with sharp edges. In the conceptual Organizational Modeling, strategies are self-contained and loosely coupled elements.   & \begin{center} \includegraphics[width= 0.1\textwidth]{strategy.png} \end{center}   \\
		
		Resources					& The people and tools needed to fulfill the middle objectives or those/that work towards the achievement of intention . Resources are represented by a rounded rectangle. Resources are linked to capabilities and actors. & \begin{center} \includegraphics[width= 0.1\textwidth]{resources.png} \end{center}   \\
		
		Actors					& People who participate in the process. Actors are represented by a stick-man and they are linked to resource as actors can be resources. Actors define the strategy and intentions.  & \begin{center} \includegraphics[width= 0.07\textwidth]{actor.png} \end{center}   \\
		
		Relationship				& A relationship is used specify the fixed links between the elements of the model. Relationship between two elements is represented by a single direction line which represents a sequence.  & \begin{center} \includegraphics[width= 0.1\textwidth]{relationship.png} \end{center}   \\
		
		
		\bottomrule
		\caption{Informal Process Modeling Notation}
		\label{tab:notations}		
	\end{longtable}	
\end{center}


%%%%%%%%%%%%%%%%%%%%%%%%%%%%%%%%%%%%%%%%%%%%%%%%%%%%%%%%%%%%%%%%%%%%%%%%%
\section{Organizational Modeling Notation Example}
\label{sec:orgmodelnotationexample}
%%%%%%%%%%%%%%%%%%%%%%%%%%%%%%%%%%%%%%%%%%%%%%%%%%%%%%%%%%%%%%%%%%%%%%%%%
\hspace{4ex} The concept of Organizational Model Notations can be explained with the following manufacturing scenario. ABC Ltd. is a budding computer technology company which designs, develops, manufactures and sells personal computers, tablets and laptops. The CEO's intention of the quarter is to increase the revenue and number of unit sales. The initial context describes the situation that motivates to start the process. The final context describes the situation that is achieved once the process completed successfully. intentions connect initial context definitions with final context definitions \cite{Sungur2014a}. The sub-intentions are the intermediate intentions which describes the expected outcome in a measurable form. intentions are reached through strategy implementation which is plan of action designed to meet a intention. 

\hspace{4ex} The example scenario ABC Ltd. helps in understanding the organizational modeling i.e., how organization's higher level intention can be achieved by amalgamation of specific, measurable and realistic sub-intentions. . The whole view has been divided into intention view and Strategy view. The \textit{intention View} shown in the Figure\ref{fig:intentionview} provides only the details of intention and its associated strategies. There can be multiple strategies followed to achieve a intention. The \textit{Strategy View} shown in the Figure\ref{fig:strategyview} connects big picture of each strategy with individual intentions that has to be carried out. In Organizational Process Modeling, strategies are self-contained and loosely coupled. So that when we extract only the strategies from Organization Process Modeling it would be similar to Informal Process Essential Modeling. 

\hspace{4ex} The Strategy view  in the Figure\ref{fig:strategyview} depicts big picture of each strategy. Strategies are associated with both intentions and capabilities. Capabilities are related to intentions and resources. As each intention needs certain capability to successfully execute the intention they both are connected using the verb \textit{"requires"}. Resources are the potential holder of the capability i.e., to satisfy a capability we need resources. The capability and its associated resources are linked using the verb \textit{"satisfied-by"}. 


\begin{figure}
	\centering
	\includegraphics[width=\textwidth]{StrategyView.pdf}
	\caption{Strategy View}
	\label{fig:strategyview}
\end{figure}

\begin{figure}
	\centering
	\includegraphics[width=\textwidth]{goalView.pdf}
	\caption{intention View}
	\label{fig:intentionview}
\end{figure}

%%%%%%%%%%%%%%%%%%%%%%%%%%%%%%%%%%%%%%%%%%%%%%%%%%%%%%%%%%%%%%%%%%%%%%%%%
\section{Organizational Modeling Process Representation}
\label{sec:orgmodelprocessrepresentation}
%%%%%%%%%%%%%%%%%%%%%%%%%%%%%%%%%%%%%%%%%%%%%%%%%%%%%%%%%%%%%%%%%%%%%%%%%
\begin{figure}
	\centering
	\includegraphics[width=\textwidth]{processmodeling.pdf}
	\caption{Process Modeling Diagram}
	\label{fig:processdiagram}
\end{figure}

\hspace{4ex} Organizational Process Modeling depicted inFigure \ref{fig:processdiagram} captures required organizational capabilities that are satisfied by resource models  to enable the achievement of organizational intentions in certain context definitions through a strategy. It is a top-down approach, i.e., first intentions are defined and then sub-intentions  are defined by refining main intention. intentions connect initial context definitions with final context definitions through a strategy.  To understand the definition of Organizational Process Modeling we need to interpret the Organizational Process Modeling Representation shown in Figure\ref{fig:processdiagram}. 

\hspace{4ex} The Organizational Process Modeling start with modeling of organizational intention (M1). Once the intention has been modeled, the second step is to model the strategies which can be a multi-instance strategy model(M2). The next step is to model the context definitions (M3.1), required organizational capabilities (M3.2) and refining the sub-intentions from main intention in parallel. Once the required capabilities(M4) are matched by required resources(P1), modeling of resources(M5) can be done.  Based on created resource models (M5) and modeled context definitions(M3.1), strategies can be executed. The organizational intentions would be iteratively updated supported to strategy execution.  


%%%%%%%%%%%%%%%%%%%%%%%%%%%%%%%%%%%%%%%%%%%%%%%%%%%%%%%%%%%%%%%%%%%%%%%%%
\section{Organizational Modeling Entity Representation}
\label{sec:orgmodelrepresentation}
%%%%%%%%%%%%%%%%%%%%%%%%%%%%%%%%%%%%%%%%%%%%%%%%%%%%%%%%%%%%%%%%%%%%%%%%%

\begin{figure}
	\centering
	\includegraphics[width=\textwidth]{entity.pdf}
	\caption{Organizational Modelling Meta-Model}
	\label{fig:metamodel}
\end{figure}

\hspace{4ex} The conceptual entity model of intentions is shown in the \ref{fig:metamodel}. This model shows that top level intention is refined into sub-intentions. A intention can be achieved through a strategy which is a plan of action designed to meet a intention. It also describes a set of interrelated resources which work together to achieve a collective intention. As reported by Sungur et al. \cite{Sungur2014a}, the concept of IPE provides an agent-based approach i.e., human performers are considered as agents who execute the processes autonomously. Based on the approach \cite{Sungur2014a} we provide a intention-oriented approach based on intentions.

\hspace{4ex} Organizational Process Modeling  has \textit{Resources} which are used to achieve the intentions. Organizational Process Modeling is Resource-centric approach as they support processes by providing required resources and thrives to successfully execute the processes by using qualified autonomous agents, i.e., actors under certain \textit{context definitions}.  Resources can be anything like people, IT tools, data that are used to accomplish the objectives.Emerging intentions can result in the requirement of new capabilities, i.e., resources. A more specific type of resource is the type \textit{Actor}, which typically refers to human performers who autonomously and collaboratively conclude an organizational process using other available Organizational Process Modeling Resources.Actors work towards the intentions defined in the process. Resource models are optional to make precise definitions of resources needed.

\hspace{4ex} In Sungur et al \cite{Sungur2014a} work, the concept of \textit{Informal Process Support Model} IPSM has been introduced which is to make use of existing knowledge of human performers. Here the initial creator of the model is experienced human performers. Based on their experience, they add relevant  resources of an informal process. Each of the resources has inter relationships among the resources themselves. The models are generated at runtime based on the interactions and activities of corresponding human performers. 

\hspace{4ex} An informal process targets for accomplishment of a intention. The intentions can be refined by defining sub-intentions, which can be defined recursively as independent informal processes. The intention-based approach enables describing processes declaratively, i.e., without describing \textit{how} the intention is achieved, and providing only information about \textit{what} is achieved. Thus, to avoid predefined business logic in the representations of informal processes. 

\hspace{4ex} Each informal process starts from an initial context, i.e., \textit{IPE Context} and aims to achieve a intention. After accomplishing the intention, there is a resulting context called as final context. Each Resource can be related to another Resource in the context of an informal process using predefined or custom \textit{Relationships}.


\cleardoublepage
\chapter{Design}
\label{chap:design}

%of the system: Class diagrams (components), ER, algorithms if there are any

%%%%%%%%%%%%%%%%%%%%%%%%%%%%%%%%%%%%%%%%%%%%%%%%%%%%%%%%%%%%%%%%%%%%%%%%%
\section{System Architecture}
\label{sec:sysarch}
%%%%%%%%%%%%%%%%%%%%%%%%%%%%%%%%%%%%%%%%%%%%%%%%%%%%%%%%%%%%%%%%%%%%%%%%%


%%%%%%%%%%%%%%%%%%%%%%%%%%%%%%%%%%%%%%%%%%%%%%%%%%%%%%%%%%%%%%%%%%%%%%%%%
\section{Entity Type Relation}
\label{sec:enttyperelation}
%%%%%%%%%%%%%%%%%%%%%%%%%%%%%%%%%%%%%%%%%%%%%%%%%%%%%%%%%%%%%%%%%%%%%%%%%


%%%%%%%%%%%%%%%%%%%%%%%%%%%%%%%%%%%%%%%%%%%%%%%%%%%%%%%%%%%%%%%%%%%%%%%%%
\section{User Interface Diagram}
\label{sec:uidiagram}
%%%%%%%%%%%%%%%%%%%%%%%%%%%%%%%%%%%%%%%%%%%%%%%%%%%%%%%%%%%%%%%%%%%%%%%%%

%%%%%%%%%%%%%%%%%%%%%%%%%%%%%%%%%%%%%%%%%%%%%%%%%%%%%%%%%%%%%%%%%%%%%%%%%
\section{Design Methodology}
\label{sec:designmethodology}
%%%%%%%%%%%%%%%%%%%%%%%%%%%%%%%%%%%%%%%%%%%%%%%%%%%%%%%%%%%%%%%%%%%%%%%%%

On the left list we should have all organizational context definitions and on the right one only ones that are contained in an informal process. The dropdown box of the initial and final context defines the selection inside of an informal process, thus right side.

So, in db.cljs, we should have only a list of context definitions no :initial-contexts and :desired-final-contexts. Only :organizational-contexts and under this all available contexts. Under the left list, we present these elements. Right list should refer to the initial-context, final-contexts, etc. of the informal process model depending on the selection of the dropbox button. For instance if we have initial contexts selected on the dropdown box, we should present the initial contexts in the right list.

I have changed the code accordingly and provided you an example how you should change data from views.cljs. All data should be stored in db.cljs. This applies to the text fields of all elements. Whenever, we want to update something we need to update the map in db.cljs and this will be propagated to the views.


On the left side of each list item, you should present all available items of context definitions or intentions whatever type is selected there. On the right side only the ones contained in the respective informal process model. Inside of another entity, you should refer to other entities using their ids and these ids should be resolved using, for instance, intentions vector. You check each intention in the intentions vector, if it’s id is the same as the id you are looking for it, you found it and you use the information about it.  


Please align it with the structure and names of the IPSM.xsd. Each variableName like this is written like variable-name. Each complex type is a map each attribute is a key value pair and each element in another element is another key value pair.

\begin{Listing}
	\begin{lstlisting}
:entity-data {:informal-process-definitions [IPD1 IPD2]
	          :context-definitions [CTX1 CTX2 CTX3]
	          :intentions [INT1 INT2]} 
	\end{lstlisting}
	\caption{Entity data definition inside db.cljs}
	\label{lst:entitydatalist}
\end{Listing}


\cleardoublepage
\chapter{Realization}
\label{chap:realization}
\cleardoublepage
\chapter{Validation and Evaluation}
\label{chap:evaluation}

\section{Validation of Components}
\cleardoublepage
\chapter{Conclusion and Future Work}
\label{chap:conclusion}
Hier bitte einen kurzen Durchgang durch die Arbeit.

\section*{Future Work}
...und anschließend einen Ausblick  yfgfdgdf \cite{WSPA}


%
%
%\renewcommand{\appendixtocname}{Anhang}
%\renewcommand{\appendixname}{Anhang}
%\renewcommand{\appendixpagename}{Anhang}
\appendix
%\input{content/latex-tipps}

%\printindex

\printbibliography

\ifdeutsch
All links were last followed on May 11, 2016.
\else
All links were last followed on May 11, 2016.
\fi

\pagestyle{empty}
\renewcommand*{\chapterpagestyle}{empty}
%\Versicherung
\end{document}
